\documentclass[../main.tex]{subfiles}

\begin{document}

\chapter{Metric Spaces}

\section{Metric Spaces}
\begin{definition}{Metric Spaces}{Metric Spaces}
	A \textbf{metric space} is a set $X$ together with a function $d: X \times X \to \mathbb{R}$, called a \textbf{metric}, that satisfies the following properties for all $x, y, z \in X$:
	\begin{enumerate}
		\item \textbf{Non-negativity:} $d(x, y) \geq 0$.
		\item \textbf{Identity of indiscernibles:} $d(x, y) = 0$ if and only if $x = y$.
		\item \textbf{Symmetry:} $d(x, y) = d(y, x)$.
		\item \textbf{Triangle inequality:} $d(x, z) \leq d(x, y) + d(y, z)$.
	\end{enumerate}
\end{definition}

\begin{example}{Metric Spaces}{Metric Spaces}
\begin{itemize}
	\item The Euclidean space $\mathbb{R}^n$ with the standard metric $d(x, y) = \|x - y\|_2$, where $\| \cdot \|_2$ is the Euclidean norm.
		\begin{equation*}
			d(x, y) = \sqrt{\sum_{i=1}^n (x_i - y_i)^2}
		\end{equation*}
	\item The bounded sequence space $\ell^{\infty }$, consisting of all bounded sequences of complex numbers, or $\ell^{\infty } = \left\{ x = (x_1, x_2, \ldots) \in \mathbb{C}^\infty : \sup_{i} |x_i| < \infty \right\}$, with the metric defined by
		\begin{equation*}
			d(x, y) = \sup_{i} |x_i - y_i|
		\end{equation*}
	\item The convergent sequence space $c$, consisting of all sequences of complex numbers that converge to a limit, with the metric defined by
		\begin{equation*}
			d(x, y) = \sup_{i} |x_i - y_i|
		\end{equation*}
	\item The function space $C[a,b]$, consisting of all continuous functions on the interval $[a, b]$, with the metric defined by
		\begin{equation*}
			d(f, g) = \max_{x \in [a, b]} |f(x) - g(x)|
		\end{equation*}
	\item The discrete metric space, where the metric is defined as
		\begin{equation*}
			d(x, y) = 
			\begin{cases} 
				0 & \text{if } x = y \\ 
				1 & \text{if } x \neq y 
			\end{cases}
		\end{equation*}
		(This can be thought of an $n$-dimensional tetrahedron with edge length 1 with the Euclidean metric.)
	\item Hamming metric, which is used in coding theory, defined for two strings of equal length as the number of positions at which the corresponding symbols are different. For example, for two binary strings $x$ and $y$ of length $n$, the Hamming distance is given by
		\begin{equation*}
			d(x, y) = \sum_{i=1}^n \mathbb{1}_{\{x_i \neq y_i\}}
		\end{equation*}
		where $\mathbb{1}_{\{\cdot\}}$ is the indicator function.
	\item The sequence space $s$, consisting of all sequences of complex numbers. The metric is defined by
		\begin{equation*}
			d(x, y) = \sum_{i=1}^\infty \frac{1}{2^i} \frac{|x_i - y_i|}{1 + |x_i - y_i|}
		\end{equation*}
	\item The bounded function space $B(X)$, consisting of all bounded functions from a set $X$ to $\mathbb{C}$. The metric is defined by
		\begin{equation*}
			d(f, g) = \sup_{x \in X} |f(x) - g(x)|
		\end{equation*}
	\item The space $\ell^p$, for $1 \leq p < \infty$, consisting of all sequences of complex numbers $x = (x_1, x_2, \ldots)$ such that $\sum_{i=1}^\infty |x_i|^p < \infty$. The metric is defined by
		\begin{equation*}
			d(x, y) = \left( \sum_{i=1}^\infty |x_i - y_i|^p \right)^{1/p}
		\end{equation*}
		When $p = 2$, this space is known as the Hilbert sequence space $\ell^2$.
\end{itemize}
\end{example}

\begin{proposition}{Some Inequalities}{Some Inequalities}
\begin{itemize}
	\item The Young's inequality: For $a, b \geq 0$ and $p, q > 1$ such that $\frac{1}{p} + \frac{1}{q} = 1$, we have
		\begin{equation*}
			ab \leq \frac{a^p}{p} + \frac{b^q}{q}
		\end{equation*}
	\item The H\"older's inequality: For sequences $x = (x_1, x_2, \ldots)$ and $y = (y_1, y_2, \ldots)$ in $\ell^p$ and $\ell^q$ respectively, where $1 < p, q < \infty$ and $\frac{1}{p} + \frac{1}{q} = 1$, we have
		\begin{equation*}
			\sum_{i=1}^\infty |x_i y_i| \leq \left( \sum_{i=1}^\infty |x_i|^p \right)^{1/p} \left( \sum_{i=1}^\infty |y_i|^q \right)^{1/q}
		\end{equation*}
		when $p=q=2$, this reduces to the Cauchy-Schwarz inequality:
		\begin{equation*}
			\sum_{i=1}^\infty |x_i y_i| \leq \left( \sum_{i=1}^\infty |x_i|^2 \right)^{1/2} \left( \sum_{i=1}^\infty |y_i|^2 \right)^{1/2}
		\end{equation*}
	\item The Minkowski inequality: For sequences $x = (x_1, x_2, \ldots)$ and $y = (y_1, y_2, \ldots)$ in $\ell^p$, where $1 \leq p < \infty$, we have
		\begin{equation*}
			\left( \sum_{i=1}^\infty |x_i + y_i|^p \right)^{1/p} \leq \left( \sum_{i=1}^\infty |x_i|^p \right)^{1/p} + \left( \sum_{i=1}^\infty |y_i|^p \right)^{1/p}
		\end{equation*}
\end{itemize}
\end{proposition}

\section{Convergence and Completeness}

\begin{definition}{Convergence}{Convergence}
	A sequence $(x_n)$ in a metric space $(X, d)$ is said to \textbf{converge} to a point $x \in X$ if for every $\epsilon > 0$, there exists an integer $N$ such that for all $n \geq N$, we have $d(x_n, x) < \epsilon$. This is denoted as $x_n \to x$ as $n \to \infty$.
	\begin{equation*}
		\lim_{n \to \infty} x_n = x
	\end{equation*}
\end{definition}

If $(X,d)$ is a metric space, we say that a sequence $(x_n)$ in $X$ is \textbf{bounded} if there exists a constant $M > 0$ such that for all $n$, $d(x_n, x_0) < M$ for some fixed point $x_0 \in X$.
\begin{itemize}
\item (Uniqueness of limits) If a sequence converges, it is bounded and its limit is unique.
\item If $x_n \rightarrow x$ and $y_n \rightarrow y$, then $d(x_n, y_n) \rightarrow d(x, y)$.
\end{itemize}

\begin{definition}{Cauchy's Sequence and Completeness}{Cauchys Sequence and Completeness}
	A sequence $(x_n)$ in a metric space $(X, d)$ is called a \textbf{Cauchy sequence} if for every $\epsilon > 0$, there exists an integer $N$ such that for all $m, n \geq N$, we have $d(x_m, x_n) < \epsilon$.
	\begin{equation*}
		\forall \epsilon > 0, \exists N \in \mathbb{N} : \forall m, n \geq N, d(x_m, x_n) < \epsilon
	\end{equation*}
	A metric space is said to be \textbf{complete} if every Cauchy sequence in the space converges to a limit in the space.
\end{definition}
A convergent sequence is a Cauchy sequence, but the converse is not necessarily true. 

\begin{theorem}{Closure in Metric Spaces}{Closure in Metric Spaces}
	Let $(X, d)$ be a metric space and let $A \subseteq X$ is not empty.
	\begin{itemize}
	\item $x\in \overline{M}$ iff there is a sequence $(x_n)$ in $M$ such that $x_n \to x$.
	\item $M$ is closed iff every convergent sequence in $M$ converges to a point in $M$. That is, $M = \overline{M}$.
	\end{itemize}
\end{theorem}

\begin{theorem}{Subspace of a Complete Space}{Subspace of a Complete Space}
	Let $(X, d)$ be a complete metric space and let $Y \subseteq X$ be a non-empty subset. Then $Y$ is a complete metric space with the induced metric $d_Y(x, y) = d(x, y)$ for all $x, y \in Y$ iff $Y$ is closed in $X$.
\end{theorem}

\begin{theorem}{Sequences and Continuous Mapping}{Sequences and Continuous Mapping}
	A function $f: (X, d_X) \to (Y, d_Y)$ between metric spaces is continuous at a point $x_0 \in X$ if and only if for every sequence $(x_n)$ in $X$ that converges to $x_0$, the sequence $(f(x_n))$ converges to $f(x_0)$ in $Y$.
\end{theorem}

\section{Completeness Proofs}
\paragraph{Completeness of $\mathbb{R}$ and $\mathbb{C}$} The spaces $\mathbb{R}$ and $\mathbb{C}$ are complete metric spaces with the standard metric.

As each component is a Cauchy sequence, and $\mathbb{C}$ is complete, each component converges. Thus, the sequence converges in $\mathbb{C}$ to the limit constructed from the limits of the components.

\paragraph{Completeness of $\mathbb{R}^n$ and $\mathbb{C}^n$} The spaces $\mathbb{R}^n$ and $\mathbb{C}^n$ are complete with the metric defined by $d(x, y) = \|x - y\|_2$, where $\| \cdot \|_2$ is the Euclidean norm.

\paragraph{Completeness of $\ell^p$} The space $\ell^p$ is complete with the metric defined by $\displaystyle d(x, y) = \left( \sum_{i=1}^\infty |x_i - y_i|^p \right)^{1/p}$ for $1 \leq p < \infty$.

\paragraph{Completeness of $\ell^{\infty }$} The space $\ell^{\infty }$ is complete with the metric defined by $d(x, y) = \sup_{i} |x_i - y_i|$.

\begin{proof}
	Let $(x_m)$ be any Cauchy sequence in $\ell^{\infty }$, and $x_m = (\xi_1^{(m)}, \xi_2^{(m)}, \ldots)$. For each fixed $i$, the sequence $(\xi_i^{(m)})_{m=1}^\infty$ is a Cauchy sequence in $\mathbb{C}$, since there is an $N\in \mathbb{Z}_+$, such that for all $m, n \geq N$, we have
	\begin{equation*}
		|\xi_i^{(m)} - \xi_i^{(n)}| \leq d(x_m, x_n) < \epsilon
	\end{equation*}
	From the completeness of $\mathbb{C}$, each sequence $(\xi_i^{(m)})$ converges to some limit $\xi_i \in \mathbb{C}$. Define $x = (\xi_1, \xi_2, \ldots)$. We need to show that $x \in \ell^{\infty }$ and that $x_m \to x$ in $\ell^{\infty }$.
	\begin{itemize}
		\item To show that $x \in \ell^{\infty }$, we note that there exists a constant $k_m>0$ that bounds $x_m$. Hence, we have
			\begin{equation*}
				|\xi_j| \leq |\xi_j-\xi_j^{(m)}| + |\xi_j^{(m)}| \leq \epsilon + k_m, \qquad m \geq N
			\end{equation*}
			for all $j$. Thus, $x$ is bounded and hence $x \in \ell^{\infty }$.
		\item As
			\begin{equation*}
				d(x_m, x) = \sup_{i} |\xi_i^{(m)} - \xi_i| < \epsilon, \qquad m \geq N
			\end{equation*}
			we have $x_m \to x$ in $\ell^{\infty }$.
	\end{itemize}
\end{proof}


\paragraph{Completeness of $c$} The space $c$ is complete with the metric defined by $d(x, y) = \sup_{i} |x_i - y_i|$.

\begin{proof}
	We shall prove that $c$ is closed in $\ell^{\infty }$. Let $(x_m)$ be a sequence in $c$ that converges to some $x \in \ell^{\infty }$. This is the uniform convergence theorem.
\end{proof}

\paragraph{Completeness of $C[a,b]$} The space $C[a,b]$ is complete with the metric defined by $d(f, g) = \max_{x \in [a, b]} |f(x) - g(x)|$.

\begin{proof}
Uniform limit of continuous functions is continuous.
\end{proof}
\begin{remark}
	This metric is called the uniform metric, and the corresponding convergence is called uniform convergence.
\end{remark}

\begin{example}{Incomplete Spaces}{Incomplete Spaces}
	\begin{itemize}
		\item The rational numbers $\mathbb{Q}$ with the standard metric inherited from $\mathbb{R}$ is not complete. For example, the sequence defined by the decimal approximations of $\sqrt{2}$ is a Cauchy sequence in $\mathbb{Q}$ that does not converge to any rational number.
		\item Continuous functions. The space $C[a,b]$ with the metric defined by $\displaystyle d(f, g) = \int_a^b |f(x) - g(x)| \, dx$ is not complete. For example, the sequence of functions $f_n(x) = x^n$ on the interval $[0, 1]$ is a Cauchy sequence in this metric but does not converge to a continuous function.
	\end{itemize}
\end{example}

\end{document}
