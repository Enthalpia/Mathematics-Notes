\documentclass[../main.tex]{subfiles}

\begin{document}

\chapter{Topological Spaces}

The concept of topological space grew out of the study of the real line and Euclidean space and the study of continuous functions on these spaces.

\section{Topological Spaces}

\begin{definition}{Topology}{Topology}
	A \emph{topology} on a set $X$ is a collection $\mathcal{T}$ of subsets of $X$ (that is, $T \subseteq P(X)$) such that the following conditions hold:
\begin{itemize}
\item $\emptyset, X \in \mathcal{T}$.
\item The union of any collection of sets in $\mathcal{T}$ is in $\mathcal{T}$.
\item The intersection of any finite collection of sets in $\mathcal{T}$ is in $\mathcal{T}$.
\end{itemize}

A topological space is a pair $(X, \mathcal{T})$ where $X$ is a set and $\mathcal{T}$ is a topology on $X$.
\end{definition}

\begin{example}{Topologies}{Topologies}
\begin{itemize}
\item The \emph{discrete topology} on a set $X$ is the topology $\mathcal{T} = P(X)$, where $P(X)$ is the power set of $X$. In this topology, every subset of $X$ is open.
\item The \emph{indiscrete topology}, or trivial topology, on a set $X$ is the topology $\mathcal{T} = \left\{ \emptyset , X \right\}$. In this topology, only the empty set and the entire set are open.
\item Topologies on $\mathbb{R}$.
	\begin{enumerate}
		\item $\mathcal{T}_1$ consisting of $\mathbb{R}$, $\emptyset $, and all open intervals $(a,b)$.
		\item $\mathcal{T}_2$ consisting of $\mathbb{R}$, $\emptyset $, and all $\left[-n,n\right]$ for $n\in \mathbb{Z}_+$.
	\end{enumerate}
\item Topologies on $\mathbb{N}$.
	\begin{enumerate}
		\item The initial segment topology: $\mathcal{T}_1$ consisting of $\mathbb{N}$, $\emptyset $ and the set $\left\{ 1, \ldots ,n \right\}$ for $n\in \mathbb{N}$.
		\item The final segment topology: $\mathcal{T}_2$ consisting of $\mathbb{N}$, $\emptyset $ and the set $\left\{ n, n+1 \ldots  \right\}$ for $n\in \mathbb{N}$.
	\end{enumerate}
\end{itemize}
\end{example}

\begin{definition}{Finer Topologies}{Finer Topologies}
	If $\mathcal{T}_1$ and $\mathcal{T}_2$ are topologies on a set $X$ and $\mathcal{T}_1 \subseteq \mathcal{T}_2$, then $\mathcal{T}_2$ is said to be \emph{finer} than $\mathcal{T}_1$. Similarly, $\mathcal{T}_1$ is said to be \emph{coarser} than $\mathcal{T}_2$. If $\mathcal{T}_1$ is neither finer nor coarser than $\mathcal{T}_2$, then $\mathcal{T}_1$ and $\mathcal{T}_2$ are said to be \emph{incomparable}.
\end{definition}

\section{Basis for a Topology}

Sometimes it is rather hard to explicitly describe all the elements of a given topology. Therefore, we turn to the concept of a basis: ``using fewer elements that generates the whole space'', just like what we do in linear algebra with linear spaces.

\begin{definition}{Basis}{Basis}
If $X$ is a set, a basis for a topology on $X$ is a collection $\mathcal{B}$ for subsets of $X$, i.e. $\mathcal{B} \subseteq P(X)$, such that
\begin{itemize}
\item $\forall x\in X, \exists B\in \mathcal{B}, x\in B$.
\item If $B_1,B_2\in \mathcal{B}, x\in B_1\cap B_2$, then $\exists B\in \mathcal{B}, x\in B \subseteq B_1\cap B_2$.
\end{itemize}

If $\mathcal{B}$ is a basis, we say that $\mathcal{T} = \left\{ B = \bigcup_{\alpha} B_{\alpha}, B_{\alpha}\in \mathcal{B} \right\}$ is the topology generated by $\mathcal{B}$. (Here the union index $\alpha$ can take $\emptyset $, so that $\emptyset \in \mathcal{T}$.)

A different equivalent to describe $\mathcal{T}$ is that $U\in \mathcal{T}$ iff $\forall x\in U, \exists B\in \mathcal{B}, x\in B \subseteq U$.
\end{definition}
\begin{remark}
We use the concept of ``infinite union'' closure of topology here. The $\mathcal{T}$ above is the smallest topology that contains the basis $\mathcal{B}$. Also, the second axiom of basis stands for the ``finite intersection'' closure. (The first is given for the whole space to be open.)

\vspace{5pt}
Another more general way to see basis is as follows: Let $\mathcal{T}$ be a topology on $X$, then $\mathcal{B}$ is a basis of $\mathcal{T}$ iff every $U\in \mathcal{T}$ is the union of some $B_{\alpha}\in \mathcal{B}$.
\end{remark}

\begin{figure}[ht]
    \centering
    \incfig{basis-of-topology}
    \caption{Basis of Topology}
    \label{fig:basis-of-topology}
\end{figure}

\begin{example}{Basis for Topologies}{Basis for Topologies}
We shall see that different basis can generate the same topology. For instance, in $\mathbb{R}^2$, Let $\mathcal{B}_1$ be all the open circular regions, and $\mathcal{B}_2$ be all open squares.
\end{example}

The following proof shows that the $\mathcal{T}$ defined above is indeed a topology on $X$.
\begin{proof}
\begin{itemize}
\item First $\emptyset \in \mathcal{T}$. Let $U_x\in \mathcal{B}$ contains $x$. Then $\bigcup_{x\in X}U_x =X \in \mathcal{T}$.
\item If $U_1,U_2\in \mathcal{T}$, if the intersection is not $\emptyset $, $\forall x\in U_1\cap U_2$, Take $B_1,B_2\in \mathcal{B}$ such that $x\in B_1 \subseteq U_1, x\in B_2 \subseteq U_2$, then $\exists B_x\in \mathcal{B}, x\in B_x \subseteq B_1\cap B_2 \subseteq U_1\cap U_2$, then $U_1\cap U_2 = \bigcup_{x\in U_1\cap U_2} B_x \in \mathcal{T}$.
\item The infinite union part is quite easy, just union all the index would do.
\end{itemize}
\end{proof}

We can compare topologies via basis.

\begin{lemma}{Comparison of Topology by Basis}{Comparison of Topology by Basis}
Let $\mathcal{B}$ and $\mathcal{B}'$ be basis for topologies  $\mathcal{T}$ and $\mathcal{T}'$ on $X$. Then the following are equivalent:
\begin{itemize}
\item $\mathcal{T}'$ is finer than $\mathcal{T}$, i.e. $\mathcal{T} \subseteq \mathcal{T}'$.
\item $\forall x\in X,\forall B\in \mathcal{B}$ such that $x\in B$, $\exists B'\in \mathcal{B}', x\in B' \subseteq B$. (For every $B\in \mathcal{B}, B = \bigcup_{\alpha} B'_{\alpha}$ for some $B'_{\alpha}\in \mathcal{B}'$.)
\item $\forall B\in \mathcal{B},B\in \mathcal{T}'$. (All the basis of $\mathcal{T}$ is open in $\mathcal{T}'$, then all the open sets in $\mathcal{T}$ is also open in $\mathcal{T}'$.)
\end{itemize}
\end{lemma}

\begin{example}{Criterion for Comparison}{Criterion for Comparison}
Using the above criterion we can easily show that the open disks and squares generate the same topology on $\mathbb{R}^2$.
\end{example}

Now we focus on topologies on $\mathbb{R}$.

\begin{definition}{Topologies on $\mathbb{R}$}{Topologies on mathbbR}
The following are 3 topologies on $\mathbb{R}$:
\begin{enumerate}
\item Let $\mathcal{B} = \left\{ (a,b): a<b \right\}$ be all the open intervals on $\mathbb{R}$, then the topology generated by $\mathcal{B}$ is the standard (Euclidean) topology on $\mathbb{R}$, denoted by $\mathbb{R}$.
\item The topology generated by $ \mathcal{B}' = \left\{ [a,b): a<b \right\}$ is called the lower limit topology, denoted $\mathbb{R}_{l}$.
\item Let $K = \left\{ \frac{1}{n}: n\in \mathbb{Z}_+ \right\}$, then the topology generated by $\mathcal{B}''= \mathcal{B} \cup \left\{ (a,b)-K: a<b \right\}$ is called the $K$-topology on $\mathbb{R}$, denoted by $\mathbb{R}_K$.
\end{enumerate}
\end{definition}

\begin{proposition}{Comparison of $\mathbb{R}$ topologies}{Comparison of mathbbR topologies}

\begin{itemize}
\item $\mathbb{R} \subseteq \mathbb{R}_l, \mathbb{R}_K$
\item $\mathbb{R}_l$ and $\mathbb{R}_K$ are not comparable.
\end{itemize}
\end{proposition}
\begin{proof}
\begin{itemize}
\item For every $(a,b) = \bigcup_{\epsilon>0} [a+\epsilon,b)$. And $\mathcal{B}'' \subseteq \mathcal{B}$.
\item First $[2,3)\in \mathbb{R}_l$ but $\notin \mathbb{R}_K$, also $U = (-1,1)-K\in \mathbb{R}_K$ but $\notin \mathbb{R}_l$, because $0\in U$ but $\forall 0\in [a,b)$ we have $b>0$, and $\exists n\in \mathbb{Z}_+, 1 / n \in [a,b)$, so there is no $0\in [a,b) \subseteq U$.
\end{itemize}
\end{proof}

To eliminate the need of all intersections, we introduce the concept of subbasis, which is a smaller collection that can also generate the whole topology.

\begin{definition}{Subbasis}{Subbasis}
A subbasis $\mathcal{S}$ for a topology $\mathcal{T}$ on $X$ is also a collection of subsets of $X$ such that $\mathcal{B} = \left\{ B = \bigcap_{n=1}^{N} S_n: S_n\in \mathcal{S}, N\in \mathbb{Z}_+ \right\}$ is a basis for $\mathcal{T}$.
\end{definition}
\begin{remark}
The topology $\mathcal{T}$ is generated by first taking finite intersections of $\mathcal{S}$ and then arbitrary union. The topology induced by $\mathcal{S}$ is also the smallest topology containing $\mathcal{S}$. Therefore, we still have the following criterion.
\end{remark}

\begin{theorem}{Criterion for Finer Topologies from Subbasis}{Criterion for Finer Topologies from Subbasis}
If $\mathcal{T}$ and $\mathcal{T}'$ are topologies on $X$, and $\mathcal{S}$ is a subbasis of $\mathcal{T}$. If $\mathcal{S} \subseteq \mathcal{T}'$ then $\mathcal{T} \subseteq \mathcal{T}'$.
\end{theorem}


\section{The Order Topology}

Suppose there is an order relation $<$ on $X$. The order here is the simple order:
\begin{itemize}
\item $\forall x\neq y, x<y\lor y<x$.
\item $\forall x\in X, \lnot (x<x)$.
\item $x<y\land y<z \rightarrow x<z$.
\end{itemize}
We define intervals similar to that of $\mathbb{R}$ :
\begin{itemize}
	\item $(a,b) = \left\{ x:a<x<b \right\}$, $[a,b] = \left\{ x: a\leq x\leq b \right\}$ 
	\item $(a,b]$ and $[a,b)$ are similar.
\end{itemize}

\begin{definition}{Order Topology}{Order Topology}
Let $X$ be a set with order $<$, and has more than $1$ elements. Let $\mathcal{B}$ consists of the following subsets:
\begin{itemize}
\item All $(a,b) $ in $X$.
\item All $[a_0,b)$ in $X$ if $a_0$ is the smallest element (if any) in $X$, i.e. $\forall x\in X, x \geq a_0$ 
\item All $(a,b_0]$ if $b_0$ is the largest element (if any) in $X$.
\end{itemize}

The topology generated by $\mathcal{B}$ is called the order topology.
\end{definition}

It is easy to show that $\mathcal{B}$ is indeed a basis.

\begin{example}{Order topology}{Order topology}
\begin{itemize}
\item The standard (Euclidean) topology on $\mathbb{R}$
\item \textbf{Dictionary order: } On $\mathbb{R}^2$, we say $(a,b)_p<(c,d)_p$ iff $(a<c) \lor (a=c \land b<d)$.
\item The discrete topology on $\mathbb{Z}_+$.
\end{itemize}
\end{example}

If $X$ is an ordered set, and $a$ is an element. The following four sets are called rays:
\begin{itemize}
	\item $(a, \infty ) = \left\{ x:x>a \right\}$
	\item $(-\infty ,a) = \left\{ x:x<a \right\}$
	\item $[a, \infty ) = \left\{ x:x\geq a \right\}$
	\item $(-\infty ,a] = \left\{ x:x\leq a \right\}$
\end{itemize}
The first two are open rays, while the last two are closed rays. (The openness here is the openness in order topology).

\begin{remark}
The open rays forms a subbasis of the open topology, actually.
\end{remark}


\section{The Product Topology}
If $X,Y$ are topological spaces, there is a standard way to define a topology on $X \times Y$.

\begin{definition}{Product Topology}{Product Topology}
If $X,Y$ are topological spaces, with topology $\mathcal{T}_X,\mathcal{T}_Y$. The product topology on $X \times Y$ is generated by a basis $\mathcal{B} = \left\{ U \times V: U\in \mathcal{T}_X, V\in \mathcal{T}_Y \right\}$.
\end{definition}
We should check that $\mathcal{B}$ is indeed a basis.
\begin{proof}
\begin{itemize}
	\item For $X \times Y\in \mathcal{B}$, all elements are contained in some basis.
	\item If $(x,y)_p \in (U_1 \times V_1)\cap (U_2 \times V_2) = (U_1 \cap U_2) \times (V_1\cap V_2)$, the latter is a basis element.
\end{itemize}
\end{proof}

\begin{remark}
We cannot directly define  the product topology to be $\mathcal{T} = \left\{ U \times V: U\in \mathcal{T}_X, V\in \mathcal{T}_Y \right\}$ because this only contains rectangles, but we want circles to be open as well.
\end{remark}

The next result shows that we can directly construct a basis for the product topology by a basis of $X$ and $Y$, without knowing all open sets of $\mathcal{T}_X,\mathcal{T}_Y$.

\begin{theorem}{Basis for the Product Topology}{Basis for the Product Topology}
If $\mathcal{B}_X,\mathcal{B}_Y$ are basis for $X$ and $Y$. Then the collection
\begin{equation*}
	\mathcal{B} = \left\{ B_x \times B_y, B_x\in \mathcal{B}_X, B_y\in B_Y \right\}
\end{equation*}
is a basis for the product topology of $X \times Y$.
\end{theorem}

\begin{remark}
We have a standard topology (The order topology or Euclidean topology) on $\mathbb{R}$, so we have the standard topology on $\mathbb{R}^2$ simply by the product topology, with the basis $(a,b) \times (c,d)$, i.e., all open rectangles.
\end{remark}

To describe a product topology with subbasis, we need projection function:

\begin{definition}{Projection}{Projection}
	The function $\pi_1: X \times Y \rightarrow X$, $\pi_1(x,y)_p = x$ and $\pi_2: X \times Y \rightarrow Y$, $\pi_2(x,y)_p = y$ are projections from $X \times Y$ onto $X$ or $Y$.
\end{definition}

If $U\in \mathcal{T}_X$, then $\pi_1^{-1}(U) = U \times Y$, which is open in the product topology on $X \times Y$. Similarly, $\pi_2^{-1}(V) = X \times V$ is also open. The intersection is the open rectangle $U \times V$.

\begin{figure}[ht]
    \centering
    \incfig{subbasis-by-projection}
    \caption{Subbasis by Projection}
    \label{fig:subbasis-by-projection}
\end{figure}

\begin{theorem}{Subbasis by Projection}{Subbasis by Projection}
\begin{equation*}
\mathcal{S} = \left\{ \pi_1^{-1}(U): U\in \mathcal{T}_X \right\} \cup \left\{ \pi_2^{-1}(V): V\in \mathcal{T}_Y \right\}
\end{equation*}
is a subbasis for the product topology on $X \times Y$.
\end{theorem}

\section{Subspace Topology}

\begin{definition}{Subspace Topology}{Subspace Topology}
Let $(X,\mathcal{T})$ be a topological space, $Y \subseteq X$ be any subset. The collection
\begin{equation*}
\mathcal{T}_Y = \left\{ Y \cap U : U\in \mathcal{T} \right\}
\end{equation*}
is a topology on $Y$
\end{definition}

It is easy to show that $\mathcal{T}_Y$ is a topology, we can use the same method to construct its basis:
\begin{lemma}{Basis for Subspace Topology}{Basis for Subspace Topology}
If $\mathcal{B}$ is a basis for the topology of $X$ then the collection
\begin{equation*}
\mathcal{B}_Y = \left\{ B\cap Y: B\in \mathcal{B} \right\}
\end{equation*}
is a basis for the subspace topology $\mathcal{T}_Y$.
\end{lemma}

\begin{remark}
Openness of a set does not preserve in $\mathcal{T}$ and $\mathcal{T}_Y$, BE CAREFUL! However, if $Y$ is open in $X$, them every open set in $Y$ is also open in $X$.
\end{remark}

\begin{lemma}{Preserve of Openness in Open Subspaces}{Preserve of Openness in Open Subspaces}
If $Y$ is open in $X$, them every open set in $Y$ is also open in $X$.
\end{lemma}

\begin{theorem}{Exchange of Subspace and Product Topology}{Exchange of Subspace and Product Topology}
If $A \subseteq X, B \subseteq Y$, then the following is equal:
\begin{itemize}
\item The subspace topology of $A \times B$ of the product topology of $ X \times Y$.
\item The product topology of subspace topology of $A \subseteq X$ and $B \subseteq Y$.
\end{itemize}
\end{theorem}
\begin{proof}
	If $\mathcal{B}_X = \left\{ B_x \right\}$ and $\mathcal{B}_Y=\left\{ B_y \right\}$ are basis of $X$ and $Y$, then $\left\{ (B_x\cap A) \times (B_y \cap B) \right\} = \left\{ (B_x \times  B_y) \cap (A \times B) \right\}$ are basis for $A \times B$ in both situation.
\end{proof}

It is shown that we can exchange the order of subspace topology and product topology, but the following example shows that order topology and subspace topology may not be exchangeable.

\begin{example}{Subspace Topology and Order Topology}{Subspace Topology and Order Topology}
Let $(X,\mathcal{T})$ be a topological space with order topology, and $Y \subseteq X$. The order topology on $Y$ MAY NOT BE the subspace topology.

\begin{itemize}
\item Consider $X = \mathbb{R}$ and $Y = [0,1]$. In this case, the subspace topology and order topology are the same.
\item Let $X = \mathbb{R}$ and $Y = [0,1)\cup \left\{ 2 \right\}$. The set $\left\{ 2 \right\}$ is open in the subspace topology but not in the oeder topology, which requires the form
	\begin{equation*}
	\left\{ x\in Y: a\in Y, a<x\leq 2 \right\}
	\end{equation*}
	(It is also clear that $Y \cong [0,1]$ in the order sense.)
\item Let $X = \mathbb{R}^2$ with the dictionary order, and $Y = [0,1] \times [0,1]$. Then $\left\{ 1 / 2 \right\}\cup [0,1]$ is open in the subspace topology but not in the order topology
\end{itemize}
\end{example}

The dictionary order in $I^2 = [0,1] \times [0,1]$ is called the ordered square, denoted $I^2_o$.

\begin{remark}
Taking a closer look, the problem occurs when there are ``break points'' in $Y$, which introduce us to the concept of convexity.
\end{remark}

\begin{definition}{Convexity}{Convexity}
Given an ordered set $X$, we say $Y \subseteq X$ is convex iff
\begin{equation*}
\forall a,b\in Y,a<b, \left\{ x\in X,a<x<b \right\} \subseteq Y
\end{equation*}
\end{definition}
Be careful that we must use $(a,b)$ in $X$ to verify convexity.

\begin{theorem}{Exchange of Ordered Topology and Subspace Topology}{Exchange of Ordered Topology and Subspace Topology}
Of $X$ is an ordered set and $Y \subseteq X$ is convex in $X$. Then the order topology on $Y$ is the same as the subspace topology.
\end{theorem}
\begin{proof}
	\begin{itemize}
	\item Consider in $X$ the ray $(a,+\infty )$, and $U = (a, +\infty )\cap Y$.

	If $a\in Y$, then $U = \left\{ x\in Y: x>a\in Y \right\}$ is just an open ray in $Y$. If $a\notin Y$, for $Y$ is convex, $a$ is either a lower bound or an upper bound,  $U=Y$ or $U=\emptyset $, both open.

	Then $(-\infty ,b)\cap Y$ is also open in order topology. As $(a,+\infty )\cap Y$ and $(-\infty ,b)\cap Y$ forms a subbasis of subspace topology, and are open in order topology, we have subspace topology $\subseteq $ order topology.
	\item For the convexity of Y, the open ray  $\left\{ x>a: x\in Y \right\} = \left\{ x>a: x\in X \right\}\cap Y$. So any open ray in the order topology is still open in the subspace topology.
	\end{itemize}
\end{proof}

!!NOTE: To avoid ambiguity, if $Y \subseteq X$, the topology on $Y$ is always the subspace topology unless explicitly stated.

\section{Closed Sets and Limit Points}
\subsection{Closed Sets}
\begin{definition}{Closed Sets}{Closed Sets}
A subset $A \subseteq X$ is closed iff $X-A$ is open.
\end{definition}

Closed sets have dual status to open sets, which means that we can also fully describe a topological space by closed sets.

\begin{theorem}{Basic Properties of Closed Sets}{Basic Properties of Closed Sets}
Let $X$ be a topological space.
\begin{itemize}
\item $\emptyset $ and $X$ are closed.
\item Arbitrary intersections of closed sets are closed.
\item Finite unions of closed sets are closed.
\end{itemize}
\end{theorem}

Just like open sets, the closed sets in the subspace topology is given by intersection of a closed set with the subset.

\begin{theorem}{Closed sets in the Subspace Topology}{Closed sets in the Subspace Topology}
Let $Y$ be a subspace of $X$, Then $A \subseteq Y$ is closed iff $A = B \cap Y$ for some closed set $B \subseteq X$.
\end{theorem}
\begin{proof}
 $A$ is closed iff $Y-A$ is open in $Y$ iff $Y-A = (X-B)\cap Y$ for some $B \subseteq X$ that $X-B$ is open in $X$.
\end{proof}

The following is analogous to the open set case.

\begin{theorem}{Criterion for Closed Sets}{Criterion for Closed Sets}
Let $Y \subseteq X$ is closed. If $A$ is closed in $Y$, then $A$ is closed in $X$.
\end{theorem}

\subsection{Closure and Interior}
\begin{definition}{Closure and Interior}{Closure and Interior}
Given a subset $A \subseteq X$.
\begin{itemize}
\item The \textbf{interior} of $A$ is the union of all open sets contained in $A$, denoted $\Int A$.
	\begin{equation*}
	\Int A = \bigcup \left\{ U: U\in \mathcal{T}, U \subseteq A \right\} 
	\end{equation*}
\item The \textbf{closure} of $A$ is the intersection of all closed sets containing $A$, denoted $\overline{A}$.
	\begin{equation*}
	\overline{A} = \bigcap \left\{ U: X-U\in \mathcal{T}, A \subseteq U\right\} 
	\end{equation*}
\end{itemize}
\end{definition}

A first observation shows that $\Int A$ is open and $\overline{A}$ is closed. Furthermore,
\begin{equation*}
\Int A \subseteq A \subseteq \overline{A}
\end{equation*}
\begin{remark}
The interior is the largest open set contained in $A$, and the closure is  the smallest closed set containing $A$.

If $A$ is open then $A = \Int A$, also if $A$ is closed, $A = \overline{A}$.
\end{remark}

\begin{theorem}{Closure in Subspace}{Closure in Subspace}
Let $Y$ be a subspace of $X$, let $A \subseteq Y$, $\overline{A}$ is the closure of $A$ in $X$. Then the closure of $A$ in $Y$ is $\overline{A}\cap Y$.
\end{theorem}
\begin{proof}
Let $B$ be the closure of $A$ in $Y$. Then  because $\overline{A}$ is closed in $X$, $\overline{A}\cap Y$ is closed in $Y$, then for $A \subseteq Y \subseteq \overline{A}\cap Y$, we have $B \subseteq \overline{A}\cap Y$.

The other side, for $B$ is closed in $Y$, then $B = C \cap Y$ for some closed $C$ in $X$. For $A \subseteq C$, then $\overline{A} \subseteq C$, then $\overline{A}\cap Y \subseteq C\cap Y = B$.
\end{proof}

\begin{notation}{Intersect}{Intersect}
We say that $A$ intersects $B$ if $A\cap B\neq \emptyset $.
\end{notation}

The definition gives us no way to actually find the closure of a set without finding all $A \subseteq U$. A basis is helpful here.

\begin{theorem}{Criterion for Closure}{Criterion for Closure}
Let $A$ be a subspace of $X$.
\begin{itemize}
\item $x\in \overline{A}$ iff $\forall U\in \mathcal{T} (x\in U \rightarrow U\cap A \neq \emptyset )$.
\item Suppose the topology has a basis $\mathcal{B}$. Then $x\in \overline{A}$ iff $\forall B\in \mathcal{B}(x\in B \rightarrow B\cap A\neq \emptyset )$.
\end{itemize}
\end{theorem}
\begin{proof}
\begin{itemize}
	\item $\forall A \subseteq V$ and $V$ is closed, then $X-V$ is open, but $X-V\cap A = \emptyset $, so $x\notin X-V$, so $x\in V$. Therefore, $x\in \overline{A}$.

	The other way, if $x\in \overline{A}$, then if $\exists U\in \mathcal{T}, x\in U, U\cap A = \emptyset $, then $X-U$ is closed and $A \subseteq X-U$, contradicts.
	\item For $U\in \mathcal{T}, U = \bigcup_{\alpha} B_{\alpha} $ for some $B_{\alpha}\in \mathcal{B}$ would do.
\end{itemize}
\end{proof}

\begin{definition}{Neighborhood}{Neighborhood}
We say $U$ is a neighborhood of $x$ when $U\in \mathcal{T}, x\in U$.

(Some would define neighborhood as: $U$ is a neighborhood of $x$ when $\exists V\in \mathcal{T}, x\in V \subseteq U$, which is looser)
\end{definition}
Using the criterion, we can restate the previous criterion as:
\begin{quote}
If $A \subseteq X$, then $x\in \overline{A}$ iff every neighborhood of  $x$ intersects $A$.
\end{quote}

\subsection{Limit Points}
Limit point are another way to describe the closure of a set.

Intuitively, a limit point is what we can reach by the limit process. In $\mathbb{R}$ the subset $(0,1]$ has the limit point set $[0,1]$. The $0$ we can reach by taking all $1 / n$ as $n \rightarrow  \infty $.

\begin{definition}{Limit Points}{Limit Points}
If $A \subseteq X$, $x\in X$, then $x$ is a limit point of $A$ iff every neighborhood of $x$ intersects $A$ in some point other than $x$ itself. Formally, if
\begin{equation*}
\forall U\in \mathcal{T}(x\in U \rightarrow \exists y\in A\cap U \land y\neq x)
\end{equation*}
then $x$ is a limit point of $A$.
\end{definition}

\begin{remark}
	We cay already see that every limit points are in the closure. But NOT vice versa, ``Single'' points that are ``out the group'' may be in $A$ but some open set containing it only intersets $A$ with it. For instance, in $\mathbb{R}$ the subset $[0,1]\cup \left\{ 2 \right\}$, the point $2$ is not a limit point but still in the closure.
\end{remark}

\begin{theorem}{Closure by Limit Points}{Closure by Limit Points}
Let $A \subseteq X$, let $A'$ be the set of all limit points of $A$. Then
\begin{equation}
\overline{A} = A\cup A'
\end{equation}
\end{theorem}
This is quite natural actually, just adding all ``single points'' would do.

\begin{proof}
\begin{itemize}
\item If $x\in \overline{A}$ : when $x\in A$, it is trivial, when $x\notin A$, $\forall U\in \mathcal{T}(x\in U \rightarrow U\cap A\notin \emptyset )$, then $\exists y\neq x,y\in U\cap A$, so $x\in A'$.
\item $A,A' \subseteq \overline{A}$ would do.
\end{itemize}
\end{proof}

\begin{corollary}{Closed Sets and Limit Points}{Closed Sets and Limit Points}
A subset $A \subseteq X$ is closed iff it contains all its limit points.
\end{corollary}

\subsection{Hausdorff Spaces}

In more abstract spaces, our intuition of $\mathbb{R}^n$ of openness and closeness would fail. Consider the topology $\left\{ \emptyset, \left\{ b \right\},\left\{ a,b \right\},\left\{ b,c \right\},\left\{ a,b,c \right\} \right\}$ on the 3-point space $\left\{ a,b,c \right\}$. Then the one-point set $\left\{ b \right\}$ is not closed.

\begin{definition}{Convergence}{Convergence}
If $(X,\mathcal{T})$ is a topological space. We say a sequence $\left\{ x_1,x_2, \ldots  \right\}$ converge to $x$ iff $\forall U\in \mathcal{T},x\in U, \exists N\in \mathbb{Z}_+(\forall n>N, x_n\in U)$.
\end{definition}

In the $\mathbb{R}^n$ case, this is exactly a restatement of the $\epsilon$-$N$ language without the need of a metric, but lead to the same result.

However, in arbitrary topological spaces, the same sequence may converge to multiple points, for example, the sequence $\left\{ x_n \right\}:x_n=b$ converge to $a,b,c$.

These examples are pretty strange. Then main problem is that we cannot explicit separate two points by open sets. Then open sets of the two points are all jagged together, making the limit process by open sets not sufficient to tell then apart.

\begin{definition}{Hausdorff Space}{Hausdorff Space}
A topological space $X$ is called a Hausdorff space if $\forall x_1,x_2\in X, x_1\neq x_2$, there exists neighborhood $U_1, U_2$ of $x_1, x_2$ such that $U_1\cap U_2 = \emptyset $. Formally,
\begin{equation}
\forall x_1,x_2\in X, x_1\neq x_2, \exists U_1,U_2\in \mathcal{T},x_1\in U_1\land x_2\in U_2\land U_1\cap U_2 = \emptyset 
\end{equation}
\end{definition}

\begin{figure}[ht]
    \centering
    \incfig{hausdorff-space}
    \caption{Hausdorff Space}
    \label{fig:hausdorff-space}
\end{figure}

\begin{theorem}{Finite Closure in Hausdorff Space}{Finite Closure in Hausdorff Space}
Every finite subset in a Hausdorff space is closed.
\end{theorem}
\begin{proof}
Let $x_0\in X$, $\forall x\in X, x\neq x_0, \exists x\in U_x\in \mathcal{T},x_0\notin U_x$, then $X-\left\{ x_0 \right\} = \bigcup_{x\neq x_0} U_x\in T$, so $\left\{ x_0 \right\}$ is closed. So all finite subset is closed.
\end{proof}

\begin{remark}
Note that we do not use the two disjoint neighborhoods simultaneously. This implies that \emph{the condition ``Every Single subset is Closed'' is weaker than the Hausdorff space condition}.

\vspace{3pt}
We call the condition ``Every Single subset is Closed'' the $T_1$-axiom. For instance, $\mathbb{R}$ with the finite complement topology is not a Hausdorff space but is a $T_1$-space. (very easy to prove)
\end{remark}

\begin{definition}{$T_1$-axiom}{T1-axiom}
If $(X,\mathcal{T})$ is a topological space, and $\forall x\in X, \left\{ x \right\}$ is closed, then $X$ satisfies the $T_1$-axiom.
\end{definition}

\begin{theorem}{Limit Points of $T_1$-spaces}{Limit Points of T1-spaces}
Let $X$ satisfies $T_1$-axiom, and $A \subseteq X$, then $x\in X$ is a limit point of $A$ iff every neighborhood of $x$ contains infinitely many points of $A$.
\end{theorem}
\begin{proof}
\begin{itemize}
\item $\rightarrow $ part: If some neighborhood $U$ of $X$ intersects $A$ with only finite points, then $U\cap (A-\left\{ x \right\}) = \left\{ x_1, \ldots ,x_n \right\}$ ($n$ could be $0$ ). Then $U\cap (X-\left\{ x_1, \ldots ,x_n \right\})$ is an open subset that intersects $A$ with at most $x$ not at all.

\item  $\leftarrow$ part: Obvious.
\end{itemize}
\end{proof}

However, $T_1$-axioms are too loose to have the following properties.

\begin{theorem}{Uniqueness of Limit Points of Hausdorff Spaces}{Uniqueness of Limit Points of Hausdorff Spaces}
If $X$ is a Hausdorff space, then a sequence in $X$ converge to at most onr point in $X$. Denoted $\lim_{n \to \infty } x_n = x$.
\end{theorem}
\begin{proof}
If $\left\{ x_n \right\}$ converge to both $x,y$ with $x\neq y$. Then let $x\in U_1\in \mathcal{T},y\in U_2\in \mathcal{T},U_1\cap U_2=\emptyset $. Then $X-U_1$ contains finite points in $x_n$, contradicts.
\end{proof}

\begin{theorem}{Properties of Hausdorff Spaces}{Properties of Hausdorff Spaces}
\begin{itemize}
\item Every topological space with order topology is a Hausdorff space. (simple order)
\item The product of two Hausdorff spaces is a Hausdorff space.
\item A subspace of a Hausdorff space is a Hausdorff space.
\end{itemize}
\end{theorem}

\begin{proof}
\begin{itemize}
\item For $x<y$, if $\exists z: x<z<y$, take $U_1 = (-\infty ,z), U_2 = (z,+\infty )$ would do. Otherwise, $\forall z\in X, z \leq x\lor z\geq y$, taking $U_1 = (-\infty ,y), U_2=(x,+\infty )$ would do.
\item For $(x_1,y_1),(x_2,y_2)\in X \times Y$, we have $x_1\in U_1,x_2\in U_2$ and $U_1\cap U_2=\emptyset $, also $y_1\in V_1,y_2\in V_2,V_1\cap V_2=\emptyset $, then we have $(x_1,y_1)\in U_1 \times V_1,(x_2,y_2)\in U_2 \times V_2$.
\item If $S \subseteq X$, then for $x\neq y$ in $S$, we take $U_1\cap S$ and $U_2\cap S$ would do.
\end{itemize}
\end{proof}


\section{Continuous Functions}
In common sense, a continuous function preserve locality between range and domain, so it is quite natural to define continuity by the following way.

\begin{definition}{Continuity of a Function}{Continuity of a Function}
Let $X,Y$ be topological spaces. A function $f: X \rightarrow Y$ is continuous if for each open $V \subseteq Y$, we have $f^{-1}(V)\subseteq X$ is open.

(Note that the continuity of a function depend on the topology defined on $X$ and $Y$.)
\end{definition}

\begin{remark}
If $Y$ is given a basis $\mathcal{B}$, then if $f^{-1}(B)$ is open for $\forall B\in \mathcal{B}$, then $f$ is continuous. 

Also, given a subbasis $\mathcal{S}$, then $f^{-1}(S)$ is open for $\forall S\in \mathcal{S}$ implies $f$ being continuous.

The above to statements is given by:
\begin{equation*}
f^{-1}(\bigcup_{\alpha} B_{\alpha}) = \bigcup_{\alpha} f^{-1}(B_{\alpha})
\end{equation*}
\begin{equation*}
f^{-1}(\bigcap_{i=1}^{n} S_i) = \bigcap_{i=1}^{n} f^{-1}(S_i)
\end{equation*}
\end{remark}

\begin{example}{Continuity of $\mathbb{R}\rightarrow \mathbb{R}$}{Continuity of mathbbRrightarrow mathbbR}
In analysis the continuity of a function $f: \mathbb{R}\rightarrow \mathbb{R}$ is given by $\epsilon$-$\delta$ language: if $\forall \epsilon>0 \exists \delta>0 \forall x (\left|x-x_0\right|<\delta \rightarrow \left|f(x)-f(x_0)\right|<\epsilon)$, then $f$ is continuous at $x_0$.

This definition is compatible with the topological definition above in the standard topology.
\begin{proof}
If $f$ follows the $\epsilon$-$\delta$ continuity, let $V = (a,b)$ be an element of the basis of $\mathbb{R}$, and $U = f^{-1}(V)$, then $\forall x\in U, f(x)\in (a,b)$, then let $\epsilon>0$ be such that $(f(x)-\epsilon,f(x)+\epsilon) \subseteq (a,b)$, then $\exists \delta>0$ such that $U_x = (x-\delta,x+\delta),f(U_x) \subseteq V$, then $U_x \subseteq f^{-1}(V)$. As $U = \bigcup_{x\in U} U_x$, we have $U$ being open.

For the other way, if $f$ follows the topological continuity, given  $x_0\in \mathbb{R}$, the interval $V = (f(x_0)-\epsilon,f(x_0)+\epsilon)$ is open, then $f^{-1}(V)$ is open. As $x_0\in f^{-1}(V)$, then there is basis element $x_0\in (a,b) \subseteq f^{-1}(V)$, letting $\delta$ be smaller would do.
\end{proof}
\end{example}

\begin{example}{Continuity and Topology}{Continuity and Topology}
Let $\mathbb{R}$ denote the standard topology, and $\mathbb{R}_l$ the lower limit topology.
\begin{itemize}
\item Let $f: \mathbb{R}\rightarrow \mathbb{R}_l$ be the identity function, then $f$ is not continuous, for $f^{-1}([a,b)) = [a,b)$ which is not open in $\mathbb{R}$.
\item However, $f^{-1}: \mathbb{R}_l \rightarrow \mathbb{R}$ is continuous.
\end{itemize}
\end{example}

\begin{theorem}{Equivalent Definitions for Continuity}{Equivalent Definitions for Continuity}
Let $X,Y$ to topological spaces, and $f: X \rightarrow Y$. Then the following are equivalent.
\begin{enumerate}
\item $f$ is continuous. $\forall A \subseteq Y, A \in \mathcal{T}_Y, f^{-1}(A)\in \mathcal{T}_X$.
\item $\forall A \subseteq X, f(\overline{A}) \subseteq \overline{f(A)}$.
\item $\forall B \subseteq Y$ which is closed, $f^{-1}(B)$ is closed.
\item $\forall x\in X$ and $\forall $ neighborhood $V$ of $f(x)$, $\exists $ a neighborhood $U$ of $x$ such that $f(U) \subseteq V$.
\end{enumerate}
\end{theorem}
\begin{proof}
\begin{itemize}
\item $1 \rightarrow 2$ : We shall show that $x\in \overline{A} \rightarrow f(x)\in \overline{f(A)}$. Let $V$ be a neighborhood of $f(x)$, then $f^{-1}(V)$ is a neighborhood of $x$, so $\exists y\in f^{-1}(V)\cap A$. So $f(y)\in f(A)\cap V$, so $f(x) \in \overline{f(A)}$.
\item $2 \rightarrow 3$ : Let $A = f^{-1}(B)$, then $f(A) \subseteq B$, so $f(\overline{A}) \subseteq \overline{f(A)} \subseteq B$, so $\overline{A} \subseteq f^{-1}(B) = A$, and $A \subseteq \overline{A}$ implies $A = \overline{A}$, meaning $A$ is closed.
\item $3 \rightarrow 1$ : $f^{-1}(B)$ and $f^{-1}(Y-B)$ part $X$ would do.
\item $1 \rightarrow 4$ : Letting $U = f^{-1}(V)$ would do.
\item $4 \rightarrow 1$: Similar of example \ref{exp:Continuity of mathbbRrightarrow mathbbR}.
\end{itemize}
\end{proof}

All the above illustrate our intuition of continuity: preserving locality.

\subsection{Homeomorphisms}
Homeomorphisms in topology is just isomorphisms between topological spaces. We need to preserve openness. As continuity show preservation of openness in one direction, we can simplify sayings by it.

\begin{definition}{Homeomorphism}{Homeomorphism}
Let $X, Y$ be topological spaces, $f:X \rightarrow Y$ be a bijection. If both $f$ and $f^{-1}$ are continuous, then $f$ is a homeomorphism.
\end{definition}

Homeomorphisms preserve all topological properties of the two spaces. If there is a homeomorphism between $X$ and $Y$, then we can treat then the same in topology.

Sometimes $X$ is homeomorphic to a subspace of $Y$, which is called an imbedding.

\begin{definition}{Topological Imbedding}{Topological Imbedding}
Suppose $f:X \rightarrow Y$ is an injective continuous map. Let $Z = f(X)$, then $f': X \rightarrow Z$ is a bijection. If $f'$ is a homeomorphism of $X$ and $Z$ ($Z$ with the subspace topology), then we say $f$ is a topological imbedding of $X$ in $Y$.
\end{definition}

\begin{example}{Continuity and Homeomorphisms}{Continuity and Homeomorphisms}
Let $S^1 \subseteq \mathbb{R}^2$ denote the unit circle. Let
\begin{equation*}
F: [0,1) \rightarrow S^1, F(t) = (\cos 2 \pi t,\sin 2\pi t)
\end{equation*}
Then $f$ is indeed continuous and bijective, but not an isomorphism, for $f([0,\frac{1}{2}))$ is not open in $S^1$.

The function $g: [0,1) \rightarrow \mathbb{R}^2,g(x)=f(x)$ is continuous but not an imbedding.
\end{example}

\subsection{Constructing Continuous Functions}

Sometimes it is rather hard to directly verify the continuity of a function. We use construction rules in analysis: compound functions and operations etc. Some can be generalized here.

\begin{theorem}{Construction Rules for Continuous Functions}{Construction Rules for Continuous Functions}
Let $X,Y,Z$ be topological spaces. All topologies in subspace are subspace topologies.
\begin{enumerate}
	\item (Constant Function) If $f: X \rightarrow Y, x \mapsto y_0$, a constant point in $Y$, then $f$ is continuous.
	\item (Inclusion) If $A \subseteq X$, the inclusion function $j:A \rightarrow X, x \mapsto x$ is continuous.
	\item (Composites) If $f:X \rightarrow Y$ and $g:Y \rightarrow Z$ are continuous, then $g \circ f: X \rightarrow Z$ is continuous.
	\item (Restricting the Domain) If $f: X \rightarrow Y$ is continuous, then for $\forall A \subseteq X$, $f|_A: A \rightarrow Y$ is continuous.
	\item (Expanding/Restricting the Range) If $f:X \rightarrow Y$ be continuous, and $f(X) \subseteq Z \subseteq Y$ or $f(X) \subseteq Y \subseteq Z$, then $f: X \rightarrow Z$ is continuous.
	\item (Local formulation of Continuity): $f:X \rightarrow Y$ is continuous if $X = \bigcup_{\alpha} U_{\alpha}$ such that $U_{\alpha}$ is open and $f|_{U_{\alpha}}$ is continuous.
\end{enumerate}
\end{theorem}

Most of the above is obvious, just using the definition would do.

\begin{theorem}{The Pasting Lemma}{The Pasting Lemma}
Let $X = A \cup B$, where $A,B$ are both closed in $X$. Let $f:A \rightarrow Y$ and $g:B \rightarrow Y$ be continuous, and $\forall x\in A\cap B, f(x)=g(x)$. And let $h: X \rightarrow Y$, where $h(x) = f(x)$ if $x\in A$ and $h(x)=g(x)$ if $x\in  B$. Then $h$ is continuous.

This theorem also holds for $A,B$ being open, which is just a special case for the local formulation of continuity.
\end{theorem}

\begin{proof}
Let $C$ be a closed subset of $Y$, we have
\begin{equation*}
h^{-1}(C) = f^{-1}(C) \cup g^{-1}(C)
\end{equation*}
Since $f,g$ are continuous, so $f^{-1}(C)$ is closed in $A$, and $A$ is closed in $X$, so $f^{-1}(C)$ is closed in $X$, so is $g^{-1}(C)$.
\end{proof}

\begin{remark}
Both closeness is quite necessary for the theorem to hold. For instance,
\begin{equation*}
h(x) = 
\begin{cases}
	x-2, &\text{ if } x \leq 0\\
	x+2, & \text{ if } x>0
\end{cases}
\end{equation*}
is not continuous in $\mathbb{R}$, but is continuous in $\mathbb{R}_{\leq 0}$ and $\mathbb{R}_{>0}$ respectively.
\end{remark}

\begin{theorem}{Maps into Products}{Maps into Products}
Let $f: A \rightarrow X \times Y$ be given by
\begin{equation*}
f(a) = (f_1(a),f_2(a))
\end{equation*}
Then $f$ is continuous iff $f_1,f_2$ are both continuous.

$f_1,f_2$ are called coordinate functions of $f$.
\end{theorem}

\begin{proof}
Taking the subbasis $\mathcal{S} = \left\{ U_1 \times Y :U_1\in \mathcal{T}_X \right\}\cup \left\{ X \times U_2: U_2\in \mathcal{T}_Y \right\}$. If $f_1,f_2$ are continuous, then $f^{-1}(U_1 \times Y) = f_1^{-1}(U_1) \times A$ which is open, so does $f^{-1}(Y \times U_2)$.

If $f$ is continuous, we also have the projections $\pi_1: X \times Y \rightarrow X$ and $\pi_2: X \times Y \rightarrow  Y$ are continuous. So the composite $f_1 = \pi_1\circ f$, $f_2 = \pi_2\circ f$ are continuous.
\end{proof}

This result is just the continuity of parametrized curve (or vector field) in calculus, with $a\in A$ as the parameter.

\section{The Product Topology}
Previous discussion shows how we can impose a topology on finite Cartesian products $X_1 \times \cdots \times X_n$, which is just a generalization for the two-space case $X \times Y$. Here we take a closer look on the infinite number of Cartesian products.

Consider the products
\begin{equation*}
X_1 \times \cdots \times X_n \text{ and } X_1 \times X_2 \times \cdots 
\end{equation*}

We give a more formal definition of Cartesian products for infinite products:
\begin{definition}{Tuples}{Tuples}
Let $J$ be an index set. Given a set $X$, define a $J$-tuple of elements of $X$ to be a function $x:J \rightarrow X$, we also denote $x(\alpha)$ by $x_{\alpha}$, called the $\alpha$-th coordinate of $x$. We also denote the function $x$ by $(x_{\alpha}) _{\alpha\in J}$.

 $X^J$ denotes all $J$-tuples of $X$.
\end{definition}

\begin{definition}{Cartesian Products}{Cartesian Products}
Let $\left\{ A_{\alpha} \right\}_{\alpha\in J}$ be an indexed family of sets. Let $X = \bigcup_{\alpha\in J} A_{\alpha}$. The Cartesian product of $\left\{ A_{\alpha} \right\}_{\alpha\in J}$ is denoted by
\begin{equation*}
\prod_{\alpha\in J} A_{\alpha} = \left\{ (x_{\alpha})_{\alpha\in J} : \forall \alpha\in J, x_{\alpha}\in A_{\alpha} \right\}
\end{equation*}

Which is all the $J$-tuples that $x_{\alpha}\in A_{\alpha}$, being a subset of $X^J$.
\end{definition}

To generate a basis, we have two ways, generalized by the two previous ways to describe the topology on $X \times Y$.
\begin{itemize}
\item \emph{The Box Topology}: Taking all $\prod_{\alpha} U_{\alpha}$ where $U_{\alpha}\in \mathcal{T}_{\alpha}$ as a basis.
\item \emph{The Product Topology}: Taking all $\pi_{\alpha}^{-1}(U_{\alpha})$ as a subbasis.
\end{itemize}

Previous discussion has shown that these two ways result in the same topology in finite case.

\begin{definition}{The Box Topology}{The Box Topology}
Let $\left\{ X_{\alpha} \right\}_{\alpha\in J}$ be an indexed set family of topological spaces. Let
\begin{equation*}
X = \prod_{\alpha\in J}X_{\alpha}
\end{equation*}
be the product space. Define a topology with the basis $\mathcal{B}$ consisting of all the following form
\begin{equation*}
\prod_{\alpha\in J}U_{\alpha}, \text{ where }\forall \alpha\in J,U_{\alpha} \text{ is open in }X_{\alpha}
\end{equation*}
the topology is called the box topology of $X$.
\end{definition}

To see that the box topology is indeed a topology, we have $X = \prod X_{\alpha}$ is itself a basis element, also
\begin{equation*}
\left(\prod_{\alpha\in J}U_{\alpha}\right)\cap \left(\prod_{\alpha\in J}V_{\alpha}\right) = \prod _{\alpha\in J} \left(U_{\alpha}\cap V_{\alpha}\right)
\end{equation*}
The intersection itself is another basis element.

To generalize the subbasis formulation, define
\begin{equation}
\pi_{\beta}: \prod_{\alpha\in J}X_{\alpha} \rightarrow X_{\beta}
\end{equation}
\begin{equation*}
\pi_{\beta}((x_{\alpha})_{\alpha\in J}) = x_{\beta}
\end{equation*}
is called the projection mapping associated with the index $\beta$.

\begin{definition}{The Product Topology (Subbasis)}{The Product Topology (Subbasis)}
Let $\mathcal{S}_{\beta}$ denote the following collection:
\begin{equation*}
\mathcal{S}_{\beta} = \left\{ \pi_{\beta}^{-1}(U_{\beta}) : U_{\beta} \text{ is open in }X_{\beta} \right\}
\end{equation*}
And let
\begin{equation*}
\mathcal{S} = \bigcup_{\beta\in J} \mathcal{S}_{\beta}
\end{equation*}
The topology generated by $\mathcal{S}$ as a subbasis is called the product topology. In this topology the product $\prod_{\beta\in J}X_{\beta}$ is called the product space.
\end{definition}

To see the basis generated by $\mathcal{S}$, we take finite intersections.

First, intersecting elements inside $\mathcal{S}_{\beta}$ would not give us anything new, for
\begin{equation*}
\pi_{\beta}^{-1}(U_{\beta})\cap \pi_{\beta}^{-1}(V_{\beta}) = \pi_{\beta}^{-1}(U_{\beta}\cap V_{\beta})
\end{equation*}

Intersection of elements among different $\mathcal{S}_{\beta_i}$ is what we want. Let $\beta_1, \ldots ,\beta_n$ be distinct element in $J$ and $U_{\beta_i}$ be open in $X_{\beta_i}$, then
\begin{equation*}
B = \pi_{\beta_1}^{-1}(U_{\beta_1} \cap \cdots \cap \pi_{\beta_n}^{-1}(U_{\beta_n})
\end{equation*}
is a typical element of the basis $\mathcal{B}$.

\begin{theorem}{Comparison of the box and product topologies}{Comparison of the box and product topologies}
\begin{itemize}
\item The box topology on $\prod X_{\alpha}$ has a basis of the form $\prod U_{\alpha}$ where $U_{\alpha}$ is open in $X_{\alpha}$.
\item The product topology on $\prod X_{\alpha}$ has a basis of the form $\prod U_{\alpha}$, where $U_{\alpha} = X_{\alpha}$ except for finite many values of $\alpha$, and all $U_{\alpha} \subseteq X_{\alpha}$ are open.
\end{itemize}

For finite Cartesian products the two topologies are the same. In general case the box topology is finer than the product topology.
\end{theorem}

The product topology plays a more significant role in topology than the box topology, for it is more natural to consider the product of topological spaces. The box topology is too fine to be useful in most cases.

\begin{quote}
NOTE: Whenever we consider the product $\prod X_{\alpha}$, we will always assume the product topology unless otherwise specified.
\end{quote}

\begin{theorem}{Product Topology by Basis}{Product Topology by Basis}
Suppose the topology on each $X_{\alpha}$ is given by a basis $\mathcal{B}_{\alpha}$. Then the box topology on $\prod X_{\alpha}$ has a basis consisting of all sets of the form
\begin{equation*}
	\prod_{\alpha\in J} B_{\alpha}, \text{ where } B_{\alpha}\in \mathcal{B}_{\alpha} \text{ for all } \alpha\in J
\end{equation*}

Similarly, the product topology on $\prod X_{\alpha}$ has a basis consisting of all sets of the form
\begin{equation*}
	\prod_{\alpha\in J} B_{\alpha}, \text{ where } B_{\alpha}\in \mathcal{B}_{\alpha} \text{ for finitely many } \alpha\in J, \text{ and } B_{\alpha} = X_{\alpha} \text{ for all other } \alpha\in J
\end{equation*}
\end{theorem}
\begin{proof}
By closure under arbitrary unions, it is obvious.
\end{proof}

\begin{theorem}{Subspace Topology of Products}{Subspace Topology of Products}
$\forall \alpha\in J$, let $A_{\alpha}$ be a subspace of $X_{\alpha}$. Then the topology $\prod_{\alpha} A_{\alpha}$ is the subspace topology on $\prod_{\alpha} X_{\alpha}$, either by the box topology or the product topology.
\end{theorem}
\begin{proof}
It is just a generalization of theorem \ref{thm:Exchange of Subspace and Product Topology}. Using the fact that
\begin{equation*}
\prod_{\alpha\in J} (B_{\alpha} \cap A_{\alpha}) = \left(\prod_{\alpha\in J}B_{\alpha}\right) \cap \left(\prod_{\alpha\in J}A_{\alpha}\right)
\end{equation*}
\end{proof}

\begin{theorem}{Products of Hausdorff Spaces}{Products of Hausdorff Spaces}
Let $\left\{ X_{\alpha} \right\}_{\alpha\in J}$ be a family of Hausdorff spaces. Then $\prod_{\alpha\in J} X_{\alpha}$ is a Hausdorff space, in both the box topology and the product topology.
\end{theorem}

\begin{theorem}{Products of Closure}{Products of Closure}
Let $\left\{ X_{\alpha} \right\}_{\alpha\in J}$ be a family of topological spaces, and $A_{\alpha} \subseteq X_{\alpha}$ for each $\alpha\in J$. Then
\begin{equation*}
	\overline{\prod_{\alpha\in J} A_{\alpha}} = \prod_{\alpha\in J} \overline{A_{\alpha}}
\end{equation*}
in both the box topology and the product topology.
\end{theorem}
\begin{proof}
	Let $x = (x_{\alpha}) \in \prod \overline{A_{\alpha}}$, we show that $x \in \overline{\prod A_{\alpha}}$. Let $U = \prod U_{\alpha}$ be a basis element of $\prod X_{\alpha}$ that $x\in U$. Then as $x_{\alpha}\in \overline{A_{\alpha}}$, then $\exists y_{\alpha}\in U_{\alpha}\cap A_{\alpha}$, so $y\in \prod U_{\alpha}\cap A_{\alpha} = \prod U_{\alpha}\cap \prod A_{\alpha}$. So $U\cap \prod A_{\alpha} \neq \emptyset$, so $x\in \overline{\prod A_{\alpha}}$.

	Conversely, suppose $x\in \overline{\prod A_{\alpha}}$. Let $U = \prod U_{\alpha}$ be a basis element of $\prod X_{\alpha}$ that $x\in U$. Then $\exists y\in U\cap \prod A_{\alpha}$. So $y_{\alpha}\in U_{\alpha}\cap A_{\alpha}$, so $x_{\alpha}\in \overline{A_{\alpha}}$ for all $\alpha\in J$. So $x\in \prod \overline{A_{\alpha}}$.
\end{proof}

Up till now, all the results holds the same for the box topology and the product topology. A first difference arises in the study of continuity.

\begin{theorem}{Continuity of Products}{Continuity of Products}
Let $f:A \rightarrow \prod _{\alpha\in J} X_{\alpha}$ be given by
\begin{equation*}
	f(a) = (f_{\alpha}(a))_{\alpha\in J}
\end{equation*}
where $f_{\alpha}: A \rightarrow X_{\alpha}$ for each $\alpha\in J$. Then $f$ is continuous in the product topology iff $f_{\alpha}$ is continuous for each $\alpha\in J$.
\end{theorem}
\begin{proof}
\begin{itemize}
\item $\Rightarrow$ part: If $f$ is continuous, then for each $\alpha\in J$, we have $\pi_{\alpha}\circ f = f_{\alpha}$ is continuous
\item $\Leftarrow$ part: If $f_{\alpha}$ is continuous for each $\alpha\in J$, then for each open set $U_{\alpha} \subseteq X_{\alpha}$, we have $\pi_{\alpha}^{-1}(U_{\alpha})$ is open in the product topology, and $f^{-1}(\pi_{\alpha}^{-1}(U_{\alpha})) = f_{\alpha}^{-1}(U_{\alpha})$ is open in $A$, so $f$ is continuous (all the preimages of a subbasis is open).
\end{itemize}
\end{proof}

Note that the $\Leftarrow$ part uses the finite intersection of the subbasis elements $\pi_{\alpha}^{-1}(U_{\alpha})$ to generate the basis elements of the product topology. While the box topology allows infinite intersection, which would bring problems

\begin{example}{Conterexamples of Continuity of Box Topology}{Conterexamples of Continuity of Box Topology}
Consider $\mathbb{R}^{\mathbb{Z}_+}$. Let $f: \mathbb{R}\rightarrow \mathbb{R}^{\mathbb{Z}_+}$ to be
\begin{equation*}
f(t) = (t,t, \ldots )
\end{equation*}
Then $f$ is continuous in the product topology. In box topology, however, let
\begin{equation*}
	 B = (-1,1) \times (-\frac{1}{2}, \frac{1}{2}) \times (-\frac{1}{3}, \frac{1}{3}) \times \cdots
\end{equation*}
We have $f^{-1}(B) = \left\{ 0 \right\}$ which is not open in $\mathbb{R}$.
\end{example}


\section{The Metric Topology}

Most of our intuition of open/closed sets and neighborhoods and locality and continuous functions are based on the metric space, mostly $\mathbb{R}^n$. Indeed, if given a metric, we can naturally define a topology based on it.

\begin{definition}{Metric}{Metric}
A metric on a set $X$ is a function $d: X \times X \rightarrow \mathbb{R}$:
\begin{itemize}
\item $d(x,y) \geq 0$ for all $x,y\in X$, and $d(x,y) = 0$ iff $x=y$.
\item $d(x,y) = d(y,x)$ for all $x,y\in X$.
\item $d(x,y) + d(y,z) \geq d(x,z)$ for all $x,y,z\in X$.
\end{itemize}
\end{definition}

Given a metric $d$ on $X$, the number $d(x,y)$ is called the distance between $x$ and $y$. Also, if given $\epsilon>0$, the set
\begin{equation*}
	B_d(x,\epsilon) = \left\{ y\in X : d(x,y) < \epsilon \right\}
\end{equation*}
is called the $\epsilon$-ball centered at $x$.

\begin{definition}{Metric Topology}{Metric Topology}
If $d$ is a metric on $X$, then the collection of all $\epsilon$-balls forms a basis
\begin{equation*}
\mathcal{B} = \left\{ B_d(x,\epsilon): x\in X, \epsilon>0 \right\}
\end{equation*}
The topology generated by $\mathcal{B}$ is called the metric topology on $X$ induced by $d$.
\end{definition}

The following proposition is widely used in analysis.
\begin{proposition}{Openness of a Ball}{Openness of a Ball}
If $y\in B(x,\epsilon)$, then $\exists \delta>0, y\in B(y,\delta) \subseteq B(x,\epsilon)$.

\begin{proof}
For we have $d(x,y)<\epsilon$, take $\delta = \epsilon - d(x,y) > 0$, then for $z\in B(y,\delta)$, we have
\begin{equation*}
d(x,z) \leq d(x,y) + d(y,z) < d(x,y) + \delta = d(x,y) + \epsilon - d(x,y) = \epsilon
\end{equation*}
So $z\in B(x,\epsilon)$, so $B(y,\delta) \subseteq B(x,\epsilon)$.
\end{proof}
\end{proposition}

Next we prove that $\mathcal{B}$ is indeed a basis for a topology on $X$.
\begin{proof}
\begin{itemize}
\item The first condition is trivial for $\forall x\in X, x\in B(x, \epsilon)$.
\item Let $B_1,B_2$ be two basis element, if $y\in B_1\cap B_2$, we can choose $\delta_1,\delta_2>0$ such that $y\in B(y,\delta_1) \subseteq B_1$ and $y\in B(y,\delta_2) \subseteq B_2$. Let $\delta = \min\left\{ \delta_1, \delta_2 \right\}$, then $B(y,\delta) \subseteq B_1\cap B_2$, so $B(y,\delta)$ is a basis element containing $y$.
\end{itemize}
\end{proof}

The following result returns us to the original definition of openness in analysis:
\begin{proposition}{Openness of Sets in Metric Topology}{Openness of Sets in Metric Topology}
A set $U$ is open in the metric space $(X,d)$ iff
\begin{equation}
\forall x\in U,\exists \delta>0, B_d(x,\delta) \subseteq U
\end{equation}
\end{proposition}
\begin{proof}
Clearly the condition implies $U$ being open. Conversely, if $U$ is open, $\forall x\in U$, there is $x\in B(y,\epsilon) \subseteq U$, then we can take $\delta$ that $B_d(x,\delta) \subseteq B(y,\epsilon) \subseteq U$.
\end{proof}

\begin{figure}[ht]
    \centering
    \incfig{openness-in-metric-spaces}
    \caption{Openness in Metric Spaces}
    \label{fig:openness-in-metric-spaces}
\end{figure}

\begin{proposition}{Continuity of Metric}{Continuity of Metric}
	Let $(X,d)$ be a metric space. Then the metric $d: X \times X \rightarrow \mathbb{R}$ is continuous, where $X\times X$ has the product topology and $\mathbb{R}$ has the standard topology.

	Also, $x \mapsto d(x,y)$ is continuous for fixed $y\in X$.
\end{proposition}
\begin{proof}
Let $(x_0,y_0) \in X \times X$, and $\epsilon>0$. Let $\delta = \frac{\epsilon}{2}$. Then for $(x,y) \in B(x_0,\delta) \times B(y_0,\delta)$, we have
\begin{equation*}
|d(x,y) - d(x_0,y_0)| \leq |d(x,y) - d(x_0,y)| + |d(x_0,y) - d(x_0,y_0)| \leq d(x,x_0) + d(y,y_0) < 2\delta = \epsilon
\end{equation*}
So $d$ is continuous.

The second part follows from the Restricting of Domain rule of continuous functions.
\end{proof}

\begin{example}{Metric Topology}{Metric Topology}
\begin{itemize}
\item The standard metric on $\mathbb{R}$ is $d(x,y) = |x-y|$, the metric topology is the standard topology on $\mathbb{R}$.
\item On $\mathbb{R}^n$, we can define the $p$-metric as
	\begin{equation*}
		d(x-y) = \left( \sum_{i=1}^{n} |x_i - y_i|^p \right)^{\frac{1}{p}}, \text{ for } p\geq 1
	\end{equation*}
	When $p=2$, it is the Euclidean metric, the balls are spheres.

	When $p=\infty$, it is the maximum metric (or square metric), defined as
	\begin{equation*}
		d(x,y) = \max_{1\leq i \leq n} |x_i - y_i|
	\end{equation*}
	the balls are cubes.
\end{itemize}
\end{example}

\begin{definition}{Metrizable}{Metrizable}
A topological space $X$ is called metrizable if there exists a metric $d$ on $X$ such that the topology induced by $d$ is the same as the topology of $X$.
\end{definition}

\begin{remark}
	Metrizable spaces are valuable but rare in general. A metric gives powerful tools to study the topology, such as compactness, connectedness, and convergence, just like what we did in analysis.

	It is worth noting that not all topological spaces are metrizable. And it is of fundamental importance to find conditions under which a topological space is metrizable: they are expressed in the Urysohn Metrization Theorem, which will be discussed later.

	We shall also note that a metric is not a topological property, and properties of a metric space may not be preserved under homeomorphisms.
\end{remark}

\begin{definition}{Boundedness}{Boundedness}
Let $(X,d)$ be a metric space. A subset $A \subseteq X$ is bounded if there exists $M>0$ such that $\forall x,y\in A, d(x,y) < M$.

If $A$ is bounded and nonempty, we define the diameter of $A$ to be
\begin{equation*}
	\diam A = \sup \left\{ d(x,y): x,y\in A \right\}
\end{equation*}

We define another metric $\tilde{d}: X \times X \rightarrow \mathbb{R}$ by
\begin{equation*}
\tilde{d}(x,y) = \min \left\{ d(x,y),1 \right\}
\end{equation*}
then $\tilde{d}$ is called the standard bounded metric corresponding to $d$. We can verify that $\tilde{d}$ is indeed a metric, and the topology induced by $\tilde{d}$ is the same as the topology induced by $d$.
\end{definition}

The open balls induced by $\tilde{d}$ are comprised of two parts:
\begin{itemize}
	\item All the balls $B_{\tilde{d}}(x,\epsilon)$ for $\epsilon < 1$ are the same as those induced by $d$.
	\item The whole space
\end{itemize}

\begin{remark}
	Different metrics may induce the same topology, such as the Euclidean metric and the square metric on $\mathbb{R}^n$.
\end{remark}

\begin{lemma}{Comparing Topologies via Metrics}{Comparing Topologies via Metrics}
Let $d_1,d_2$ be two metrics on a set $X$. Let $\mathcal{T}_1,\mathcal{T}_2$ be the topologies induced by $d_1,d_2$ respectively. Then $\mathcal{T}_2$ is finer than $\mathcal{T}_1$, ($\mathcal{T}_1 \subseteq \mathcal{T}_2$ ), iff
\begin{equation*}
	\forall x\in X, \forall \epsilon>0, \exists \delta>0, B_{d_1}(x,\delta) \subseteq B_{d_2}(x,\epsilon)
\end{equation*}
\end{lemma}
This is just restatement of proving the basis for $\mathcal{T}_2$ is all open in $\mathcal{T}_1$.

Using the lemma, it is quite easy to see that all the $p$-metric on $\mathbb{R}^n$ gives the same topology.

Now we consider metrics on the infinite Cartesian product $\mathbb{R}^{\mathbb{Z}_+}$, for which we can just generalize the $p$-metric.
\begin{itemize}
\item For finite $p$, we can define
\begin{equation*}
d(x,y) = \left( \sum_{i=1}^{\infty} |x_i - y_i|^p \right)^{\frac{1}{p}}
\end{equation*}
This does not always make sense, for the infinite sum may not converge. However, if $X \subseteq \mathbb{R}^{\mathbb{Z}_+}$ consists of all sequences $x$ such that $\sum_{i=1}^{\infty } x_i^p$ converges, then the above metric is well-defined and induces a topology on $X$.

\item For $p=\infty$, we can define the metric
\begin{equation*}
	d(x,y) = \sup_{i\in \mathbb{Z}_+} |x_i - y_i|
\end{equation*}
This also does not always make sense, but if we replace $d$ with $\tilde{d}$ of $\mathbb{R}$, this is well-defined then, called the uniform metric on $\mathbb{R}^{\mathbb{Z}_+}$.
\end{itemize}

\begin{definition}{Uniform Metric}{Uniform Metric}
Given an index set $J$, and $x,y\in \mathbb{R}^J$, define a metric $\tilde{\rho}$ on $\mathbb{R}^J$ by
\begin{equation}
	\tilde{\rho}(x,y) = \sup \left\{ \tilde{d}(x_{\alpha},y_{\alpha}): \alpha\in J \right\}
\end{equation}
where $\tilde{d}$ is the standard bounded metric on $\mathbb{R}$.

Then $\tilde{\rho}$ is called the uniform metric on $\mathbb{R}^J$. And the topology it induces is called the uniform topology on $\mathbb{R}^J$.
\end{definition}

The basis of the uniform topology is given by all the rectangles whose edges has length $<1$ or the whole $\mathbb{R}$. So it is quite easy to compare it to the product topology and box topology on $\mathbb{R}^J$.

\begin{theorem}{Comparison of Uniform Topology}{Comparison of Uniform Topology}
	The uniform topology on $\mathbb{R}^J$ is finer than the product topology and coarser than the box topology.

	The three topologies are the same if and only if $J$ is finite.
\end{theorem}
\begin{proof}
Using the fact that $\mathcal{B}_{\text{product}} \subseteq \mathcal{B}_{\text{uniform}} \subseteq \mathcal{B}_{\text{box}}$ would do.

In finite case, it is easy to see that all three topologies are the same. For the infinite case, 
\begin{itemize}
\item The cube $(0, \frac{1}{2})^{J}$ is not open in the product topology, but is open in the uniform topology.
\item Consider the set
\begin{equation*}
	B = (-1,1) \times (-\frac{1}{2}, \frac{1}{2}) \times (-\frac{1}{3}, \frac{1}{3}) \times \cdots 
\end{equation*}
For a countable part of $J$, and the whole $\mathbb{R}$ for the rest of $J$. 

Then $B$ is open in the box topology, but not in the uniform topology, for $0\in B$, and $\forall \epsilon>0, B(0,\epsilon)$ has element outside $(-\frac{1}{n},\frac{1}{n})$ for sufficiently large $n$.
\end{itemize}
\end{proof}

When $J$ is infinite, we have not determined whether $\mathbb{R}^J$ is metrizable in box or product topology. We shall further see that the only case that it is metrically is when $J$ is countable, and in the product topology.

\begin{theorem}{Metrization of $\mathbb{R}^{\mathbb{Z}_+}$}{Metrization of mathbbRmathbbZ+}
Let $\tilde{d} = \min \left\{ |a-b|,1 \right\}$ be the standard bounded metric on $\mathbb{R}$. $\forall x,y\in \mathbb{R}^{\mathbb{Z}_+}$, we define
\begin{equation*}
D(x,y) = \sup_{i\in \mathbb{Z}_+} \left\{ \frac{\tilde{d}(x_i,y_i)}{i} \right\}
\end{equation*}
Then $D$ is a metric that induced the product topology on $\mathbb{R}^{\mathbb{Z}_+}$.
\end{theorem}
We notice that the $D$ given above gives a set of balls with first finite elements are some bounded intervals, and the rest are all $\mathbb{R}$. For sufficiently large $i$, $\forall x_i,y_i\in \mathbb{R}$, we have $\tilde{d}(x_i,y_i) / i < \epsilon$. This is just what we want for the product topology.
\begin{proof}
The properties of a metric are easy to verify: For the triangle inequality, because
\begin{equation*}
	\forall i\in \mathbb{Z}_+, \frac{\tilde{d}(x_i,z_i)}{i} \leq \frac{\tilde{d}(x_i,y_i)}{i} + \frac{\tilde{d}(y_i,z_i)}{i} \leq D(x,y) + D(y,z)
\end{equation*}
So $D(x,y) \leq D(x,z) + D(z,y)$.

Next we prove that $D$ gives the product topology. Let $U$ be open in the metric topology, and $x\in U$, then $\exists \epsilon>0$ such that $B_D(x,\epsilon) \subseteq U$. So for $N > 1 / \epsilon$, the coordinates of $\mathcal{B}_D(x,\epsilon)$ range over $\mathbb{R}$. So we take
\begin{equation*}
	V = \prod_{i=1}^{N} B_{\tilde{d}}(x_i,\epsilon) \times \prod_{i>N} \mathbb{R}
\end{equation*}
Then $V$ is open in the product topology, and $x\in V \subseteq B_D(x,\epsilon) \subseteq U$. Therefore, $U$ is open in the product topology.

Conversely, consider a basis element
\begin{equation*}
	U = \prod_{i=1}^{N} U_i \times \prod_{i>N} \mathbb{R}
\end{equation*}
of the product topology, where $U_i$ is open in $\mathbb{R}$ for $i\leq N$. For $x\in U$, we take $\epsilon_i$ that $B_{\tilde{d}}(x,\epsilon_i) \subseteq U_i$. Define
\begin{equation*}
\epsilon = \min \left\{ \epsilon_i / i : 1\leq i \leq N \right\}
\end{equation*}
Then $x\in B_D(x,\epsilon) \subseteq U$.
\end{proof}

Now we discuss the relation of metric spaces to other interesting properties.
\begin{proposition}{Properties of Metric Spaces}{Properties of Metric Spaces}
\begin{itemize}
\item Subspace: If $A$ is a subspace of a metric space $(X,d)$, then the restriction of $d$ on to $A \times A \rightarrow \mathbb{R}$ is a metric of $A$ that induces the subspace topology on $A$.
\item The Hausdorff Axiom: Every metric space is Hausdorff.
\item The product topology: Countable products of metric spaces are metrizable in the product topology.
\end{itemize}
\end{proposition}
\begin{proof}
\begin{itemize}
\item The subspace topology is generated by the basis elements of the form $B_d(x,\epsilon) \cap A$, which is just the open balls in the restricted metric.
\item Let $x,y\in X$ with $x\neq y$, then $d(x,y)>0$. Let $\epsilon = \frac{d(x,y)}{2}$, then $B_d(x,\epsilon) \cap B_d(y,\epsilon) = \emptyset$, so $X$ is Hausdorff.
\item This proof is similar (more to say, exactly the same) to that of theorem \ref{thm:Metrization of mathbbRmathbbZ+}. Let $d_i$ be the metric on each $X_{i}$, then we can define a metric on $\prod_{i=1}^{\infty } X_i$ by
\begin{equation*}
D(x,y) = \sup_{i\in \mathbb{Z}_+} \left\{ \frac{\tilde{d_i}(x_i,y_i)}{i} \right\}
\end{equation*}
where $\tilde{d_i}$ is the standard bounded metric on $X_i$. Then $D$ is a metric that induces the product topology on $\prod_{i=1}^{\infty } X_i$.
\end{itemize}
\end{proof}

About continuous functions there is more to say, as in analysis.

First, we shall see the familiar $\epsilon$-$\delta$ language can be carried out to describe arbitrary metric spaces, and so does the convergence-of-sequences language.

Next, we shall also construct continuous functions via the limit of a uniformly convergent sequence of continuous functions, which we took much care in analysis.

\begin{theorem}{The $\epsilon$-$\delta$ Language for Metric Spaces}{The epsilon-delta Language for Metric Spaces}
Let $(X,d_X)$ and $(Y,d_Y)$ be two metric spaces. A function $f: X \rightarrow Y$ is continuous iff 
\begin{equation}
\forall x\in X,\forall \epsilon>0, \exists \delta>0, \forall y\in X, (d_X(x,y)<\delta \rightarrow d_Y(f(x),f(y))<\epsilon)
\end{equation}
\end{theorem}
\begin{proof}
Quite similar to the $\mathbb{R} \rightarrow \mathbb{R}$ case.
\begin{itemize}
\item Suppose $f$ is continuous, Given $x$ and $\epsilon$, consider
	\begin{equation*}
		f^{-1}(B(f(x),\epsilon))
	\end{equation*}
	is open in $X$ and contains $x$. Then $\exists \delta>0,B(x,\delta) \subseteq f^{-1}(B(f(x),\epsilon))$, as desired.

\item Conversely, suppose the $\epsilon$-$\delta$ condition holds. Let $V$ be open in $Y$. Let $x\in f^{-1}(V)$, then $\exists \epsilon>0, B(f(x),\epsilon) \subseteq V$, then $\exists \delta>0, B(x,\delta) \subseteq f^{-1}(B(f(x),\epsilon)) \subseteq f^{-1}(V)$, so $f^{-1}(V)$ is open in $X$, so $f$ is continuous.
\end{itemize}
\end{proof}

From our intuition in analysis, if $x$ lies in the closure of $A \subseteq X$, then there is a sequence of points in $A$ converging to $x$. This is not true in general, but is in metric spaces.

\begin{lemma}{The Sequence Lemma}{The Sequence Lemma}
Let $X$ be a topological space. Let $A \subseteq X$. If there is a sequence of points of $A$ converging to $x\in X$, then $x\in \overline{A}$.

The converse holds if $X$ is a metric space.
\end{lemma}
\begin{proof}
\begin{itemize}
\item The first part is just the definition of closure, for if $x$ is a limit point of $A$, then $\forall U$ open containing $x$, $U\cap A \neq \emptyset$, so $x\in \overline{A}$.
\item Conversely, if $X$ is a metric space, and $x\in \overline{A}$, then take $x_n\in B(x,1 / n)\cap A$ for all $n\in \mathbb{Z}_+$. We assert that $x_n \rightarrow x$. For every open $U$ containing $x$ has a $x\in B(x,\epsilon) \subseteq U$, then we take $N > 1 / \epsilon$, then for all $n>N$, we have $x_n \in B(x,1/n) \subseteq B(x,\epsilon) \subseteq U$. So $x_n \rightarrow x$.
\end{itemize}
\end{proof}

\begin{theorem}{Continuity and Convergence}{Continuity and Convergence}
Let $f: X \rightarrow Y$. If $f$ is continuous, then for every convergent sequence $x_n \rightarrow x$ in $X$, we have $f(x_n) \rightarrow f(x)$ in $Y$.

The converse holds if $X$ is a metric space. (The metrizability of $Y$ is not required.)
\end{theorem}
\begin{proof}
\begin{itemize}
\item If $f$ continuous and $x_n \rightarrow x$, then let $V$ be a neighborhood of $f(x)$, then $f^{-1}(V)$ is a neighborhood of $x$. So $\exists N, \forall n>N,x_n\in f^{-1}(V), f(x_n)\in V$.

\item To prove the converse, we need to show that $f(\overline{A}) \subseteq \overline{f(A)}$. If $x\in \overline{A}$, then by lemma \ref{lem:The Sequence Lemma}, there is a sequence $x_n \in A$ converging to $x$. By the assumptions, we have $f(x_n) \rightarrow f(x)$, so $f(x) \in \overline{f(A)}$.
\end{itemize}
\end{proof}

\begin{remark}
	We notice that we do not need the full strength of metrizability of $X$ in the proof of lemma \ref{lem:The Sequence Lemma} and theorem \ref{thm:Continuity and Convergence}. The only place we use the metric is to show that there is a good-behaved basis $B(x,1 / n)$, so that every neighborhood of $x$ contains one of the basis. This fact leads us to a weaker condition, called first-countability.
\end{remark}

\begin{theorem}{The First Countability Axiom}{The First Countability Axiom}
	A topological space $X$ is first-countable (have a countable basis at every point $x$ ) if $\forall x\in X$, there is a countable collection $\left\{ U_n \right\}_{n\in \mathbb{Z}_+}$ of neighborhoods of $x$, such that $\forall $ neighborhood $U$ of $x$, $\exists n\in \mathbb{Z}_+$ such that $U_n \subseteq U$.
\end{theorem}

\begin{theorem}{First Countability and Continuity}{First Countability and Continuity}
If $X, Y$ are topological spaces, and $X$ is first countable. Let $f: X \rightarrow Y$. Then $f$ is continuous iff for every convergent sequence $x_n \rightarrow x$ in $X$, we have $f(x_n) \rightarrow f(x)$ in $Y$.
\end{theorem}
\begin{proof}
	Similar to that of lemma \ref{lem:The Sequence Lemma} and theorem \ref{thm:Continuity and Convergence}. Just replace the basis $B(x,1/n)$ with the countable collection $\left\{ U_n \right\}_{n\in \mathbb{Z}_+}$ of neighborhoods of $x$.
\end{proof} 

Next, we consider additional ways to construct continuous functions: by operations, just like in analysis.

\begin{theorem}{Operations of Continuous Functions}{Operations of Continuous Functions}
	If $X$ is a topological space, and $f,g: X \rightarrow \mathbb{R}$ are continuous, then $f+g, f-g, f \cdot g, f / g$ (if $\forall x\in X,g(x) \neq 0$) are continuous.
\end{theorem}
\begin{proof}
The map $h: X \rightarrow \mathbb{R} \times \mathbb{R}$ defined by
\begin{equation*}
h(x) = (f(x),g(x))
\end{equation*}
is continuous according to \ref{thm:Continuity of Products}. Then addition $+: \mathbb{R} \times \mathbb{R} \rightarrow \mathbb{R}$ is continuous, so $f+g = + \circ h$ is continuous. The others are similar.
\end{proof}

Now we come to uniformly convergence.

\begin{definition}{Uniformly Convergence}{Uniformly Convergence}
Let $X,Y$ be topological spaces and $Y$ has a metric $d$. Let $f_n: X \rightarrow Y$ be a sequence of functions. We say that the sequence $f_n$ converges uniformly to $f: X \rightarrow Y$ if
\begin{equation*}
\forall \epsilon>0, \exists N\in \mathbb{Z}_+, \forall n>N, \forall x\in X, d(f_n(x),f(x)) < \epsilon
\end{equation*}
\end{definition}

\begin{theorem}{Uniform Limit Theorem}{Uniform Limit Theorem}
Let $X,Y$ be topological spaces and $Y$ has a metric $d$. Let $f_n: X \rightarrow Y$ be a sequence of continuous functions. If $f_n$ converges uniformly to $f: X \rightarrow Y$, then $f$ is continuous.
\end{theorem}
\begin{proof}
This is similar of that in analysis. Let $V$ be open in $Y$ and $x_0\in f^{-1}(V)$, We want to find a neighborhood $U$ of $x_0$ such that $f(U) \subseteq V$.

Take $\epsilon>0$ that $B(f(x_0),\epsilon) \subseteq V$, then $\exists N\in \mathbb{Z}_+, \forall n>N, \forall x\in X$ we have
\begin{equation*}
d(f_n(x),f(x)) < \epsilon / 3
\end{equation*}
Then use the continuity of $f_N$, we take $U = f_N^{-1}(B(f_N(x_0), \frac{\epsilon}{3}))$, i.e. $f_N(U) \subseteq B(f_N(x_0), \frac{\epsilon}{3})$. We claim that $f(U) \subseteq V$. $\forall x\in U$, we have
\begin{equation*}
\begin{aligned}
	d(f(x),f_N(x)) < \epsilon / 3  &\quad\text{ by uniform convergence. } \\
	d(f_N(x),f_N(x_0)) < \epsilon / 3 &\quad\text{ by continuity of $f_N$.}\\
	d(f_N(x_0),f(x_0)) < \epsilon / 3 &\quad\text{ by uniform convergence. } \\
\end{aligned}
\end{equation*}
Through the triangle inequality, we have $d(f(x),f(x_0)) < \epsilon$, so $f(x) \in B(f(x_0),\epsilon) \subseteq V$. So $f(U) \subseteq V$, and $f$ is continuous.
\end{proof}

\begin{remark}
The concept of uniform convergence is relevant to the uniform metric $\tilde{\rho}$. Consider the space $\mathbb{R}^X$ of all functions $f:X \rightarrow \mathbb{R}$ with the uniform metric $\tilde{\rho}$, where $\rho(f,g) = \sup_{x\in X} |f(x)-g(x)|$. Then $f_n$ converges uniformly to $f$ iff $f_n$ converge to $f$ in the metric space $(\mathbb{R}^X,\tilde{\rho})$.
\begin{proof}
This is just the definition of uniform convergence.
\end{proof}
\end{remark}

Now we give some examples of spaces that are not metrizable.

\begin{example}{Spaces that are not Metrizable}{Spaces that are not Metrizable}
\begin{itemize}
\item $\mathbb{R}^{\mathbb{Z}_+}$ in the box topology is not metrizable.
	\begin{proof}
	We show that the sequence lemma \ref{lem:The Sequence Lemma} does not hold here. Let $A \subseteq \mathbb{R}^{\mathbb{Z}_+}$ consist of points whose coordinates are all positive:
	\begin{equation*}
	A = \left\{ (x_1,x_2, \ldots ): \forall i\in \mathbb{Z}_+,x_i>0 \right\}
	\end{equation*}
	\end{proof}
	We have $0\in \overline{A}$, obviously. However, there is no sequence that converge to $0$ in $A$. Let $a_n = (x_{1n}, x_{2n}, \ldots )$ be a sequence in $A$, that is $\forall i,n,x_{in}>0$. We let
	\begin{equation*}
	B = (-x_{11},x_{11}) \times (-x_{22},x_{22}) \times (-x_{33},x_{33}) \times \cdots
	\end{equation*}
	Then $B$ is open in the box topology, and $0\in B$, but $a_n \notin B$ for all $n$. So there is no sequence in $A$ converging to $0$.
\item Let $J$ be a uncountable set. The $\mathbb{R}^J$ is not metrizable in both the box and the product topology.
	\begin{proof}
	The box topology case follows the same as the previous one, just pick a countable $j_1, j_2, \ldots $ from $J$ would do. For any sequence $a_n$, consider the basis element $B$ whose coordinates are given:
	\begin{equation*}
		B_{j_n} = (-(a_n)_{j_n},(a_n)_{j_n}) \text{ for } i\in \mathbb{Z}_+, \text{ and the rest of the coordinates are all } \mathbb{R}
	\end{equation*}

	For the product topology, consider $A \subseteq \mathbb{R}^J$ consisting of all points $(x_{\alpha})$ such that $x_{\alpha}=1$ for all but finite many values of $\alpha$.

	We assert that $0\in \overline{A}$. Let $0\in\prod U_{\alpha}$ be a basis element, all but finite many $U_{\alpha} = \mathbb{R}$, say, except for $\alpha_1, \ldots ,\alpha_n$. Let $x_{\alpha} = 0$ for $\alpha = \alpha_1, \ldots ,\alpha_n$ and $x_{\alpha}=1$ for the rest. Then $(x_{\alpha})\in A\cap \prod U_{\alpha}$.

	Next we prove that there is no sequence in $A$ that converge to $0$. Let $a_n$ be a sequence of points in $A$. As in each $a_n$, there are only finite many coordinates that are not $1$, so there are only countable many coordinates that are not $1$ in the sequence. So $\exists \beta\in J$, such that $\forall n\in \mathbb{Z}_+,(a_n)_{\beta} = 1$. 

	Now let $U = \pi_{\beta}^{-1}((-1,1))$, then $U$ is a neighborhood of $0$ in the product topology. But $\forall n\in \mathbb{Z}_+, a_n \notin U$.
\end{proof}
\end{itemize}
\end{example}

NOTE: It is sometimes hard to visulize the uncountable-infinite $\mathbb{R}^J$. Think of it as all the functions $\mathbb{R} \rightarrow \mathbb{R}$ if it helps. The product topology basis elements are mostly $\mathbb{R}^2$ except for finite vertical lines, and the box topology basis elements are vertical intervals aligned on $x$-axis.


\section{The Quotient Topology}

The motivation of quotient topology comes in two ways: Geometrically, it is used for the ``cut and paste'' operation, such as gluing two points together, or identifying a point with a set of points. Algebraically, it is used to construct a new space from an equivalence relation on a given space, meaning that two points we see them as the same point.

A torus can be constructed by taking a square and gluing the opposite edges together.

\begin{figure}[ht]
    \centering
    \incfig{construction-of-a-torus}
    \caption{Construction of a torus}
    \label{fig:construction-of-a-torus}
\end{figure}

\begin{definition}{Quotient Map}{Quotient Map}
Let $X,Y$ be topological spaces. Let $p: X \rightarrow Y$ be a surjective map. Then $p$ is a quotient map iff
\begin{equation*}
	\forall U \subseteq Y, U\in \mathcal{T}_Y \Leftrightarrow p^{-1}(U) \in \mathcal{T}_X
\end{equation*}
\end{definition}

The condition of a quotient map is stronger than continuity. An equivalent restatement via closeness is obvious:
\begin{equation*}
	\forall V \subseteq Y, V \text{ is closed in } Y \Leftrightarrow p^{-1}(V) \text{ is closed in } X
\end{equation*}
If $p$ is bijective, then a quotient map $p$ is just a homeomorphism.

We also give a restatement via saturated set. Any subjective map provides an equivalent relation on $X$ by
\begin{equation*}
x\sim y \Leftrightarrow p(x) = p(y)
\end{equation*}
A saturated set is a set containing some equivalent classes of the relation.
\begin{definition}{Saturated sets}{Saturated sets}
A subset $A \subseteq X$ is saturated with respect to a surjective map $p: X \rightarrow Y$ iff $\exists B \subseteq Y,A = p^{-1}(B)$.

The saturation of a set $A$ with respect to $p$ is defined as $p^{-1}(p(A))$.
\end{definition}
To say that $p$ is a quotient map is equivalent to say that $p$ is continuous and $p$ maps all open saturated sets to open sets. Algebraically, quotient topologies are topologies constructed on the set of equivalent classes.

An important collection of quotient maps are continuous open maps/closed maps.
\begin{definition}{Open and Closed Maps}{Open and Closed Maps}
A map $p: X \rightarrow Y$ is called an open map if it maps open sets to open sets, i.e. $\forall U\in \mathcal{T}_X, p(U) \in \mathcal{T}_Y$.

A map $p: X \rightarrow Y$ is called a closed map if it maps closed sets to closed sets.
\end{definition}

\begin{remark}
To verify an open map we only need to check that the image of a basis element is open in the target space. For the fact that
\begin{equation*}
p(\bigcup_{\alpha\in J} B_{\alpha}) = \bigcup_{\alpha\in J} p(B_{\alpha})
\end{equation*}
\end{remark}

It is easy to see that continuous open maps and closed maps are quotient maps, but the converse is not true in general.

\begin{example}{Quotient Maps}{Quotient Maps}
\begin{itemize}
\item Let $X = [0,1]\cup [2,3]$, and $Y = [0,2]$, let
\begin{equation*}
p(x) = 
\begin{cases}
	x, &\text{ if } x\in [0,1] \\
	x-1, &\text{ if } x\in [2,3]
\end{cases}
\end{equation*}
then $p$ is surjective, continuous and closed, thus is a quotient map. However, it is not open, for $p([0,1]) = [0,1]$ is not open in $Y$.

If we let $A = [0,1)\cup [2,3]$, then restricting $p$ to $q: A \rightarrow Y$ is not a quotient map. For $q^{-1}([1,2]) = [2,3]$ is open in $A$.
\item $\pi_1: \mathbb{R} \times \mathbb{R} \rightarrow \mathbb{R}$ is continuous and surjective, and it is also open. For $U \times V$ is an nonempty basis element, then $\pi_1(U \times V) = U$ is open in $\mathbb{R}$. Thus $\pi_1$ is a quotient map. However, $\pi_1$ is not a closed map, for
	\begin{equation*}
	C = \left\{ (x,y)_p:xy=1 \right\}
	\end{equation*}
	is closed in $\mathbb{R}^2$, but $\pi_1(C) = \mathbb{R}-\left\{ 0 \right\}$ is not closed.

	Restricting $\pi_1$ to $A = C\cup \left\{ 0 \right\}$ is not a quotient map. For $\left\{ 0 \right\} = \pi_1^{-1}(\left\{ 0 \right\})$ is open in $\mathbb{R}$.
\end{itemize}
\end{example}

Now we show that quotient maps can induce a topology on the target space, called the quotient topology.

\begin{definition}{Quotient Topology}{Quotient Topology}
If $X$ is a topological space and $A$ is a set and $p:X \rightarrow A$ is a surjective map, then there exists exactly one topology on $A$ that makes $p$ a quotient map, called the quotient topology induced by $p$.

The topology $\mathcal{T}$ is given by: $U \in \mathcal{T} \Leftrightarrow p^{-1}(U) \in \mathcal{T}_X$.
\end{definition}
It is easy to see that $\mathcal{T}$ is indeed a topology. Using the fact $p^{-1}(\emptyset ) = \emptyset $, $p^{-1}(A)=X$, and
\begin{equation*}
	p^{-1}(\bigcup_{\alpha\in J} U_{\alpha}) = \bigcup_{\alpha\in J} p^{-1}(U_{\alpha})
\end{equation*}
\begin{equation*}
	p^{-1}(\bigcap_{i=1}^n U_i) = \bigcap_{i=1}^n p^{-1}(U_i)
\end{equation*}

\begin{example}{Quotient Topologies}{Quotient Topologies}
\begin{itemize}
\item Let $p: \mathbb{R}\rightarrow A$, where $A = \left\{ a,b,c \right\}$, given by
	\begin{equation*}
	p(x) =
	\begin{cases}
		a, &\text{ if } x>0 \\
		b, &\text{ if } x<0 \\
		c, &\text{ if } x=0
	\end{cases}
	\end{equation*}
	then the quotient topology is given by $\left\{ \emptyset ,\left\{ a \right\},\left\{ b \right\},\left\{ a,b \right\},\left\{ a,b,c \right\} \right\}$.
\end{itemize}
\end{example}

The set $A$ in the definition above is actually the set of equivalent classes of the equivalence relation induced by $p$. We can also define the quotient topology on a set of equivalent classes directly.

\begin{definition}{Quotient Spaces}{Quotient Spaces}
Let $X$ be a topological space, and $X^*$ be a partition of $X$ into disjoint subsets whose union is $X$. Let $p: X \rightarrow X^*$ be the surjective map that $p(x)$ is the element $U\in X^*,x\in U$. In the quotient topology induced by $p$, the space $X^*$ is called the quotient space.

If the partition is given by an equivalent relation $\sim$, we can denote $X^* = X / \sim$.
\end{definition}

\begin{example}{Quotient Spaces}{Quotient Spaces}
\begin{itemize}
\item Let $X = \left\{ (x,y)_p: x^2+y^2 \leq 1 \right\}$ be the unit closed disk in $\mathbb{R}^2$, then let $\sim$ be
	\begin{equation*}
	x\sim y \Leftrightarrow x=y\lor \|x\| = \|y\| = 1
	\end{equation*}
	Then we can show that $X / \sim$ is homeomorphic to $S^2$.
\item Let $X = [0,1] \times [0,1]$, and define $\sim$ be:
	\begin{equation*}
	a\sim b \Leftrightarrow a=b\lor (a=(0,y)_p\land b=(1,y))_p\lor (a=(x,0)_p\land b=(x,1)_p) \text{ for some }x,y
	\end{equation*}
	Then $X / \sim$ is homeomorphic to the torus $T^2$.
\end{itemize}
\end{example}

Now we consider the relation of quotient spaces with the previous topological properties.

We've noticed that subspaces do not behave well in the quotient topology. If $p: X \rightarrow Y$ is a quotient map, and $A \subseteq X$, then the restriction map $q: A \rightarrow p(A)$ may not be a quotient map. However, we have the following theorem.

\begin{theorem}{Subspaces of Quotient Spaces}{Subspaces of Quotient Spaces}
	Let $p: X \rightarrow Y$ be a quotient map, and $A \subseteq Y$ be saturated with respect to $p$. Let $q: A \rightarrow p(A)$ be the restriction of $p$ to $A$.
	\begin{itemize}
	\item If $A$ is either open or closed, the $q$ is a quotient map.
	\item If $p$ is either an open map or a closed map, then $q$ is a quotient map.
	\end{itemize}
\end{theorem}
\begin{proof}
\begin{itemize}
\item (Saturation) For either case, we have
	\begin{equation*}
	\begin{aligned}
		q^{-1}(V) = p^{-1}(V), & \text{ if } V \subseteq p(A) \\
		p(U\cap A) = p(U) \cap p(A), & \text{ if } U \subseteq X
	\end{aligned}
	\end{equation*}
	These lines come form saturation of $A$. (We have $p(U\cap A) \subseteq p(U)\cap p(A)$ for any $U,A$, and if $y=p(u)=p(a)$, then as $A$ is saturated, $u \in p^{-1}(p(a)) \subseteq A$, so $u\in U\cap A$.)
\item (Open $A$ or $p$ ) Given $A \subseteq p(A)$, assume $q^{-1}(V)$ is open in $A$:
	\begin{itemize}
	\item If $A$ is open, then $q^{-1}(V) = q^{-1}(V)$ is open in $X$, so $V$ is open in $Y$, so $V = V\cap A$ is open in $A$.
	\item If $p$ is open, then as $q^{-1}(V) = p^{-1}(V)$ is open in $A$, we have $p^{-1}(V) = U\cap A$ for some open $U \subseteq X$. For $p$ is surjective, we have
		\begin{equation*}
		V = p(p^{-1}(V)) = p(U\cap A) = p(U) \cap p(A)
		\end{equation*}
	Then as $p(U)$ is open in $Y$, we have $V$ is open in $p(A)$.
	\end{itemize}
\item For closed $A$ or $p$ it is similar, just replace open with closed.
\end{itemize}
\end{proof}

The products of quotient spaces do not behave well. The Cartesian product of two quotient maps may not be a quotient map. To make it work, we need to use additional conditions, such as local compactness, or when $p,q$ are both open maps. (The latter is easy to verify, as the product of two open maps is an open map.)

The product of two maps is defined: If $p:X_1 \rightarrow Y_1,q:X_2 \rightarrow Y_2$, then $p \times q: X_1 \times X_2 \rightarrow Y_1 \times Y_2$ is defined by $p \times q (x_1,x_2) = (p(x_1),q(x_2))$.

The Hausdorff condition is also not preserved in the quotient topology. The product of two Hausdorff spaces is Hausdorff, but the quotient of a Hausdorff space may not be Hausdorff.

The interesting part took place in the study of continuous functions on the quotient space. We've studies whether a map $f: Z \rightarrow \prod X_{\alpha}$ is continuous, and the counterpart is to determine the continuity of $f: X^* \rightarrow Z$ out of a quotient space.

\begin{theorem}{Continuity on the Quotient Space}{Continuity on the Quotient Space}
Let $p:X \rightarrow Y$ be a quotient map, and $Z$ be a space, $g:X \rightarrow Z$ be a map that is constant for each $p^{-1}(\left\{ y \right\}),\forall y\in Y$. (In this case, all the element in an equivalent class have the same target). Then $\exists f: Y \rightarrow Z$ that $g = f\circ p$.
\begin{itemize}
\item $f$ is continuous iff $g$ is continuous.
\item $f$ is a quotient map iff $g$ is a quotient map.
\end{itemize}
\end{theorem}
\begin{proof}
\begin{itemize}
\item Using the fact
	\begin{equation*}
	g^{-1}(V) = p^{-1}(f^{-1}(V))
	\end{equation*}
	and $p$ being a quotient map would do.
\item If $f$ is a quotient map, then $g = f\circ p$ is also a quotient map. Conversely, if $g$ is a quotient map, then $f$ is surjective, Let $V \subseteq Z$, if $f^{-1}(V)$ is open, then $p^{-1}(f^{-1}(V)) = g^{-1}(V)$ is open, so $V$ is open.
\end{itemize}
\end{proof}

\begin{figure}[ht]
    \centering
    \incfig{continuity-on-quotient-spaces}
    \caption{Continuity on Quotient Spaces}
    \label{fig:continuity-on-quotient-spaces}
\end{figure}

\begin{remark}
	If we take $Y = X^* = \left\{ g^{-1}(\left\{ z \right\}) : z\in Z \right\}$, then $X^*$ is just the equivalent classes by $g$, and $f$ becomes a homeomorphism.
\end{remark}

\begin{example}{The Product of two quotient map may not be a quotient map}{The Product of two quotient map may not be a quotient map}
Let $X = \mathbb{R}$ and $\sim: x\sim y \Leftrightarrow x=y\lor x,y\in \mathbb{Z}_+$. The equivalent class of all $\mathbb{Z}_+$ is denoted $b$. Let $p: X \rightarrow X^*$ be the quotient map, and $i: \mathbb{Q}\rightarrow \mathbb{Q}$ be the identity. We show that
\begin{equation*}
p \times i: X \times \mathbb{Q} \rightarrow X^* \times \mathbb{Q}
\end{equation*}
is not a quotient map.

\begin{proof}
	Consider $U$ be the following region \ref{fig:the-region-u}: where $c_n = \sqrt{2} / n$, and the width of each strip is $1 / 2$. Then $U$ contains each $\left\{ n \right\} \times \mathbb{Q}$ for $c_n $ is not rational. 

	Then $U$ is open in $X \times \mathbb{Q}$, it is also saturated for it contains all $\mathbb{Z}_+ \times \mathbb{Q}$. If $p \times i$ is a quotient map, then $U' = p \times i(U)$ is open in $X^* \times \mathbb{Q}$.

	Consider $(b,0)_p \in p \times i(U) \subseteq  X^* \times \mathbb{Q}$. Then $U'$ contains some $W \times I_{\delta}$. Where $W$ is some neighborhood of $b$ in $X^*$ and $I_{\delta} = \left\{ y\in \mathbb{Q}: |y| < \delta \right\}$. Choose $N$ sufficiently large so that $c_N < \delta$. As $p^{-1}(W)$ is open in $X$ and contains $\mathbb{Z}_+$, then $(n,0)_p \in p^{-1}(W)$. We can choose $\epsilon<\frac{1}{4}$ that $V = (N-\epsilon,N+\epsilon) \times I_{\delta} \subseteq  p^{-1}(W) \subseteq U$. But the figure shows that there are points that do not lie in $V$, which contradicts.
\end{proof}
\end{example}

\begin{figure}[ht]
    \centering
    \incfig{the-region-u}
    \caption{The Region U}
    \label{fig:the-region-u}
\end{figure}

\section{A Note on Topological Groups}

So far, we have seen that the quotient spaces behave similar to the quotient groups in algebra. In fact, the quotient topology gets its name from the quotient of a topological group by a subgroup.

\begin{definition}{Topological Groups}{Topological Groups}
	A topological group is a group $G$ with a topology $\mathcal{T}$ such that the following holds:
	\begin{itemize}
	\item $G$ satisfies the $T_1$-axiom.
	\item The group operation $\mu : G \times G \rightarrow G, (x,y)_p \mapsto x \cdot y$ is continuous.
	\item The inverse map $\iota : G \rightarrow G, x \mapsto x^{-1}$ is continuous.
	\end{itemize}
\end{definition}

\end{document}
