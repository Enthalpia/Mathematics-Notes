\documentclass[../main.tex]{subfiles}

\begin{document}
\chapter{The Tychonoff Theorem}

\section{The Tychonoff Theorem}
In previous chapters, we discussed the compactness of the product of two compact spaces $X \times Y$, by covering sliced $x \times Y$ by finite subcovers and then using these to cover $X \times Y$. However, it is rather tricky to extend this argument to an arbitrary product of compact spaces, one must well-order the index set and use transfinite induction. Another way is to tackle the problem is to use the closed set definition of compactness, using Zorn's lemma.

For simplicity, consider the product of two compact spaces $X_1 \times X_2$. Let $\mathcal{A}$ be a collection of closed subsets of $X_1 \times X_2$ with the finite intersection property, that is, any finite intersection of elements of $\mathcal{A}$ is non-empty. We want to show that the intersection of all elements of $\mathcal{A}$ is non-empty.

\begin{itemize}
	\item The projection collection $\{\pi_1(A) : A \in \mathcal{A}\}$ is a collection of closed subsets of $X$ with the finite intersection property, so are their closure $\{\overline{\pi_1(A)} : A \in \mathcal{A}\}$. Since $X$ is compact, $\bigcap_{A \in \mathcal{A}} \overline{\pi_1(A)}$ is non-empty. The result holds for $\pi_2$ also.
	\item Take $x_1 \in \bigcap_{A \in \mathcal{A}} \overline{\pi_1(A)}$ and $x_2 \in \bigcap_{A \in \mathcal{A}} \overline{\pi_2(A)}$. We wish that $x_1 \times x_2 \in \bigcap_{A \in \mathcal{A}} A$. However, this is not necessarily true. Consider the following counterexample: $X_1=X_2=[0,1]$ and $p = (1 / 3,1 / 3)_p, q = (1 / 2,2 / 3)_p$, then $\mathcal{A}$ being all elliptical closed sets with $p,q$ as focii has the finite intersection property, but we show that $(1 / 2, 1 / 2)_p \notin A$ for any $A \in \mathcal{A}$. To get a valid point, we need to choose on the segment $pq$.

	So we need a better choice of $x_1, x_2$, the current choice is to free.
	\item We can expand the collection $\mathcal{A}$ to $\mathcal{D}$, retaining the finite intersection property. For example, we can add all all circles with $p$ as center. This forces us to choose $x_1 = 1 / 3$ and $x_2 = 1 / 3$. The new collection still has the finite intersection property.
	\item To determine the correct expansion, considering choosing $\mathcal{D}$ ``as large as possible'', so that no larger collection containing $\mathcal{D}$ has the finite intersection property. This is where Zorn's lemma comes in.
\end{itemize}

\begin{remark}
	Note that we do not use the closedness of elements of $\mathcal{A}$ until the very end, when we need to show the closed set formulation of compactness. We may as well begin with an arbitrary collection of subsets of $X$ with the finite intersection property, and expand it to a maximal such collection.
\end{remark}

\begin{figure}[ht]
    \centering
    \incfig{conterexample-for-product-common-point}
    \caption{Conterexample for Product Common Point}
    \label{fig:conterexample-for-product-common-point}
\end{figure}

\begin{lemma}{Maximal Collection with Finite Intersection Property}{Maximal Collection with Finite Intersection Property}
	Let $X$ be any set, and $\mathcal{A} \subseteq P(X)$ having the finite intersection property. Then there exists a collection $\mathcal{D} \subseteq P(X)$ such that $\mathcal{A} \subseteq \mathcal{D}$, $\mathcal{D}$ has the finite intersection property, and any collection of subsets of $X$ properly containing $\mathcal{D}$ does not have the finite intersection property.

	The collection $\mathcal{D}$ is called a \textbf{maximal collection with the finite intersection property} containing $\mathcal{A}$. Then
	\begin{itemize}
		\item Any finite intersection of elements of $\mathcal{D}$ is an element of $\mathcal{D}$.
		\item If $A \subseteq X$ such that $A \cap D \neq \emptyset$ for all $D \in \mathcal{D}$, then $A \in \mathcal{D}$.
	\end{itemize}
\end{lemma}
\begin{proof}
	Let the set
	\begin{equation*}
		\mathscr{A} = \{\mathcal{C} \subseteq P(X) : \mathcal{A} \subseteq \mathcal{C}, \mathcal{C} \text{ has the finite intersection property}\}.
	\end{equation*}
	The partial order is given by set inclusion. For any chain $\mathscr{B} \subseteq \mathscr{A}$, let
	\begin{equation*}
		\mathcal{E} = \bigcup_{\mathcal{C} \in \mathscr{B}} \mathcal{C}.
	\end{equation*}
	We shall show that $\mathcal{E} \in \mathscr{A}$, so that $\mathcal{E}$ is an upper bound of $\mathscr{B}$. Clearly, $\mathcal{A} \subseteq \mathcal{E}$. To show that $\mathcal{E}$ has the finite intersection property, let $E_1, E_2, \ldots, E_n \in \mathcal{E}$. Then for each $i$, there exists $\mathcal{C}_i \in \mathscr{B}$ such that $E_i \in \mathcal{C}_i$. Since $\mathscr{B}$ is a chain, there exists some $\mathcal{C} \in \mathscr{B}$ such that $\mathcal{C}_i \subseteq \mathcal{C}$ for all $i$. Thus $E_i \in \mathcal{C}$ for all $i$, and since $\mathcal{C}$ has the finite intersection property, we have $\bigcap_{i=1}^n E_i \neq \emptyset$. By Zorn's lemma, $\mathscr{A}$ has a maximal element $\mathcal{D}$, which is the desired collection.

	The two properties follows smoothly from the maximally of $\mathcal{D}$. For the first property, if not, joining the finite intersection to $\mathcal{D}$ gives a larger collection with the finite intersection property, contradicting maximality. For the second property, if not, joining $A$ to $\mathcal{D}$ gives a larger collection with the finite intersection property, contradicting maximality.
\end{proof}

\begin{remark}
	The maximal collection $\mathcal{D}$ is not unique. For example, if $X = \{1,2\}$ and $\mathcal{A} = \{X\}$, then both $\mathcal{D}_1 = \{\{1\}, X\}$ and $\mathcal{D}_2 = \{\{2\}, X\}$ are maximal collections with the finite intersection property containing $\mathcal{A}$.
\end{remark}

\begin{theorem}{Tychonoff Theorem}{Tychonoff Theorem}
	Let $\{X_\alpha : \alpha \in J\}$ be an indexed family of compact topological spaces. Then the product space $X = \prod_{\alpha \in J} X_\alpha$ is compact in the product topology.
\end{theorem}
\begin{proof}
	Let $\mathcal{A}$ be a collection of subsets of $X$ with the finite intersection property. We prove the collection $\bigcap_{A\in \mathcal{A}} \overline{A}$ is non-empty, where the closure is taken in $X$. By Lemma \ref{lem:Maximal Collection with Finite Intersection Property}, choose $\mathcal{D}$ a maximal collection with the finite intersection property containing $\mathcal{A}$. For each $\alpha \in J$, let $\pi_{\alpha}$ be the projection map from $X$ to $X_{\alpha}$, and let
	\begin{equation*}
		x_{\alpha} \in \bigcap_{D \in \mathcal{D}} \overline{\pi_{\alpha}(D)},
	\end{equation*}
	the intersection is not empty since $\{\overline{\pi_{\alpha}(D)} : D \in \mathcal{D}\}$ is a collection of closed subsets of the compact space $X_{\alpha}$ with the finite intersection property. Let $x = (x_{\alpha})_{\alpha \in J} \in X$. We show that $x \in \overline{D}$ for all $D \in \mathcal{D}$, which implies that $x \in \overline{A}$ for all $A \in \mathcal{A}$ since $\mathcal{A} \subseteq \mathcal{D}$.

	From the second property of Lemma \ref{lem:Maximal Collection with Finite Intersection Property}, we know that all subbasis $\pi_{\beta}^{-1} (U_{\beta})$ containing $x$ intersects $D$, because $x_{\beta} \in \overline{\pi_{\beta}(D)}$ implies $\pi_{\beta}(D) \cap U_{\beta} \neq \emptyset$. Since finite intersections of subbasis elements form a basis, any basis element containing $x$ intersects $D$. Thus $x \in \overline{D}$ for all $D \in \mathcal{D}$.
\end{proof}


\section{The Stone-\v{C}ech Compactification}

We have already studied one-point compactification, which is some sense the minimal compactification. The Stone-\v{C}ech compactification is the maximal compactification.

\begin{definition}{Compactification}{Compactification}
	A compactification of a space $X$ is a compact Hausdorff space $Y$ that $\overline{X}=Y$. Two compactification $Y_1, Y_2$ of $X$ are said to be equivalent if there exists a homeomorphism $f : Y_1 \to Y_2$ such that $f(x) = x$ for all $x \in X$.
\end{definition}

If $X$ has a compactification, then $X$ is completely regular because $Y$ is completely regular. Conversely, if $X$ is completely regular, then $X$ has a compactification because $X$ can be embedded in some $[0,1]^J$.

\begin{lemma}{Compactification by Imbedding}{Compactification by Imbedding}
	Let $X$ be a space, and $h: X \rightarrow Z$ be an imbedding of $X$ into a compact Hausdorff space $Z$. Then there is a compactification $Y$ of $X$ that there is an imbedding $H: Y \rightarrow Z$ that $H|_X = h$. $Y$ is uniquely determined up to equivalence.

	We call $Y$ the compactification of $X$ induced by the imbedding $h$.
\end{lemma}
\begin{proof}
	Let $X_0 = h(X) \subseteq Z$, and $Y_0 = \overline{X_0} \subseteq Z$. Then $Y_0$ is compact Hausdorff so is a compactification of $X_0$.

	Now we construct $Y$: let $Y$ be the set $X \cup (Y_0 - X_0)$, and define $H : Y \to Y_0$ by
	\begin{equation*}
		H(y) = \begin{cases}
			h(y) & y \in X \\
			y & y \in Y_0 - X_0
		\end{cases}.
	\end{equation*}
	Then $H$ is a bijection. We give $Y$ the topology such that $H$ is a homeomorphism. Then $Y$ is a compactification of $X$, and $H|_X = h$.
\end{proof}

\begin{proposition}{Criterion for Compactifiablility}{Criterion for Compactifiablility}
	A space $X$ has a compactification if and only if $X$ is completely regular.
\end{proposition}

Usually there are various ways to compactify a given space.
\begin{example}{Compactification}{Compactification}
	Compactify $(0,1) \subseteq \mathbb{R}$:
	\begin{itemize}
		\item Take $h:(0,1) \rightarrow S^1$ by $h(x) = (\cos 2\pi x, \sin 2\pi x)$, then the compactification induced by $h$ is $S^1$, which is just the one-point compactification.
		\item Take $h:(0,1) \rightarrow [0,1]$ by $h(x) = x$, then the compactification induced by $h$ is $[0,1]$.
		\item Take $h:(0,1) \rightarrow [0,1]^2$ by $h(x) = (x, \sin 1 / x)_p$, then the compactification induced by $h$ is the topologist's sine curve.
	\end{itemize}
\end{example}

For many occasions, we wish that a continuous real function on $X$ can be extended to a continuous real function on the compactification. First, $f$ must be bounded, since the image $f(Y)$ is compact in $\mathbb{R}$ thus is bounded. But this is not sufficient, taking $f(x) = \sin 1 / x$ on $(0,1)$ for example.
\begin{remark}
	In the third case, $f$ can be extended to the compactification: Take the composite function $\pi_2\circ H$, that $x \rightarrow x \times \sin 1 / x \rightarrow \sin 1 / x$.

	But in the first and second case, it can't.

	Actually, this gives us an idea, that to extend a whole collection of bounded continuous functions on $X$, we can use them as component functions for an imbedding on $\mathbb{R}^J$ for some $J$, and thereby get a compactification for which every function in the collection can be extended.
\end{remark}
This gives us the intuition behind the Stone-\v{C}ech compactification.

\begin{theorem}{Stone-\v{C}ech Compactification}{Stone-vCech Compactification}
	Let $X$ be a completely regular space. There exists a compactification of $X$ that every bounded continuous map $f: X \rightarrow \mathbb{R}$ extends uniquely to a continuous map $\overline{f}: Y \rightarrow \mathbb{R}$.
\end{theorem}
\begin{proof}
	Let $\left\{ f_{\alpha} \right\}_{\alpha\in J}$ be the collection of all bounded continuous functions from $X$ to $\mathbb{R}$. For each $\alpha \in J$, let $I_{\alpha}$ be a closed bounded interval containing the image of $f_{\alpha}$. Specifically, choose
	\begin{equation*}
		I_{\alpha} = [\inf f_{\alpha}(X), \sup f_{\alpha}(X)].
	\end{equation*}
	Then define
	\begin{equation*}
		h: X \rightarrow \prod_{\alpha \in J} I_{\alpha}, \qquad h(x) = (f_{\alpha}(x))_{\alpha \in J}.
	\end{equation*}
	By the Tychonoff theorem, $\prod_{\alpha \in J} I_{\alpha}$ is compact Hausdorff. Since $X$ is completely regular, $h$ separates points from closed sets, so $h$ is an imbedding. Let $Y$ be the compactification of $X$ induced by $h$. Then there is a unique imbedding $H: Y \rightarrow \prod_{\alpha \in J} I_{\alpha}$ such that $H|_X = h$. Take $\overline{f} = \pi_{\alpha} \circ H$, where $\pi_{\alpha}$ is the projection map from $\prod_{\alpha \in J} I_{\alpha}$ to $I_{\alpha}$. Then $\overline{f}$ is the desired extension of $f$.
\end{proof}

The uniqueness follows from the next lemma.
\begin{lemma}{Uniqueness of Extended Functions}{Uniqueness of Extended Functions}
	Let $A \subseteq X$, and $f:A \rightarrow Z$ be a continuous map, where $Z$ is Hausdorff. Then there exists at most one continuous map $\overline{f}: \overline{A} \rightarrow Z$ such that $\overline{f}|_A = f$.
\end{lemma}
\begin{proof}
	Suppose that there are two such maps $\overline{f}_1, \overline{f}_2$. Let $x \in \overline{A}$, such that $\overline{f}_1(x) \neq \overline{f}_2(x)$. Since $Z$ is Hausdorff, there exist disjoint open neighborhoods $U_1, U_2$ of $\overline{f}_1(x), \overline{f}_2(x)$ respectively. Then $\overline{f}_1^{-1}(U_1)$ and $\overline{f}_2^{-1}(U_2)$ are disjoint open neighborhoods of $x$ in $\overline{A}$. Since $x \in \overline{A}$, there exists some $a \in A$ such that $a \in \overline{f}_1^{-1}(U_1) \cap \overline{f}_2^{-1}(U_2)$. But then $\overline{f}_1(a) = f(a) = \overline{f}_2(a)$, contradicting the disjointness of $U_1$ and $U_2$. Thus $\overline{f}_1 = \overline{f}_2$.
\end{proof}

\begin{theorem}{Universal Property of Stone-\v{C}ech Compactification}{Universal Property of Stone-vCech Compactification}
	Let $X$ be a completely regular space, and $Y$ be the Stone-\v{C}ech compactification of $X$. Given any continuous map $f: X \rightarrow C$, where $C$ is a compact Hausdorff space, there exists a unique continuous map $\overline{f}: Y \rightarrow C$ such that $\overline{f}|_X = f$.
\end{theorem}
\begin{proof}
	Note that $C$ is completely regular so can be imbeded into $[0,1]^J$ form some $J$. So assume $C \subseteq [0,1]^J$, the for each bounded real $f_{\alpha}$, there is a unique extension $g_{\alpha}:Y \rightarrow \mathbb{R}$, and $g:Y \rightarrow \mathbb{R}^J$ where $g(y) = (g_{\alpha}(y))_{\alpha \in J}$. Then $g|_X = f$, and $g(Y) \subseteq C$ since $g$ is continuous and
	\begin{equation*}
		g(Y) = g(\overline{X}) \subseteq \overline{g(X)} = \overline{f(X)} \subseteq \overline{C} = C.
	\end{equation*}
	thereby giving the desired extension $\overline{f} = g$. The uniqueness follows from the uniqueness of each $g_{\alpha}$.
\end{proof}

\begin{theorem}{Uniqueness of Stone-\v{C}ech Compactification}{Uniqueness of Stone-vCech Compactification}
	Let $X$ be a completely regular space, and $Y_1, Y_2$ be two Stone-\v{C}ech compactifications of $X$. Then $Y_1$ and $Y_2$ are equivalent.	
\end{theorem}
\begin{proof}
	Consider the inclusion $j_2: X \rightarrow Y_2$, which is a continuous map, from theorem \ref{thm:Universal Property of Stone-vCech Compactification}, we may extend $j_2$ to $f_2: Y_1 \rightarrow Y_2$ being a continous map. Similarly, we may extend the inclusion $j_1: X \rightarrow Y_1$ to $f_1: Y_2 \rightarrow Y_1$ being a continuous map. Then $f_1 \circ f_2: Y_1 \rightarrow Y_1$ is a continuous map such that $(f_1 \circ f_2)|_X = id_X$. By the uniqueness of extension, $f_1 \circ f_2 = id_{Y_1}$. Similarly, $f_2 \circ f_1 = id_{Y_2}$. Thus $f_1, f_2$ are homeomorphisms, and $Y_1, Y_2$ are equivalent.
\end{proof}

\end{document}
