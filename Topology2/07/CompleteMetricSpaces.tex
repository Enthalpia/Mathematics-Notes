\documentclass[../main.tex]{subfiles}

\begin{document}
\chapter{Complete Metric Spaces and Function Spaces}
Completeness is a fundamental concept in analysis, referring to a property of metric spaces that, while metric in nature, underlies many important topological theorems. The most familiar example is Euclidean space, but another key example is the space $C(X, Y)$ of all continuous functions from a space $X$ to a metric space $Y$, which is complete in the uniform metric if $Y$ is complete. This chapter examines such examples, demonstrates the completeness of $C(X, Y)$ in the uniform metric, and constructs the Peano space-filling curve as an application. It also explores the relationship between compactness and completeness, leading to Ascoli's theorem about compact subsets of function spaces. Finally, the chapter discusses alternative topologies on $C(X, Y)$ and proves a general version of Ascoli's theorem.

\section{Complete Metric Spaces}

\begin{definition}{Cauchy Sequence and Completeness}{Cauchy Sequence and Completeness}
	Let $(X,d)$ be a metric space, and a sequence $(x_n)$ in $X$ is a Cauchy sequence in $(X,d)$ if it satisfies:
	\begin{equation}
		\forall \epsilon > 0, \exists N \in \mathbb{N}, \forall m,n > N, d(x_n, x_m) < \epsilon.
	\end{equation}
	A metric space $(X,d)$ is complete if every Cauchy sequence in $X$ converges to a limit in $X$.
\end{definition}

\begin{remark}
	Of course every convergent sequence is a Cauchy sequence, but the converse is not true in general. For example, the space of rational numbers $\mathbb{Q}$ with the usual metric is not complete.
\end{remark}

\begin{theorem}{Closed Subspace of Complete Spaces}{Closed Subspace of Complete Spaces}
	In a complete space, any closed subspace is complete, and any complete subspace is closed.
\end{theorem}
\begin{proof}
	First, a Cauchy sequence in the closed subspace is also a Cauchy sequence in the complete space, so it converges to some point in the complete space. Since the subspace is closed, the limit point is also in the subspace.

	Next, let $Y$ be a complete subspace of a complete space $X$. Let $x$ be a limit point of $Y$, then there exists a sequence $(y_n)$ in $Y$ that converges to $x$. So $(y_n)$ is a Cauchy sequence in $Y$, and it converges to some point in $Y$. But the limit is unique, so $x \in Y$.
\end{proof}

\begin{proposition}{Standard Bounded Metric and Completeness}{Standard Bounded Metric and Completeness}
	$X$ is complete under $d$ if and only if $X$ is complete under the standard bounded metric $\overline{d}(x,y) = \min \left\{d(x,y),1  \right\}$.
\end{proposition}
This is quite obvious as a sequence is Cauchy under $d$ if and only if it is Cauchy under $\overline{d}$, and a sequence converges under $d$ if and only if it converges under $\overline{d}$.

\begin{lemma}{Subsequence Criterion for Completeness}{Subsequence Criterion for Completeness}
	A metric space $(X,d)$ is complete if and only if every Cauchy sequence in $X$ has a convergent subsequence.
\end{lemma}
\begin{proof}
	The $\Rightarrow$ side obvious, taking the original sequence would do.

	For the $\Leftarrow$ side, let $(x_n)$ be a Cauchy sequence in $X$. Let $(x_{n_i})$ is a convergent subsequence of $(x_n)$, and let $x = \lim_{i \to \infty} x_{n_i}$. For any $\epsilon > 0$, there exists $N$ that
	\begin{equation*}
		d(x_n,x_m) < \epsilon/2, \forall m,n > N.
	\end{equation*}
	Also, there exists $I$ that $n_I>N$ and
	\begin{equation*}
		d(x_{n_i}, x) < \epsilon/2, \forall i > I.
	\end{equation*}
	So for any $n > N$, we have
	\begin{equation*}
		d(x_n, x) \leq d(x_n, x_{n_I}) + d(x_{n_I}, x) < \epsilon/2 + \epsilon/2 = \epsilon.
	\end{equation*}
\end{proof}

\begin{theorem}{Completeness of $\mathbb{R}^k$}{Completeness of mathbbRk}
	$\mathbb{R}^k$ is complete under the usual metric $d(x,y) = \sqrt{\sum_{i=1}^k (x_i - y_i)^2}$ and the square metric $\rho(x,y) = \max_{1 \leq i \leq k} |x_i - y_i|$.
\end{theorem}
\begin{proof}
	\textbf{Proof of $(\mathbb{R}^k,\rho)$}, let $(x_n)$ be a Cauchy sequence in $(\mathbb{R}^k,\rho)$, it is easy to see that $x_n$ is bounded in some cube $[-M,M]^k$. Since $[-M,M]^k$ is compact, it is sequentially compact from theorem \ref{thm:Equivalence of Compactness in Metric Spaces}. So $(x_n)$ has a convergent subsequence $(x_{n_i})$ that converges to some $x \in [-M,M]^k$. From lemma \ref{lem:Subsequence Criterion for Completeness}, $(x_n)$ converges to $x$.

	\textbf{Proof of $(\mathbb{R}^k,d)$}, equivalently.
\end{proof}

\begin{theorem}{Compact Metric Spaces are Complete}{Compact Metric Spaces are Complete}
	Every compact metric space is complete.
\end{theorem}
\begin{proof}
	From theorem \ref{thm:Equivalence of Compactness in Metric Spaces} and lemma \ref{lem:Subsequence Criterion for Completeness}.
\end{proof}

Now we deal with the product space $\mathbb{R}^{\mathbb{Z}_+}$.

\begin{lemma}{Convergence in Product Space}{Convergence in Product Space}
	Let $X = \prod X_{\alpha}$ be a product space, and $(x_n)$ be a sequence in $X$. Then $(x_n)$ converges to $x \in X$ if and only if for each $\alpha$, the sequence $(\pi_{\alpha}(x_n))$ converges to $\pi_{\alpha}(x)$ in $X_{\alpha}$.
\end{lemma}
\begin{proof}
	The $\Rightarrow$ side is obvious as continuity preserve convergence.

	For the $\Leftarrow$ side, let $U = \prod U_{\alpha}$ be a basis containing $x$, for $U_{\alpha} \neq X_{\alpha}$, choose $N_{\alpha}$ that $\pi_{\alpha}(x_n) \in U_{\alpha}$ for all $n > N_{\alpha}$. Let $N = \max \{N_{\alpha}\}$, then for all $n > N$, $x_n \in U$.
\end{proof}

\begin{theorem}{Compactness of $\mathbb{R}^{\mathbb{Z}_+}$}{Compactness of mathbbRmathbbZ+}
	There is a metric for $\mathbb{R}^{\mathbb{Z}_+}$ that makes $\mathbb{R}^{\mathbb{Z}_+}$ a complete metric space.
\end{theorem}
\begin{proof}
	Let $\overline{d}$ be the standard bounded metric on $\mathbb{R}$, and define a metric $D$ on $\mathbb{R}^{\mathbb{Z}_+}$ by
	\begin{equation}
		D(x,y) = \sup \left\{\frac{\overline{d}(x_n,y_n)}{n} : n \in \mathbb{Z}_+ \right\}.
	\end{equation}
	Then $D$ induces the product topology on $\mathbb{R}^{\mathbb{Z}_+}$, from theorem \ref{thm:Metrization of mathbbRmathbbZ+}.

	Now we prove that $\mathbb{R}^{\mathbb{Z}_+}$ is complete with $D$: Take $x_n$ be a Cauchy sequence, and
	\begin{equation*}
		\overline{d}(\pi_i(x), \pi_i(y)) \leq i D(x,y).
	\end{equation*}
	So for a fixed $i$, $(\pi_i(x_n))$ is a Cauchy sequence in $(\mathbb{R},\overline{d})$. From lemma \ref{lem:Convergence in Product Space} we have that $(x_n)$ converges.
\end{proof}

\begin{example}{Non-complete Spaces}{Non-complete Spaces}
	\begin{itemize}
		\item The space $\mathbb{Q}$ of rational numbers with the usual metric is not complete. For example, the sequence
		      \begin{equation*}
			      1, 1.4, 1.41, 1.414, 1.4142, \ldots
		      \end{equation*}
		      is a Cauchy sequence in $\mathbb{Q}$ that does not converge to a rational number.
		\item The space $(-1,1)$ is not complete under the usual metric. For example, the sequence
		      \begin{equation*}
			      0, 0.9, 0.99, 0.999, 0.9999, \ldots
		      \end{equation*}
		      is a Cauchy sequence in $(-1,1)$ that does not converge to a point in $(-1,1)$.
	\end{itemize}
\end{example}

We now turn to $\mathbb{R}^J$ when $J$ is not countable. Well actually, this is not even metrizable with the product topology, see example \ref{exp:Spaces that are not Metrizable}. We turn to the topology induced by the uniform metric.

From definition \ref{def:Uniform Metric}, we know that if $(Y,d)$ is a topological space, then $\overline{\rho}$ is a metric defined by
\begin{equation}
	\overline{\rho}(x,y) = \sup \{\overline{d}(x_{\alpha}, y_{\alpha}) : \alpha \in J \}, \quad \overline{d} = \min \{d,1\}.
\end{equation}
Or written in function form, $Y^J$ is the set of all functions from $J$ to $Y$, and
\begin{equation}
	\overline{\rho}(f,g) = \sup \{\overline{d}(f(\alpha), g(\alpha)) : \alpha \in J \}, \quad \overline{d} = \min \{d,1\}.
\end{equation}

\begin{theorem}{Completeness in Uncountable Products}{Completeness in Uncountable Products}
	If $(Y,d)$ is a complete metric space, then $(Y^J, \overline{\rho})$ is a complete metric space, for any set $J$.
\end{theorem}
\begin{proof}
	If $(Y,d)$ is complete, so is $(Y,\overline{d})$. Suppose $f_n$ is a Cauchy sequence in $(Y^J, \overline{\rho})$. For each $\alpha \in J$, $f_n(\alpha)$ is a Cauchy sequence in $(Y,\overline{d})$, so it converges to some $f(\alpha) \in Y$. Let $f$ be the function defined by $f(\alpha) = \lim_{n \to \infty} f_n(\alpha)$ for each $\alpha \in J$. We show that $f_n$ converges to $f$ in $(Y^J, \overline{\rho})$:

	For any $\epsilon > 0$, there exists $N$ such that for all $m,n > N$, $\overline{\rho}(f_n, f_m) < \epsilon / 2$. We have
	\begin{equation*}
		\overline{d}(f_n(\alpha), f(\alpha)) = \lim_{m \to \infty} \overline{d}(f_n(\alpha), f_m(\alpha)) \leq \epsilon / 2, \forall n > N, \forall \alpha \in J.
	\end{equation*}
	(follows from the continuity of metric proposition \ref{prop:Continuity of Metric} and theorem \ref{thm:Continuity and Convergence})

	So we have $\overline{\rho}(f_n, f) \leq \epsilon / 2 < \epsilon$ for all $n > N$.
\end{proof}

Now to be more specific, we may consider the function space $Y^X$ where $X$ is a topological space.

\begin{theorem}{Continuous and Bounded Functions}{Continuous and Bounded Functions}
	Let $X$ be a topological space and $(Y,d)$ be a metric space. Let $\mathscr{C}(X,Y)$ be the set of all continuous functions from $X$ to $Y$, and $\mathscr{B}(X,Y)$ be the set of all bounded functions from $X$ to $Y$.

	Then both $\mathscr{C}(X,Y)$ and $\mathscr{B}(X,Y)$ are closed subsets of $(Y^X, \overline{\rho})$. In particular, if $(Y,d)$ is complete, then both $\mathscr{C}(X,Y)$ and $\mathscr{B}(X,Y)$ are complete metric space under the uniform metric.
\end{theorem}
\begin{proof}
	For $\mathscr{C}(X,Y)$ it is just the uniform limit throrem \ref{thm:Uniform Limit Theorem}, for $\mathscr{B}(X,Y)$ it is also obvious.
\end{proof}

\begin{definition}{Sup Metric}{Sup Metric}
	Let $X$ be a topological space and $(Y,d)$ be a metric space. The sup metric on $\mathscr{B}(X,Y)$ is defined by
	\begin{equation}
		\rho(f,g) = \sup \{d(f(x), g(x)) : x \in X \}.
	\end{equation}
\end{definition}
In fact, in $\mathscr{B}(X,Y)$, $\rho$ and $\overline{\rho}$ are equivalent metrics. And $\overline{\rho}$ is just the standard bounded metric of $\rho$.

If $X$ is compact, then every continuous function from $X$ to $Y$ is bounded, so $\mathscr{C}(X,Y) \subset \mathscr{B}(X,Y)$. And if $Y$ is complete, then $\mathscr{C}(X,Y)$ is complete under the sup metric.

We now show that every metric space can be isometrically embedded in a complete metric space. Meaning that it is just some part of a complete metric space just like $\mathbb{Q} \subseteq \mathbb{R}$.

\begin{theorem}{Completion of Metric Spaces}{Completion of Metric Spaces}
	Let $(X,d)$ be a metric space. There exists a complete metric space $(X',d')$ and an isometric embedding $f : X \to X'$.

	The subspace $\overline{f(X)}$ in $Y$ is a complete metric space containing $f(X)$ as a dense subset and is called the completion of $X$.
\end{theorem}
\begin{proof}
	Let $\mathscr{B}(X,\mathbb{R})$ be the set of all bounded functions from $X$ to $\mathbb{R}$, let $x_0\in X$, and given $a\in X$. Define $\phi_a:X \rightarrow \mathbb{R}$ by
	\begin{equation*}
		\phi_a(x) = d(a,x) - d(x,x_0).
	\end{equation*}
	The triangle inequality shows that $\left|\phi_a(x)\right| \leq d(a.x_0)$ which is bounded.

	Define $\Phi: X \rightarrow \mathscr{B}(X,\mathbb{R}), \Phi(a) = \phi_a$. We prove that $\Phi$ is an isometric embedding:
	\begin{equation*}
		\rho(\phi_a,\phi_b) = \sup \left\{|\phi_a(x) - \phi_b(x)| : x \in X \right\} = \sup \left\{|d(a,x) - d(b,x)| : x \in X \right\} = d(a,b).
	\end{equation*}
\end{proof}

\begin{theorem}{Uniqueness of Completion}{Uniqueness of Completion}
	Let $X$ be a metric space. If $h:X \rightarrow Y$ and $h':X \rightarrow Y'$ are isometric embeddings of $X$ into complete metric spaces $Y$ and $Y'$ such that both $h(X)$ and $h'(X)$ are dense in $Y$ and $Y'$, respectively, then there exists an isometry $g:Y \rightarrow Y'$ such that $g \circ h = h'$.
\end{theorem}

\section{A Space Filling Curve}
An application for the completeness of $\mathscr{C}(X,Y)$ in the uniform metric when $Y$ is complete.
\begin{theorem}{Peano's Curve}{Peano's Curve}
	Let $I = [0,1]$, there exists a continuous surjective map $f:I \rightarrow I^2$.
\end{theorem}

\begin{figure}[ht]
    \centering
    \incfig{transformation-of-path}
    \caption{Transformation of Path}
    \label{fig:transformation-of-path}
\end{figure}

\begin{figure}[ht]
    \centering
    \incfig{peano-curve-construction}
    \caption{Peano Curve Construction}
    \label{fig:peano-curve-construction}
\end{figure}

\begin{proof}
	\textbf{Step 1}: We construct a sequence of continuous functions and taking the limit. First we define an operation on paths:

	Let $J^2$ be any square, and $g: [a,b] \rightarrow J^2$ be a path that looks like the following figure. Take a transformation $T$ that maps $g$ to $g':[a,b] \rightarrow J^2$ as follows:

	\textbf{Step 2}: Define a sequence of paths $f_n: I \rightarrow I^2$ as follows:
	\begin{itemize}
		\item Let $f_0:I \rightarrow I^2$ be the first triangle picture in step 1.
		\item Apply the transformation $T$ to $f_0$ to get $f_1$.
		\item Apply the transformation $T$ to each small square in $f_n$ to get $f_{n+1}$.
	\end{itemize}

	\textbf{Step 3}: Take
	\begin{equation*}
		d(x,y) = \max \{|x_1 - y_1|, |x_2 - y_2|\}, \quad \rho(f,g) = \sup \{d(f(t), g(t)) : t \in I \}.
	\end{equation*}
	Because $I^2$ is closed in $\mathbb{R}^2$, it is complete, so $\mathscr{C}(I,I^2)$ is complete under $\rho$. We prove that $f_n$ is a Cauchy sequence in $\mathscr{C}(I,I^2)$: When passing the operator, the distance of the to functions is at most $A^2$, where $A$ is the edge length of the square. So passing form $f_n$ to $f_{n+1}$, we have $\rho(f_n,f_{n+1}) \leq  1 / 2^n$. Then
	\begin{equation*}
		\rho(f_n,f_{n+m}) \leq \sum_{i=n}^{n+m-1} \rho(f_i,f_{i+1}) \leq \sum_{i=n}^{n+m-1} \frac{1}{2^i} < \frac{1}{2^{n-1}}.
	\end{equation*}
	So $f_n$ is a Cauchy sequence, and it converges to some $f \in \mathscr{C}(I,I^2)$.

	\textbf{Step 4}: We prove that $f$ is surjective: For any $x\in I^2$, the distance between $x$ and $f_n$ $< 1 / 2^n$, so for every $\epsilon > 0$, there exists $N$ that
	\begin{equation*}
		\rho(f_N, f) < \frac{\epsilon}{2}, \qquad \frac{1}{2^N} < \frac{\epsilon}{2}.
	\end{equation*}
	Then as $\exists t_0\in I, d(x,f_N(t_0))<1 / 2^N$, then
	\begin{equation*}
		d(x,f(t_0)) \leq d(x,f_N(t_0)) + d(f_N(t_0),f(t_0)) < \frac{\epsilon}{2} + \frac{\epsilon}{2} = \epsilon.
	\end{equation*}
	So we have every neighborhood of $x$ intersects $f(I)$, so $x \in \overline{f(I)} = f(I)$. We also have $I$ is compact and $f$ continuous, so $f(I)$ is compact, thus closed. So $f(I) = I^2$.
\end{proof}



\section{Compactness in Metric Spaces}
We've already shown in theorem \ref{thm:Equivalence of Compactness in Metric Spaces} that in metric spaces, compactness, limit point compactness and sequential compactness are equivalent.

Following the equivalence, every compact metric space is sequentially compact, and thus complete. But the converse is not true, for example, $\mathbb{R}$ is complete but not compact.

\begin{definition}{Totally Bounded}{Totally Bounded}
	A metric space $(X,d)$ is totally bounded if for every $\epsilon > 0$, there exists a finite cover of $X$ by open balls of radius $\epsilon$.
\end{definition}

\begin{proposition}{Properties of Totally Boundedness}{Properties of Totally Boundedness}
	\begin{itemize}
		\item Totally boundedness implies boundedness, but not conversely. For example, $\mathbb{R}$ is bounded in $\overline{d}(x,y) = \min \left\{ \left|x-y\right|,1 \right\}$ but not totally bounded.
		\item Under $d(x,y) = \left|x-y\right|$. $\mathbb{R}$ is complete but not totally bounded. The subspace $(-1,1)$ is totally bounded but not complete. The subspace $[0,1]$ is both complete and totally bounded.
	\end{itemize}
\end{proposition}

\begin{theorem}{Compactness of Metric Spaces}{Compactness of Metric Spaces}
	A metric space is compact if and only if it is complete and totally bounded.
\end{theorem}
\begin{proof}
	\begin{itemize}
		\item The $\Rightarrow$ side is from theorem \ref{thm:Compact Metric Spaces are Complete} and the definition of compactness.
		\item Let $X$ be complete and totally bounded, and prove that $X$ is sequentially compact. Let $(x_n)$ be a sequence in $X$. Since $X$ is totally bounded, there exists a finite cover of $X$ by open balls of radius $1$. So some ball contains infinitely many terms of the sequence. Call this ball $B_1$, and let $(x_{1,n})$ be the subsequence of $(x_n)$ consisting of all terms that lie in $B_1$.

		      Next, since $X$ is totally bounded, there exists a finite cover of $X$ by open balls of radius $1/2$. So some ball contains infinitely many terms of the sequence $(x_{1,n})$. Call this ball $B_2$, and let $(x_{2,n})$ be the subsequence of $(x_{1,n})$ consisting of all terms that lie in $B_2$.

		      Continuing in this way, we obtain a sequence of nested balls $B_k$ of radius $1/k$, each containing infinitely many terms of the previous subsequence. Let $(x_{k,n})$ be the subsequence of $(x_{k-1,n})$ consisting of all terms that lie in $B_k$. Now define a new sequence $(y_k)$ by letting $y_k = x_{k,k}$. Then $(y_k)$ is a subsequence of $(x_n)$, and for $m,n > N$, both $y_m$ and $y_n$ lie in the ball $B_N$ of radius $1/N$. So
		      \begin{equation*}
			      d(y_m,y_n) < 2/N.
		      \end{equation*}
		      This shows that $(y_k)$ is a Cauchy sequence in $X$, and since $X$ is complete, $(y_k)$ converges to some point in $X$. Thus every sequence in $X$ has a convergent subsequence, and so $X$ is sequentially compact.
	\end{itemize}
\end{proof}

We can apply this to find the compact subsets of $\mathscr{C}(X,\mathbb{R}^n)$ in the uniform topology. We know that a subset of $\mathbb{R}^n$ is compact iff it is closed and bounded. But this is not true in $\mathscr{C}(X,\mathbb{R}^n)$. We need additional conditions called equicontinuity.

\begin{definition}{Equicontinuity}{Equicontinuity}
	Let $(Y,d)$ be a metric space, and $\mathcal{F}$ be a subset of $\mathscr{C}(X,Y)$. Then $\mathcal{F}$ is equicontinuous at $x_0 \in X$ if for every $\epsilon > 0$, there exists a neighborhood $U$ of $x_0$ such that
	\begin{equation*}
		\forall f\in \mathcal{F}, \forall x \in U, d(f(x), f(x_0)) < \epsilon.
	\end{equation*}
	$\mathcal{F}$ is equicontinuous on $X$ if it is equicontinuous at each point of $X$.
\end{definition}

\begin{remark}
	This means that we can choose a common $\delta$ for all functions in $\mathcal{F}$ to make them continuous at $x_0$. Note that this concept relies on the chosen metric $d$.
\end{remark}

\begin{lemma}{Totally Bounded Implies Equicontinuity}{Totally Bounded Implies Equicontinuity}
	Let $X$ be a topological space, and $(Y,d)$ be a metric space. If $\mathcal{F} \subseteq \mathscr{C}(X,Y)$ is totally bounded in the uniform metric, then $\mathcal{F}$ is equicontinuous under $d$.
\end{lemma}
\begin{proof}
	Assume $\mathcal{F}$ is totally bounded. For any $0 < \epsilon < 1, x_0\in X$, set $\delta = \epsilon / 3$, cover $\mathcal{F}$ by finite open balls
	\begin{equation*}
		B(f_i, \delta), i = 1,2,\ldots,n.
	\end{equation*}
	where $f_i \in \mathcal{F}$. Since $f_i$ is continuous at $x_0$, there exists a neighborhood $U_i$ of $x_0$ such that
	\begin{equation*}
		d(f_i(x), f_i(x_0)) < \delta, \forall x \in U_i.
	\end{equation*}
	Take $U = \bigcap_{i=1}^n U_i$, then for any $f \in \mathcal{F}$, there exists $i$ such that $\rho(f,f_i) < \delta$, so for any $x \in U$,
	\begin{equation*}
		d(f(x), f(x_0)) \leq d(f(x), f_i(x)) + d(f_i(x), f_i(x_0)) + d(f_i(x_0), f(x_0)) < \delta + \delta + \delta = \epsilon.
	\end{equation*}
\end{proof}

Now we can prove the classical version of Ascoli's theorem. First we give a partial converse to the above lemma.

\begin{lemma}{Equicontinuity Implies Totally Bounded}{Equicontinuity Implies Totally Bounded}
	Let $X$ be a compact topological space, and $(Y,d)$ be a compact metric space. If $\mathcal{F} \subseteq \mathscr{C}(X,Y)$ is equicontinuous under $d$, then $\mathcal{F}$ is totally bounded in the uniform and sup metrics.
\end{lemma}
\begin{proof}
	Since $X$ is compact, then $\mathscr{C}(X,Y) \subseteq \mathscr{B}(X,Y)$, and the sup metric is well defined. Here total boundedness under the uniform and sup metrics are equivalent.

	For any $\epsilon > 0$, set $\delta = \epsilon / 3$, for all $a\in X$, there exists a neighborhood $U_a$ of $a$ such that
	\begin{equation*}
		\forall f\in \mathcal{F}, \forall x \in U_a, d(f(x), f(a)) < \delta.
	\end{equation*}
	Cover $X$ by such $U_a$, and there is a finite subcover, denoted $U_{a_i} = U_i$. Cover $Y$ by finite open balls $V_1, \ldots ,V_m$ with diameter $< \delta$.

	Let $J$ be the set of all functions $\alpha: \{1,2,\ldots,n\} \rightarrow \{1,2,\ldots,m\}$. For each $\alpha \in J$, define
	\begin{equation*}
		J' = \left\{ \alpha\in J: \exists f\in \mathcal{F}, \forall i=1, \ldots ,k, f(a_i)\in V_{\alpha(i)} \right\}
	\end{equation*}
	This selects only those $\alpha$ in $J$ that correspond to some function in $\mathcal{F}$ that follows the pattern specified by $\alpha$. The set isn't empty because for any $f\in \mathcal{F}$, we can define $\alpha$ by letting $\alpha(i)$ be the index of a ball containing $f(a_i)$.

	For each $\alpha \in J'$, choose $f_{\alpha} \in \mathcal{F}$ corresponding to $\alpha$. We prove that the open balls $B(f_{\alpha}, \epsilon)$, $\alpha \in J'$ cover $\mathcal{F}$:

	Let $f \in \mathcal{F}$, there exists $\alpha \in J'$ such that $f(a_i) \in V_{\alpha(i)}$ for each $i$. Then for any $x \in X$, there exists $i$ such that $x \in U_i$, so
	\begin{equation*}
		d(f(x), f_{\alpha}(x)) \leq d(f(x), f(a_i)) + d(f(a_i), f_{\alpha}(a_i)) + d(f_{\alpha}(a_i), f_{\alpha}(x)) < \delta + \delta + \delta = \epsilon.
	\end{equation*}
	So we have $\rho(f, f_{\alpha}) < \epsilon$, thus $f \in B(f_{\alpha}, \epsilon)$.
\end{proof}


\begin{definition}{Pointwise Bounded}{Pointwise Bounded}
	Let $X$ be a topological space, and $(Y,d)$ be a metric space. A subset $\mathcal{F}$ of $\mathscr{C}(X,Y)$ is pointwise bounded if for each $x \in X$, the set $\mathcal{F}_a = \{f(x) : f \in \mathcal{F}\}$ is a bounded subset of $Y$.
\end{definition}

\begin{theorem}{Ascoli's Theorem (Classical)}{Ascolis Theorem Classical}
	Let $X$ be a compact space, and $(\mathbb{R}^n,d)$ be a metric space either in the square metric or the Euclidean metric. Give $\mathscr{C}(X,\mathbb{R}^n)$ the uniform topology. A subspace $\mathcal{F}$ of $\mathscr{C}(X,\mathbb{R}^n)$ has compact closure if and only if $\mathcal{F}$ is equicontinuous and pointwise bounded.
\end{theorem}
\begin{proof}
	SORRY
\end{proof}

\begin{corollary}{Compact Subspace of Continuous Functions}{Compact Subspace of Continuous Functions}
	Let $X$ be a compact space, and $(\mathbb{R}^n,d)$ be a metric space either in the square metric or the Euclidean metric. Give $\mathscr{C}(X,\mathbb{R}^n)$ the uniform topology. A subspace $\mathcal{F}$ of $\mathscr{C}(X,\mathbb{R}^n)$ is compact if and only if $\mathcal{F}$ is closed, equicontinuous and bounded.
\end{corollary}
\begin{proof}
	In a metric space, a compact space is closed and bounded, so it has compact closure, and thus is equicontinuous and pointwise bounded. Conversely, if $\mathcal{F}$ is closed, equicontinuous and bounded, then it is pointwise bounded, so it has compact closure, thus is compact.
\end{proof}


\section{Pointwise and Compact Convergence}
There are other useful topologies on $\mathscr{C}(X,Y)$, such as the topology of pointwise convergence, the topology of compact convergence and the compact-open topology.

\begin{definition}{Topology of Pointwise Convergence}{Topology of Pointwise Convergence}
	For a point $x\in X$ and an open set $U \subseteq Y$, let
	\begin{equation}
		S(x,U) = \{f \in Y^X : f(x) \in U \}.
	\end{equation}
	The topology of pointwise convergence on $Y^X$ is the topology generated by the subbasis consisting of all sets of the form $S(x,U)$.

	It is just the product topology on $Y^X$.
\end{definition}

\begin{theorem}{Pointwise Convergence}{Pointwise Convergence}
	A sequence $(f_n)$ in $Y^X$ converges to $f \in Y^X$ in the topology of pointwise convergence if and only if for each $x \in X$, the sequence $(f_n(x))$ converges to $f(x)$ in $Y$.
\end{theorem}

\begin{example}{Pointwise Convergence}{Pointwise Convergence}
	Consider $[0,1] \rightarrow  \mathbb{R}$, then $f_n(x) = x^n$ converges to
	\begin{equation*}
		f(x) = \begin{cases}
			0, & 0 \leq x < 1 \\
			1, & x = 1
		\end{cases}
	\end{equation*}
	in the topology of pointwise convergence, but $f$ is not continuous.
\end{example}

We may wonder that if there is a ``middle'' topology that lies between the uniform topology and the topology of pointwise convergence that ensure the limit of a convergent sequence of continuous functions is continuous. The answer is yes, the topology of compact convergence.

\begin{definition}{Topology of Compact Convergence}{Topology of Compact Convergence}
	Let $(Y,d)$ be a metric space, and $X$ be a topological space. Given $f\in Y^X$, a compact subset $C$ of $X$ and $\epsilon > 0$, let
	\begin{equation*}
		B_C(f,\epsilon) = \{g \in Y^X : \sup \{d(f(x), g(x)) : x \in C\} < \epsilon \}.
	\end{equation*}
	Then all such $B_C(f,\epsilon)$ form a basis for a topology on $Y^X$, called the topology of compact convergence.
\end{definition}
\begin{remark}
	It is fairly easy to notice that this topology lies between the uniform topology and the pointwise topology. The former requires the supremum to be small on the whole space, while the latter only requires it to be small on some finite points. Now the supremum is required to be small on every compact subset.
\end{remark}

\begin{theorem}{Compact Convergence}{Compact Convergence}
	Let $(Y,d)$ be a metric space, and $X$ be a topological space. A sequence $(f_n)$ in $Y^X$ converges to $f \in Y^X$ in the topology of compact convergence if and only if for every compact subset $C$ of $X$, the sequence $(f_n|_C)$ converges to $f|_C$ in the uniform topology on $\mathscr{C}(C,Y)$.
\end{theorem}
\begin{proof}
	Obvious.
\end{proof}

\begin{definition}{Compactly Generated}{Compactly Generated}
	A topological space $X$ is compactly generated if for all $A \subseteq X$:
	\begin{equation*}
		\forall \text{ compact } C \subseteq X, A\cap C \in \mathcal{T}_C \rightarrow A \in \mathcal{T}.
	\end{equation*}
	Or equivalently, $A$ is closed in $X$ if for every compact subset $C$ of $X$, $A \cap C$ is closed in $C$.
\end{definition}
The equivalence is easy to prove. And we shall see that it is a fairly mild restriction on the space.

\begin{lemma}{Compact Generated Spaces}{Compact Generated Spaces}
	Locally compact spaces or first countable spaces are compactly generated.
\end{lemma}
\begin{proof}
	\textbf{Locally Compact Spaces}, let $A \subseteq X$ such that for every compact subset $C$ of $X$, $A \cap C$ is open in $C$, then given $x\in A$, take a neighborhood $U$ that lies in a compact $C$. Then $A \cap C$ is open in $C$, so $A\cap U$ is open in $U$, hence open in $X$. So $A$ is open in $X$.

	\textbf{First Countable Spaces}, If $B\cap C$ is closed in $C$ for all compact $C$, let $x\in \overline{B}$, take a sequence $(x_n)$ in $B$ that converges to $x$ (from first countable). Then the subspace
	\begin{equation*}
		C = \left\{ x \right\} \cup \left\{ x_n: n\in \mathbb{Z}_+ \right\}
	\end{equation*}
	is compact, so $B\cap C$ is closed in $C$, thus $x\in B\cap C \subseteq B$. So $B$ is closed in $X$.
\end{proof}

\begin{theorem}{Continuity in Compactly Generated Spaces}{Continuity in Compactly Generated Spaces}
	Let $X$ be a compactly generated space. Then a function $f:X \rightarrow Y$ is continuous if and only if for every compact subset $C$ of $X$, the restriction $f|_C$ is continuous.
\end{theorem}
\begin{proof}
	For any open $V \subseteq Y$, then for any $C \subseteq X$:
	\begin{equation*}
		f^{-1}(V) \cap C = (f|_C)^{-1}(V).
	\end{equation*}
	If $C$ is compact, then $f|_C$ is continuous, so $(f|_C)^{-1}(V)$ is open in $C$. Thus $f^{-1}(V)$ is open in $X$, so $f$ is continuous.
\end{proof}

\begin{remark}
	This clearly shows where our intuition of compact generation comes from.
\end{remark}

\begin{theorem}{Closedness of Continuity Space in Compact Convergence}{Closedness of Continuity Space in Compact Convergence}
	Let $X$ be a compactly generated space, and $(Y,d)$ be a metric space. Then $\mathscr{C}(X,Y)$ is closed in $Y^X$ under the topology of compact convergence.
\end{theorem}
\begin{proof}
	Let $f \in \overline{\mathscr{C}(X,Y)}$, we prove that $f$ is continuous: For any compact subset $C$ of $X$, choose $f_n\in \mathscr{C}(X,Y)$ that lies in $B_C(f,1 / n)$. Then $f_n|_C$ converges to $f|_C$ in the uniform topology on $\mathscr{C}(C,Y)$, so $f|_C$ is continuous. From theorem \ref{thm:Continuity in Compactly Generated Spaces}, $f$ is continuous.
\end{proof}

\begin{corollary}{Inner Compact Convergence Theorem}{Inner Compact Convergence Theorem}
	Let $X$ be a compactly generated space, and $(Y,d)$ be a complete metric space. If a continuous sequence $(f_n)$ in $\mathscr{C}(X,Y)$ converges to $f$ in the topology of compact convergence, then $f \in \mathscr{C}(X,Y)$.
\end{corollary}

The relation of the three topologies is summarized as follows:
\begin{theorem}{Relation of Function Space Topologies}{Relation of Function Space Topologies}
	Let $X$ be a topological space, and $(Y,d)$ be a metric space. Then we have the relation of topologies on $Y^X$:
	\begin{equation}
		\text{ pointwise convergence } \subseteq \text{ compact convergence } \subseteq \text{ uniform convergence.}
	\end{equation}
	where the first inclusion is an equality if $X$ is discrete, and the second inclusion is an equality if $X$ is compact.
\end{theorem}

\begin{remark}
	All three topologies make use of the metric $d$ on $Y$, but we found that the poitwise convergence topology does not depend on the choice of $d$. In fact, it is just the product topology on $Y^X$. A natural question is whether there is a generalization of the compact convergence topology to general topological spaces.
\end{remark}

\begin{definition}{Compact-Open Topology}{Compact-Open Topology}
	Let $X$ and $Y$ be topological spaces. Given a compact subset $C$ of $X$ and an open set $U \subseteq Y$, let
	\begin{equation*}
		S(C,U) = \{f \in Y^X : f(C) \subseteq U \}.
	\end{equation*}
	The compact-open topology on $Y^X$ is the topology generated by the subbasis consisting of all sets of the form $S(C,U)$.
\end{definition}
And it coincides with the compact convergence topology when $Y$ is a metric space, obviously.

\begin{theorem}{Conincidence of Compact-Open Topology}{Coincidence of Compact Convergence and Compact-Open Topology}
	Let $X$ be a topological space, and $(Y,d)$ be a metric space. Then the topology of compact convergence on $\mathscr{C}(X,Y)$ coincides with the compact-open topology.	
\end{theorem}
\begin{proof}
SORRY
\end{proof}

\begin{corollary}{Independence of Metric of Compact Convergence}{Independence of Metric of Compact Convergence}
	Let $X$ be a topological space, and $Y$ be a topological space that is metrizable. Then the topology of compact convergence on $\mathscr{C}(X,Y)$ does not depend on the choice of metric on $Y$.
\end{corollary}

\begin{remark}
	The compact convergence shows that it is both continuous on $x$ and $f$, in some sense, this is called joint continuity. And it is fairly useful in many areas, such as functional analysis and algebraic topology.
\end{remark}

\begin{theorem}{Evaluation Map}{Evaluation Map}
	Let $X$ be locally compact Hausdorff, and $\mathscr{C}(X,Y)$ be the compact-open topology. Then the map
	\begin{equation}
		e: X \times \mathscr{C}(X,Y) \rightarrow Y, \quad e(x,f) = f(x)
	\end{equation}
	is continuous. We call $e$ the evaluation map.
\end{theorem}
\begin{proof}
	Take $(x,f)\in X \times \mathscr{C}(X,Y)$, and an open neighborhood $V \subseteq Y$ of $f(x)$. By the continuity of $f$, and the local compact Hausdorff of $X$, choose an open neighborhood $U$ of $x$ that $\overline{U}$ is compact and $f(U) \subseteq V$. Consider $U \times S(\overline{U},V)$, if $(y,g) \in U \times S(\overline{U},V)$, then $g(\overline{U}) \subseteq V$, so $g(y) \in V$. Thus $e(U \times S(\overline{U},V)) \subseteq V$, so $e$ is continuous.
\end{proof}

\begin{figure}[ht]
    \centering
    \incfig{the-evaluation-map}
    \caption{The Evaluation Map}
    \label{fig:the-evaluation-map}
\end{figure}

\end{document}
