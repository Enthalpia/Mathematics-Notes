\documentclass[../main.tex]{subfiles}

\begin{document}
\chapter{Metrization Theorems and Paracompactness}

The Urysohn Metrization Theorem states that a topological space is metrizable if it is regular and has a countable basis. And we hope to strengthen the condition to give an equivalent condition.

We know that the regularity is necessary, but the second countability is not. The eventual conclusion involves a new notion called locally finiteness.

Another way to say a basis $\mathcal{B}$ is countable is that $\mathcal{B}$ can be expressed in
\begin{equation*}
	\mathcal{B} = \bigcup_{n\in \mathbb{Z}_+} \mathcal{B}_n, \qquad \text{where each } \mathcal{B}_n \text{ is finite.}
\end{equation*}
It is rather aukward, but it gives us a hint to weaken the second countability condition: The Nagata-Smirnov condition, by means of local finiteness.

\section{Local Finiteness}
\begin{definition}{Local Finiteness}{Local Finiteness}
	Let $X$ be a topological space. A collection $\mathcal{A}$ of subsets of $X$ is said to be \textbf{locally finite} if for every $x \in X$, there exists an open neighborhood $U$ of $x$ such that $U$ intersects only finitely many elements of $\mathcal{A}$.
\end{definition}

\begin{example}{Local Finiteness}{Local Finiteness}
	\begin{itemize}
		\item The collection $\mathcal{A} = \left\{ (n,n+2): n\in \mathbb{Z} \right\}$ is locally finite in $\mathbb{R}$.
		\item The collection $\mathcal{B} = \left\{ (1, 1 / n): n \in \mathbb{Z}^+ \right\}$ is locally finite in $(0,1)$ but not in $\mathbb{R}$.
	\end{itemize}
\end{example}

\begin{proposition}{Properties of Local Finiteness}{Properties of Local Finiteness}
	Let $X$ be a topological space and let $\mathcal{A}$ be a locally finite collection of subsets of $X$. Then
	\begin{itemize}
		\item Every subcollection of $\mathcal{A}$ is locally finite.
		\item The collection $\mathcal{B} = \left\{ \overline{A} : A \in \mathcal{A} \right\}$ is locally finite.
		\item $\overline{\bigcup_{A \in \mathcal{A}} A} = \bigcup_{A \in \mathcal{A}} \overline{A}$.
	\end{itemize}
\end{proposition}
\begin{proof}
	First one is trivial. For the second one, note that any open set intersecting $\overline{A}$ needs to intersect $A$. For the last one, we let
	\begin{equation*}
		Y = \bigcup_{A \in \mathcal{A}} A.
	\end{equation*}
	In general, we have $\overline{Y} \supseteq \bigcup_{A \in \mathcal{A}} \overline{A}$. For the other direction, let $x \in \overline{Y}$ and let $U$ be an open neighborhood of $x$ intersecting $Y$. Since $\mathcal{A}$ is locally finite, there exists an open neighborhood $V$ of $x$ intersecting only finitely many elements of $\mathcal{A}$, say $A_1, A_2, \ldots, A_n$. Then $x$ is in one of $\overline{A_i}$: of not, then $U - \bigcup_{i=1}^n \overline{A_i}$ is an open neighborhood of $x$ not intersecting $Y$, a contradiction.
\end{proof}

\begin{remark}
	An analogous concept is a \textbf{local finite index family} $\left\{ A_{\alpha} \right\}_{\alpha \in J}$, where $J$ is an index set. (This allows repeat sets). It is obvious that a local finite index family is a local finite collection if it is local finite as a collection of sets and there are at most finite multiple repeat non-empty set in the family.
\end{remark}

\begin{definition}{Countably Local Finite}{Countably Local Finite}
	A collection $\mathcal{A}$ of subsets of a topological space $X$ is said to be \textbf{countably locally finite} if $\mathcal{A}$ can be expressed in
	\begin{equation*}
		\mathcal{A} = \bigcup_{n\in \mathbb{Z}_+} \mathcal{A}_n, \qquad \text{where each } \mathcal{A}_n \text{ is locally finite.}
	\end{equation*}
	This is also called $\sigma$-locally finite.
\end{definition}
Both a countable collection and a locally finite collection are countably locally finite.

\begin{definition}{Refinement}{Refinement}
	Let $\mathcal{A}$ and $\mathcal{B}$ be collections of subsets of a set $X$. We say that $\mathcal{B}$ is a \textbf{refinement} of $\mathcal{A}$ if for every $B \in \mathcal{B}$, there exists an $A \in \mathcal{A}$ such that $B \subseteq A$. (This means that $\mathcal{B}$ is finer than $\mathcal{A}$.)

	If the elements in $\mathcal{B}$ are open sets in $X$, then we say that $\mathcal{B}$ is an \textbf{open refinement} of $\mathcal{A}$. If they are closed sets, then we say that $\mathcal{B}$ is a \textbf{closed refinement} of $\mathcal{A}$.
\end{definition}

\begin{lemma}{Existence of Locally Finite Open Refinement}{Existence of Locally Finite Open Refinement}
	Let $X$ be a metrizable space, and $\mathcal{A}$ be an open cover of $X$. Then there is an open covering $\mathcal{E}$ of $X$ which is a locally finite open refinement of $\mathcal{A}$.
\end{lemma}
\begin{proof}
	SORRY
\end{proof}

\section{The Nagata-Smirnov Metrization Theorem}
Now we give an equivalent condition for metrizability.

\begin{definition}{$F_{\sigma}$ and $G_{\delta}$ Sets}{Fsigma and Gdelta Sets}
	Let $X$ be a topological space. A subset $A$ of $X$ is called an \textbf{$F_{\sigma}$-set} if $A$ can be expressed as a countable union of closed sets in $X$. A subset $B$ of $X$ is called a \textbf{$G_{\delta}$-set} if $B$ can be expressed as a countable intersection of open sets in $X$.
\end{definition}

\begin{lemma}{Countably Locally Finite Basis in a Regular Space}{Countably Locally Finite Basis in a Regular Space}
	Let $X$ be a regular space with a countably locally finite basis $\mathcal{B}$, then $X$ is normal and every closed set in $X$ is a $G_{\delta}$-set.
\end{lemma}
\begin{proof}
	SORRY
\end{proof}

\begin{lemma}{Closed $G_{\delta}$ Sets}{Closed Gdelta Sets}
	Let $X$ be normal, and let $A$ be a closed $G_{\delta}$-set in $X$. Then there is a continuous function $f: X \to [0,1]$ such that $f(x) = 0$ for all $x \in A$ and $f(x) > 0$ for all $x \notin A$.
\end{lemma}
\begin{proof}
	Write $A = \bigcap_{n=1}^{\infty} U_n$, where each $U_n$ is open in $X$. Since $X$ is normal, for each $n$ there is a continuous function $f_n: X \to [0,1]$ such that $f_n(x) = 0$ for all $x \in A$ and $f_n(x) = 1$ for all $x \in X - U_n$ (Urysohn's Lemma). Now define
	\begin{equation*}
		f(x) = \sum_{n=1}^{\infty} \frac{f_n(x)}{2^n}.
	\end{equation*}
	Then $f$ is continuous due to uniform convergence, and $f(x) = 0$ for all $x \in A$ and $f(x) > 0$ for all $x \notin A$, because there is an $n$ such that $x \notin U_n$, so $f_n(x) = 1$.
\end{proof}

\begin{theorem}{The Nagata-Smirnov Metrization Theorem}{The Nagata-Smirnov Metrization Theorem}
	For a topological space $X$, then $X$ is metrizable if and only if $X$ is regular and has a $\sigma$-locally finite basis.
\end{theorem}
\begin{proof}
	SORRY
\end{proof}

\section{Paracompactness}
This is another generalization of compactness: Recall that a space is compact if every open cover has a finite subcover. We can reformulate it as:
\begin{quote}
	A space is compact if every open cover has a finite open refinement.
\end{quote}

So we can generalized it:
\begin{definition}{Paracompactness}{Paracompactness}
	A topological space $X$ is said to be \textbf{paracompact} if every open cover of $X$ has a locally finite open refinement.
\end{definition}

\begin{example}{Paracompact Space}{Paracompact Space}
	$\mathbb{R}^n$ is paracompact: Let $X = \mathbb{R}^n$, and let $\mathcal{A}$ be an open cover of $X$. Let $B_m$ be the radius $m\in \mathbb{N}$ open ball centered at the origin ($B_0 = \emptyset$). Then for $m$, choose finite subcover $\mathcal{A}_m$ of $\mathcal{A}$ covering $\overline{B_m}$. Denote
	\begin{equation*}
		\mathcal{C}_m = \left\{ A \cap (X - \overline{B_{m-1}}) : A \in \mathcal{A}_m \right\}.
	\end{equation*}
	Then easily $\mathcal{C} = \bigcup_{m=1}^{\infty} \mathcal{C}_m$ is a locally finite open refinement of $\mathcal{A}$.
\end{example}

\begin{proposition}{Properties of Paracompact Spaces}{Properties of Paracompact Spaces}
	\begin{itemize}
		\item Every compact space is paracompact.
		\item The subspace of a paracompact space is NOT NECESSARILY paracompact.
		\item A closed subspace of a paracompact space is paracompact.
	\end{itemize}
\end{proposition}

\begin{theorem}{Paracompact Hausdorff Spaces are Normal}{Paracompact Hausdorff Spaces are Normal}
	Every paracompact Hausdorff space $X$ is normal.
\end{theorem}
\begin{proof}
	This is rather similar to the proof of theorem \ref{thm:Compact Hausdorff Implies Normal}, we first prove that $X$ is regular. Let $a\in X$ and $B$ be a closed set not containing $a$. For each $b \in B$, choose $U_b\in \mathcal{T}, a\notin \overline{U_b}$. Cover $X$ with all $U_b$ and $X-B$, and choose a locally finite open refinement $\mathcal{C}$. Let $\mathcal{D} \subseteq \mathcal{C}$ be all sets that intersects $B$, then $\mathcal{D}$ covers $B$.

	Moreover, if $D\in \mathcal{D}$, then $D$ lies in some $U_b$, so $a\notin \overline{D}$. Let
	\begin{equation*}
		V = \bigcup_{D\in \mathcal{D}} D, \Rightarrow \overline{V} = \bigcup_{D\in \mathcal{D}} \overline{D} \Rightarrow a\notin \overline{V}.
	\end{equation*}
	then regularity is proved. For normality, repeat the process would do.
\end{proof}

\begin{theorem}{Closed Subspaces of Paracompact Spaces}{Closed Subspaces of Paracompact Spaces}
	Every closed subspace of a paracompact space is paracompact.
\end{theorem}
\begin{proof}
	Let $X$ be a paracompact space, and let $Y$ be a closed subspace of $X$. Let $\mathcal{A}$ be an open cover of $Y$ with open sets in $X$. Cover $X$ with $\mathcal{A}$ and $X-Y$, and choose a locally finite open refinement $\mathcal{C}$, then intersecting elements of $\mathcal{C}$ with $Y$ would do.
\end{proof}

\begin{remark}
	\begin{itemize}
		\item A paracompact subspace of a Hausdorff space is NOT NECESSARILY closed. Well, $(0,1)$ is paracompact, homeomorphic to $\mathbb{R}$, but not closed in $\mathbb{R}$.
		\item A subspace of a paracompact space is NOT NECESSARILY paracompact.
		\item The product of two paracompact spaces is NOT NECESSARILY paracompact. Like $\mathbb{R}_l^2$.
		\item The space $\mathbb{R}^{\mathbb{Z}_+}$ is paracompact in both product and uniform topology. It is not known that whether the box topology is paracompact.
		\item $\mathbb{R}^J$ is not paracompact for uncountable $J$, for $\mathbb{R}^J$ is Hausdorff but not normal.
	\end{itemize}
\end{remark}

We shall see that every metrizable space is paracompact.
\begin{lemma}{Michael's Lemma}{Michael's Lemma}
	Let $X$ be regular, then the following are equivalent:

	Every open cover of $X$ has a refinement that is:
	\begin{itemize}
		\item An open cover that is $\sigma$-locally finite.
		\item A covering that is locally finite.
		\item A closed cover that is locally finite.
		\item An open cover that is locally finite.
	\end{itemize}
\end{lemma}
\begin{proof}
	SORRY
\end{proof}

\begin{theorem}{Every Metrizable Space is Paracompact}{Every Metrizable Space is Paracompact}
	Every metrizable space is paracompact.
\end{theorem}
\begin{proof}
	Let $X$ be a metrizable space, and from lemma \ref{lem:Existence of Locally Finite Open Refinement}, every open cover of $X$ has a $\sigma$-locally finite open refinement. Since $X$ is regular, by Michael's lemma, every open cover of $X$ has a locally finite open refinement, so $X$ is paracompact.
\end{proof}

\begin{theorem}{Every Regular Lindel\"of Space is Paracompact}{Every Regular Lindelof Space is Paracompact}
	Every regular Lindel\"of space is paracompact. (A Lindel\"of space is a space in which every open cover has a countable subcover.)
\end{theorem}
\begin{proof}
	A countable subcover is obviously a $\sigma$-local finite refinement, using Michael's lemma would do.
\end{proof}

One more intersecting features of paracompact spaces is the existence of partitions of unity.

\begin{definition}{Partitions of Unity}{Partitions of Unity}
	Let $\left\{ U_{\alpha} \right\}_{\alpha\in J}$ be an indexed open covering of $X$, then an indexed family of continuous functions
	\begin{equation*}
		\phi_{\alpha}: X \to [0,1], \quad \alpha \in J
	\end{equation*}
	is called a \textbf{partition of unity} on $X$ dominated by $\left\{ U_{\alpha} \right\}_{\alpha\in J}$ if
	\begin{itemize}
		\item $\forall \alpha\in J, \supp \phi_{\alpha} \subseteq U_{\alpha}$.
		\item The indexed family $\left\{ \supp \phi_{\alpha} \right\}$ is locally finite.
		\item $\forall x \in X, \sum_{\alpha\in J} \phi_{\alpha}(x) = 1$.
	\end{itemize}
	(The sum is well-defined due to local finiteness: there are only finitely many non-zero terms.)
\end{definition}

Now, we construct partitions of unity in paracompact Hausdorff spaces.

\begin{lemma}{Shrinking Lemma}{Shrinking Lemma}
	Let $X$ be a paracompact Hausdorff space, and let $\left\{ U_{\alpha} \right\}_{\alpha\in J}$ be an open cover of $X$. Then there is a local finite indexed family open cover $\left\{ V_{\alpha} \right\}_{\alpha\in J}$ of $X$ such that $\overline{V_{\alpha}} \subseteq U_{\alpha}$ for each $\alpha \in J$.
\end{lemma}
\begin{proof}
	SORRY
\end{proof}

\begin{remark}
	Sometimes we say that $\left\{ V_{\alpha} \right\}_{\alpha\in J}$ is a \textbf{precise refinement} of $\left\{ U_{\alpha} \right\}_{\alpha\in J}$ if $\overline{V_{\alpha}} \subseteq U_{\alpha}$ for each $\alpha \in J$.
\end{remark}

\begin{theorem}{Existence of Partitions of Unity}{Existence of Partitions of Unity}
	Let $X$ be a paracompact Hausdorff space, and let $\left\{ U_{\alpha} \right\}_{\alpha\in J}$ be an open cover of $X$. Then there is a partition of unity on $X$ dominated by $\left\{ U_{\alpha} \right\}_{\alpha\in J}$.
\end{theorem}
\begin{proof}
	SORRY
\end{proof}

Partitions of unity are very useful in patching local constructed functions to a global one.

\begin{theorem}{Patch Theorem}{Patch Theorem}
	Let $X$ be a paracompact Hausdorff space, and let $\mathcal{C}$ be a collection of subsets of $X$. For each $C\in \mathcal{C}$, let $\epsilon_{C}>0$. If $\mathcal{C}$ is locally finite, then there is a continuous function $f: X \to \mathbb{R}_+$ such that for each $C\in \mathcal{C}$, $f(x) \leq \epsilon_C$ for all $x \in C$.
\end{theorem}
\begin{proof}
SORRY
\end{proof}

\section{The Smirnov Metrization Theorem}

\begin{definition}{Locally Metrizable}{Locally Metrizable}
	A topological space $X$ is said to be \textbf{locally metrizable} if for each $x \in X$, there is a neighborhood $U$ of $x$ that is metrizable.
\end{definition}

\begin{theorem}{The Smirnov Metrization Theorem}{The Smirnov Metrization Theorem}
	A topological space $X$ is metrizable if and only if $X$ is paracompact, Hausdorff, and locally metrizable.
\end{theorem}

\end{document}
