\documentclass[../main.tex]{subfiles}

\begin{document}
\chapter{Metrization Theorems and Paracompactness}

The Urysohn Metrization Theorem states that a topological space is metrizable if it is regular and has a countable basis. And we hope to strengthen the condition to give an equivalent condition.

We know that the regularity is necessary, but the second countability is not. The eventual conclusion involves a new notion called locally finiteness.

Another way to say a basis $\mathcal{B}$ is countable is that $\mathcal{B}$ can be expressed in
\begin{equation*}
	\mathcal{B} = \bigcup_{n\in \mathbb{Z}_+} \mathcal{B}_n, \qquad \text{where each } \mathcal{B}_n \text{ is finite.} 
\end{equation*}
It is rather aukward, but it gives us a hint to weaken the second countability condition: The Nagata-Smirnov condition, by means of local finiteness.


\section{Local Finiteness}
\begin{definition}{Local Finiteness}{Local Finiteness}
	Let $X$ be a topological space. A collection $\mathcal{A}$ of subsets of $X$ is said to be \textbf{locally finite} if for every $x \in X$, there exists an open neighborhood $U$ of $x$ such that $U$ intersects only finitely many elements of $\mathcal{A}$.
\end{definition}

\begin{example}{Local Finiteness}{Local Finiteness}
	\begin{itemize}
		\item The collection $\mathcal{A} = \left\{ (n,n+2): n\in \mathbb{Z} \right\}$ is locally finite in $\mathbb{R}$.
		\item The collection $\mathcal{B} = \left\{ (1, 1 / n): n \in \mathbb{Z}^+ \right\}$ is locally finite in $(0,1)$ but not in $\mathbb{R}$.
	\end{itemize}
\end{example}

\begin{proposition}{Properties of Local Finiteness}{Properties of Local Finiteness}
	Let $X$ be a topological space and let $\mathcal{A}$ be a locally finite collection of subsets of $X$. Then
	\begin{itemize}
		\item Every subcollection of $\mathcal{A}$ is locally finite.
		\item The collection $\mathcal{B} = \left\{ \overline{A} : A \in \mathcal{A} \right\}$ is locally finite.
		\item $\overline{\bigcup_{A \in \mathcal{A}} A} = \bigcup_{A \in \mathcal{A}} \overline{A}$.
	\end{itemize}
\end{proposition}
\begin{proof}
	First one is trivial. For the second one, note that any open set intersecting $\overline{A}$ needs to intersect $A$. For the last one, we let
	\begin{equation*}
		Y = \bigcup_{A \in \mathcal{A}} A.
	\end{equation*}
	In general, we have $\overline{Y} \supseteq \bigcup_{A \in \mathcal{A}} \overline{A}$. For the other direction, let $x \in \overline{Y}$ and let $U$ be an open neighborhood of $x$ intersecting $Y$. Since $\mathcal{A}$ is locally finite, there exists an open neighborhood $V$ of $x$ intersecting only finitely many elements of $\mathcal{A}$, say $A_1, A_2, \ldots, A_n$. Then $x$ is in one of $\overline{A_i}$: of not, then $U - \bigcup_{i=1}^n \overline{A_i}$ is an open neighborhood of $x$ not intersecting $Y$, a contradiction.
\end{proof}


\begin{remark}
	An analogous concept is a \textbf{local finite index family} $\left\{ A_{\alpha} \right\}_{\alpha \in J}$, where $J$ is an index set. (This allows repeat sets). It is obvious that a local finite index family is a local finite collection if it is local finite as a collection of sets and there are at most finite multiple repeat non-empty set in the family.
\end{remark}



\end{document}
