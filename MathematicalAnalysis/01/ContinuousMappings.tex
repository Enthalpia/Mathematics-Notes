\documentclass[../main.tex]{subfiles}

\begin{document}
\chapter{Continuous Mappings (General)}

\section{Metric Spaces}
\subsection{Definition and Examples}
\begin{definition}{Metric Spaces}{Metric Spaces}
A set  $X$ is a metric space if it has a function
\begin{equation}
d: X \times X \rightarrow \mathbb{R}
\end{equation}
such that
\begin{itemize}
\item $d(x_1,x_2)=0 \Leftrightarrow x_1=x_2$.
\item $d(x_1,x_2)=d(x_2,x_1)$.
\item $d(x_1,x_3) \leq d(x_1,x_2)+d(x_2,x_3)$.
\end{itemize}
\end{definition}
Note that setting $x_3=x_1$ in triangle inequality we have $d(x_1,x_2) \geq 0$.

\begin{example}{Metrics on $\mathbb{R}^n$}{Metrics Rn}
In $\mathbb{R}^n$ we have the traditional Euclidean metric
\begin{equation}
d(x,y) = \sqrt{\sum_{i=1}^{n} \left|x_i-y_i\right|^2}
\end{equation}
Or we can have a more general
\begin{equation}
d_p(x,y) = \left(\sum_{i=1}^{n} \left|x_i-y_i\right|^p\right)^{\frac{1}{p}}, \text{ where }p \geq 1.
\end{equation}
The validity comes from the Minkowski inequality.

Generalizing by $p \rightarrow \infty $ we have clearly
\begin{equation}
d(x,y) = \max_{a\leq x\leq b} \left|x_i-y_i\right|
\end{equation}
\end{example}

\begin{example}{Metrics on $C[a,b]$}{Metrics on Cab}
Similarly in $C[a,b]$, that is the continuous functions on $[a,b]$, we can define
\begin{equation}
d_p(f,g) = \left(\int_a^b \left|f-g\right|^p(x) \mathrm{d} x\right)^{\frac{1}{p}}, \text{ where }p\geq 1
\end{equation}
and limiting to infinity we have
\begin{equation}
d(f,g) = \sup \left|f(x)-g(x)\right|
\end{equation}
\end{example}

\subsection{Open and Closed Sets of a Metric Space}

\begin{definition}{Open Balls}{Open Balls}
For $\delta>0$ and $a\in X$, we define the set
\begin{equation}
B(a,\delta) = \left\{ x\in X \mid d(a,x)<\delta \right\}
\end{equation}
to be the open ball with center $a\in X$ and radius $\delta$ or the $\delta$-neighborhood of $a$.
\end{definition}
\begin{definition}{Open Sets and Closed Sets}{Open Sets and Closed Sets}
A set $G \subseteq X$ is open in $(X,d)$ if
\begin{equation*}
\forall x\in G, \exists \delta>0, B(x,\delta)\subseteq G
\end{equation*}

A set $F \subseteq X$ is closed iff $X-F$ is open.

An open set containing $x$ is said to be a neighborhood of $x$.
\end{definition}

We now denote the closed ball
\begin{equation}
\tilde{B}(a,r) = \left\{ x\in X\mid d(a,x)\leq r \right\}
\end{equation}

\begin{definition}{Interior, Exterior and Boundary points}{Interior Exterior and Boundary points}
Let $E \subseteq X$
\begin{itemize}
\item An interior point of $E$ iff some neighborhood of it $\subseteq E$.
\item An exterior point of $E$ iff some neighborhood of it $\subseteq X-E$.
\item A boundary point of $E$ is neither an interior point nor an exterior point of $E$.
\end{itemize}
\end{definition}

\begin{definition}{Limit Points}{Limit Points}
A point  $a\in X$ is a limit point of $E \subseteq X$ iff $\forall $ neighborhood $O(a)$ we have $E \cap O(a)$ is infinite.

We denote $\overline{E} = E \cup \text{ the limit points of }E$.
\end{definition}

\begin{proposition}{Condition to be Closed}{Condition to be Closed}
A set $F \subseteq X$ is closed iff it contains all its limit points. That is $F = \overline{F}$.
\end{proposition}

\subsection{Subspaces of a Metric Space}
\begin{definition}{Subspace of a Metric Space}{Subspace of a Metric Space}
A metric space $(X_1,d_1)$ is a subspace of $(X,d)$ iff
\begin{itemize}
\item $X_1 \subseteq X$.
\item $\forall a,b\in X_1, d_1(a,b)=d(a,b)$.
\end{itemize}
\end{definition}

\begin{proposition}{Open sets in Subspaces}{Open sets in Subspaces}
If $(X_1,d_1)$ is a subspace of $(X,d)$, then the open sets in $X_1$ is exactly $X_1\cap E$ where $E$ is an open set of $X$.
\end{proposition}

\subsection{Direct Product of Metric Spaces}
If $(X_1,d_1)$ and $(X_2,d_2)$ are two metric spaces, one can introduce a metric on the set $X_1 \times X_2$. Like
\begin{equation*}
d((x_1,x_2),(x_1',x_2')) = \sqrt{d_1^2(x_1,x_1') + d_2^2(x_2,x_2')}
\end{equation*}
\begin{equation*}
d((x_1,x_2),(x_1',x_2')) = d_1(x_1,x_1')+d_2(x_2,x_2')
\end{equation*}
\begin{equation*}
d((x_1,x_2),(x_1',x_2')) = \max \left\{ d_1(x_1,x_1'),d_2(x_2,x_2') \right\}
\end{equation*}


\section{Topological Spaces}
\begin{definition}{Topological Spaces}{Topological Spaces}
A set $X$ has a topology $\mathcal{T}$, where $\mathcal{T} \subseteq \mathcal{P}(X)$ is a collection of subsets of $X$ that are called open sets, with the restriction
\begin{itemize}
\item $\emptyset \in \mathcal{T}, X\in \mathcal{T}$.
\item $\forall \alpha\in A, \mathcal{T}_{\alpha}\in \mathcal{T}$, we have $\bigcup_{\alpha\in A} \mathcal{T}_{\alpha}\in \mathcal{T} $.
\item $\forall \mathcal{T}_i\in \mathcal{T}, \bigcap_{i=1}^{n} \mathcal{T}_i\in \mathcal{T}$.
\end{itemize}
\end{definition}

A topology can be generated by a metric as above. We now introduce base of a topology.
\begin{definition}{Base of a Topology}{Base of a Topology}
A base of a topological space $(X,\mathcal{T})$ is a set $\mathcal{B} \subseteq \mathcal{T}$ such that
\begin{equation*}
\forall G\in \mathcal{T}, G = \bigcup_{\alpha\in A}B_{\alpha}, \text{ for some }B_{\alpha}\in \mathcal{B} 
\end{equation*}

The minimal cardinality among all the bases of a topological space is called its weight.
\end{definition}

Thus all the open balls is a base of the topology given by a metric.

\begin{example}{The germs of Continuous Functions}{The germs of Continuous Functions}
Consider the set $C(\mathbb{R},\mathbb{R})$ of real-valued continuous functions defined on the entire $\mathbb{R}$ line. For an $a\in \mathbb{R}$, we define an equivalence relation $\sim$ :
\begin{equation}
f\sim g \Leftrightarrow  \exists \text{ a neighborhood }U(a),\forall x\in U(a), f(x)=g(x)
\end{equation}
We denote the equivalent class (called germs) $f_a$.

We now define a neighborhood of $f_a$. Let $f$ be a function that generates $f_a$, the set $\left\{ f_x \mid x\in \mathbb{R} \right\}$ is a neighborhood of $f_a$. Taking all the neighborhoods as a base we get a topology.
\end{example}

\begin{definition}{Hausdorff Space}{Hausdorff}
A topological space is Hausdorff if \emph{any two distinct points have non-intersecting neighborhoods}.
\end{definition}
\begin{definition}{Dense}{Dense}
A set $E \subseteq X$ is (everywhere) dense in $X$ if
\begin{equation*}
\forall x\in X, \forall U(x), E\cap U(x) \neq \emptyset 
\end{equation*}
\end{definition}
It is easy to show that $\mathbb{Q}$ is dense in $\mathbb{R}$.

\begin{definition}{Separable Spaces}{Separable Spaces}
A metric space having a countable dense set is called separable.
\end{definition}


\section{Compact Sets}
\subsection{Definition}
\begin{definition}{Compact Sets}{Compact Sets}
A set $K$ in topological space $(X,\mathcal{T})$ is compact if every open cover of $K$ has a finite subcover.
\end{definition}

\begin{proposition}{Compact Conditions}{Compact Conditions}
A subset $K \subseteq X$ is compact in $(X,d)$ iff $K$ is compact in $(K,d)$.

Which mean that compactness has some sense of locality.
\end{proposition}
\begin{proof}
	Using proposition \ref{prop:Open sets in Subspaces} would do.
\end{proof}

\begin{lemma}{Compact Sets are Closed}{Compact Sets are Closed}
If $K$ is a compact set in a Hausdorff space $(X,\mathcal{T})$, then $K$ is closed in $X$
\end{lemma}
\begin{proof}
We shall show that every limit point of $K$ belongs to $K$. Suppose $x_0\notin K$ is a limit point of $K$, then $\forall x\in K$,we construct an open neighborhood $G(x)$ such that $\exists O_x(x_0)\cap G(x) = \emptyset $, then all of $G(x)$ forms an open cover of $K$. Select a finite subcover $G(x_1), \ldots ,G(x_n)$, then $O = \bigcap_{i=1}^{n} O_{x_i}(x_0)$ is a neighborhood of $x_0$ but $K\cap O = \emptyset $, so $x_0$ cannot be a limit point of $K$.
\end{proof}


\begin{lemma}{Nested Compact Sets}{Nested Compact Sets}
If $K_1 \supset K_2 \supset \cdots \supset K_n \supset \cdots $ is a nested sequence of nonempty compact sets, then $\bigcap_{i=1}^{\infty } K_i$ is nonempty.
\end{lemma}
\begin{proof}
	By lemma \ref{lem:Compact Sets are Closed} the sets $G_i = K_1-K_i$ are open in $K_1$. If the intersection $\bigcap_{i=1}^{\infty } G_i$ is empty, then the sequence $G_1 \subseteq G_2 \subseteq \cdots \subseteq G_n \subseteq  \cdots $ forms a covering of $K_1$. Extracting a finite subcover gives a contradiction.
\end{proof}

\begin{lemma}{Closed subsets of Compact Sets}{Closed subsets of Compact Sets}
A closed subset $F$ of a compact set $K$ is itself compact.
\end{lemma}
\begin{proof}
Let $\left\{ G_{\alpha} \right\}$ be an open covering of $F$. Adjoining $\left\{ G_{\alpha} \right\}\cup K \backslash F$ we obtain an open covering of $K$.
\end{proof}

\subsection{Metric Compact Sets}

\begin{definition}{$\epsilon$-grid}{epsilon-grid}
The set $E \subseteq X$ is called an $\epsilon$-grid in the metric space $(X,d)$ if for $\forall x\in X, \exists e\in E, d(e,x) < \epsilon$.
\end{definition}

\begin{lemma}{Finite $\epsilon$-grid}{Finite epsilon-grid}
If a metric space $(X,d)$ is compact, then for $\forall \epsilon>0$ there exists a finite $\epsilon$-grid in $X$.
\end{lemma}
\begin{proof}
$\forall x\in K$ we choose an open ball $B(x,\epsilon)$. From the open covering of $K$ by these balls we select a finite subcover $B(x_i,\epsilon)$, and the $x_i$ forms a finite $\epsilon$-grid.
\end{proof}

\begin{theorem}{Criterion for Compactness in a metric space}{Criterion for Compactness in a metric space}
A metric space $(X,d)$ is compact iff from each sequence there is a subsequence that converges to a point in $K$.
\end{theorem}

\section{Connected Topological Spaces}
\begin{definition}{Connected Topological Spaces}{Connected Topological Spaces}
A topological space $(X,\tau)$ is connected if the only clopen subsets are $X$ and $\emptyset $. (It cannot be represented as the union of two disjoint nonempty open/closed subsets).
\end{definition}

\section{Complete Metric Spaces}
\begin{definition}{Complete Metric Spaces}{Complete Metric Spaces}
A metric space $(X,d)$ is complete if every Cauchy sequence of its points is convergent.
\end{definition}

\subsection{Completion of a Metric Space}
\begin{definition}{Completion of a Metric Space}{Completion of a Metric Space}
The smallest complete metric space containing a given metric space $(X,d)$ is the completion of $(X,d)$.
\end{definition}

\section{Continuous Mapping of Topological Space}
\subsection{The limit of a Mapping}
\begin{definition}{Limit of a Mapping}{Limit of a Mapping}
Let $f:X \rightarrow Y$. Let $\mathcal{B}$ be a basis of $X$. Then the point $A\in Y$ is the limit of the mapping $f$ over basis $\mathcal{B}$ if $\forall $ neighborhood $V(A)$ of $A\in Y$ there exists $B\in \mathcal{B}$ such that $f(B) \subseteq V(A)$. Denoted $\lim_{\mathcal{B}}f(x)=A $.
\end{definition}

If $(X,d_X)$ and $(Y,d_Y)$ are metric spaces, we can rephrase the definition as follows by $\epsilon-\delta$ language:

\begin{equation*}
\forall \epsilon>0, \exists \delta>0, \forall x\in X (0 < d_X(a,x) <\delta \rightarrow d_Y(f(x),A) <\epsilon)
\end{equation*}
Then we denote
\begin{equation*}
\lim_{x \to a} f(x) = A.
\end{equation*}

\begin{definition}{Continuity on Topological Spaces}{Continuity on Topological Spaces}
A mapping $f:X \rightarrow Y$ of a topological space $(X,\tau_X)$ and $(Y,\tau_Y)$ is continuous at a point $a\in X$ if for every neighborhood $V(f(a)) \subseteq Y$, there exists a neighborhood $U(a) \subseteq X$ such that $f(U(a)) \subseteq V(f(a))$.
\begin{equation*}
\forall V(f(a)), \exists U(a),f(U(a)) \subseteq V(f(a)).
\end{equation*}

The mapping $f:X \rightarrow Y$ is continuous iff it is continuous at every point $x\in X$. The set of continuous mappings $f:X \rightarrow  Y$ will be denoted $C(X,Y)$.
\end{definition}

In the sense of metric space, we can rephrase that:
\begin{equation*}
\forall \epsilon>0, \exists \delta>0, \forall x\in X (0 < d_X(a,x) <\delta \rightarrow d_Y(f(x),f(a)) <\epsilon)
\end{equation*}
or
\begin{equation*}
\lim_{x \to a} f(x) = f(a).
\end{equation*}

\subsection{Local Properties of a Continuous Mapping}

\begin{theorem}{Continuity of Composition}{Continuity of Composition}
Let $X,Y,Z$ be topological spaces, if $f:X \rightarrow Y$ is continuous at $a\in X$ and $g:Y \rightarrow Z$ is continuous at $f(a)\in Y$, then $g \circ f$ is continuous at $a$.
\end{theorem}


\end{document}
