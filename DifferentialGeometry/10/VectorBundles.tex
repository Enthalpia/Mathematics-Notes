\documentclass[../main.tex]{subfiles}

\begin{document}
\chapter{Vector Bundles}

\section{Vector Bundles}
We have encountered the tangent bundle $TM$ of a smooth manifold $M$. Locally it looks like a product $U \times \mathbb{R}^n$, but globally it may be twisted in a nontrivial way. This motivates the following definition.

\begin{definition}{Vector Bundles}{Vector Bundles}
  Let $M$ be a topological space. A (real) \textbf{vector bundle} of rank $k$ over $M$ is a topological space $E$ together with a surjective continuous map $\pi: E \to M$ such that
  \begin{itemize}
    \item For each $p \in M$, the fiber $E_p := \pi^{-1}(p)$ is equipped with the structure of a real vector space of dimension $k$.
    \item For each $p \in M$, there exists an open neighborhood $U$ of $p$ in $M$ and a homeomorphism $\Phi: \pi^{-1}(U) \to U \times \mathbb{R}^k$ called a \textbf{local trivialization} such that 
      \begin{itemize}
        \item $\pi_U \circ \Phi = \pi$, where $\pi_U: U \times \mathbb{R}^k \to U$ is the projection onto the first factor.
        \item For each $q \in U$, the restriction $\Phi|_{E_q}: E_q \to \{q\} \times \mathbb{R}^k$ is a vector space isomorphism.
      \end{itemize}
  \end{itemize}
  If $M,E$ are smooth manifolds, with or without boundary, and $ \pi$ is a smooth map, and the local trivializations $\Phi$ are diffeomorphisms, then we say that $E$ is a \textbf{smooth vector bundle} over $M$.

  The space $E$ is called the \textbf{total space} of the vector bundle, $M$ is called the \textbf{base space}, and the map $\pi$ is called the \textbf{projection map}.
\end{definition}

\begin{figure}[ht]
    \centering
    \incfig{vector-bundles}
    \caption{Vector Bundles}
    \label{fig:vector-bundles}
\end{figure}

If there exists a local trivialization of $E$ over all of $M$, then we say that it is a \textbf{global trivialization}, and $E$ is called a \textbf{trivial vector bundle}. In this case, $E$ is homeomorphic (or diffeomorphic, in the smooth case) to the product manifold $M \times \mathbb{R}^k$.

\begin{example}{Vector Bundles}{Vector Bundles}
  \begin{itemize}
    \item Product bundles: For any topological space (or smooth manifold) $M$ and integer $k \geq 0$, the product space $M \times \mathbb{R}^k$ with the projection map $\pi: M \times \mathbb{R}^k \to M$ defined by $\pi(p,v) = p$ is a trivial vector bundle of rank $k$ over $M$.
    \item The M\"obius Bundle: Define an equivalence relation on $\mathbb{R}^2$ generated by
      \begin{equation*}
        (x,y) \sim (x + 1, -y).
      \end{equation*}
      Let $E = \mathbb{R}^2 / \sim$ be the quotient space, and let $q: \mathbb{R}^2 \to E$ be the quotient map. For any $[-r,r] \subseteq \mathbb{R}$, the image of the strip $\mathbb{R} \times [-r,r]$ under $q$ is called the \textbf{M\"obius band}. It is a compact manifold with boundary. It is easy to see that $E \rightarrow S^1$ is a smooth vector bundle of rank 1 over the circle, called the \textbf{M\"obius bundle}, which is nontrivial.
  \end{itemize}
\end{example}

\begin{proposition}{Tangent Bundle as Vector Bundle}{Tangent Bundle as Vector Bundle}
  Let $M$ be a smooth manifold of dimension $n$, with or without boundary, and let $TM$ be its tangent bundle. With the projection map $\pi: TM \to M$ defined by $\pi(v) = p$ for $v \in T_pM$, $TM$ is a smooth vector bundle of rank $n$ over $M$.
\end{proposition}

Next, we show how two local trivializations of a vector bundle are related on their overlap.

\begin{lemma}{Local Trivialization Overlap}{Local Trivialization Overlap}
  Let $\pi: E \to M$ be a smooth vector bundle of rank $k$ over a smooth manifold $M$. Suppose $\Phi: \pi^{-1}(U) \to U \times \mathbb{R}^k$ and $\Psi: \pi^{-1}(V) \to V \times \mathbb{R}^k$ are two local trivializations of $E$ that $U \cap V \neq \emptyset$. Then there exists a smooth map $ \tau: U \cap V \to GL(k, \mathbb{R})$ such that the composition $\Phi \circ \Psi^{-1}: (U \cap V) \times \mathbb{R}^k \to (U \cap V) \times \mathbb{R}^k$ is given by
  \begin{equation}
    \Phi \circ \Psi^{-1}(p,v) = (p, \tau(p)v)
  \end{equation}
  Here $\tau$ is called the \textbf{transition function} between the two local trivializations.
\end{lemma}
This means that the change of coordinates between two local trivializations is a smoothly varying linear isomorphism along the manifold points.

\begin{lemma}{Vector Bundle Chart Lemma}{Vector Bundle Chart Lemma}
  Let $M$ be a smooth manifold, with or without boundary, and suppose for each $p \in M$, we have a vector space $E_p$ of dimension $k$. Let $E = \bigsqcup_{p \in M} E_p$ and $\pi: E \to M$ be the natural projection map $E_p \mapsto p$. Suppose we are given the following data:
  \begin{itemize}
    \item An open cover $\{U_\alpha\}$ of $M$.
    \item For each $\alpha$, a bijection $\Phi_\alpha: \pi^{-1}(U_\alpha) \to U_\alpha \times \mathbb{R}^k$ whose restriction to each $E_p$ is a vector space isomorphism onto $\{p\} \times \mathbb{R}^k \cong \mathbb{R}^k$.
    \item For each $\alpha, \beta$ that $U_\alpha \cap U_\beta \neq \emptyset$, a smooth map $\tau_{\alpha \beta}: U_\alpha \cap U_\beta \to GL(k, \mathbb{R})$ such that for all $p \in U_\alpha \cap U_\beta$ and $v \in \mathbb{R}^k$,
      \begin{equation}
        \Phi_\alpha \circ \Phi_\beta^{-1}(p,v) = (p, \tau_{\alpha \beta}(p)v).
      \end{equation}
  \end{itemize}
  Then $E$ has a unique topology and smooth structure such that it is a smooth manifold with or without boundary, and also a smooth vector bundle of rank $k$ over $M$ with smooth local trivializations $\{U_\alpha, \Phi_\alpha\}$.
\end{lemma}

\begin{example}{Whitney Sums}{Whitney Sums}
  Given a smooth manifold $M$ and two smooth vector bundles $\pi_1: E_1 \to M$ and $\pi_2: E_2 \to M$ of ranks $k_1$ and $k_2$ respectively, we can construct a new vector bundle called the \textbf{Whitney sum} of $E_1$ and $E_2$,:
  \begin{itemize}
    \item The total space is $E = E_1 \oplus E_2 = \bigsqcup_{p \in M} (E_1)_p \oplus (E_2)_p$.
    \item The projection map $\pi: E \to M$ is defined by $(E_1)_p \oplus (E_2)_p \rightarrow p$.
    \item For a neighborhood $U$ of $p \in M$ with local trivializations $\Phi_1: \pi_1^{-1}(U) \to U \times \mathbb{R}^{k_1}$ and $\Phi_2: \pi_2^{-1}(U) \to U \times \mathbb{R}^{k_2}$, we define a local trivialization for $E$ by
      \begin{equation*}
        \Phi: \pi^{-1}(U) \to U \times \mathbb{R}^{k_1 + k_2}, \quad \Phi(v_1, v_2) = (p, ( \pi_{\mathbb{R}^{k_1}} \circ \Phi_1(v_1), \pi_{\mathbb{R}^{k_2}} \circ \Phi_2(v_2))),
      \end{equation*}
  \end{itemize}
\end{example}

\begin{example}{Restricting a Vector Bundle}{Restricting a Vector Bundle}
  Suppose $\pi: E \to M$ is a vector bundle of rank $k$ and $S \subseteq M$ is any subset. Then the restriction of $E$ to $S$ is a vector bundle. The total space is $E|_S = \bigsqcup_{p \in S} E_p$, and the projection map is the restriction $\pi|_{E|_S}: E|_S \to S$. If $\Phi$ is a local trivialization of $E$ over an open set $U$ in $M$, then its restriction is $\Phi|_S: \pi^{-1}(U\cap S) \to (U \cap S) \times \mathbb{R}^k$ is a local trivialization of $E|_S$ over $U \cap S$.

  If $E$ is a smooth vector bundle and $S \subseteq M$ is an immersed or embedded submanifold, then $E|_S$ is also a smooth vector bundle over $S$.

  For the tangent bundle, the restriction $TM|_S$ is called the \textbf{ambient tangent bundle} of $S$ in $M$. (Note it is NOT the same as the tangent bundle $TS$ of $S$ because it still maintains full dimension of $M$ in each fiber.)
\end{example}

\section{Local and Global Sections}
Similar to previous definitions about sections.

\begin{definition}{Sections of Vector Bundles}{Sections of Vector Bundles}
  Let $ \pi: E \rightarrow M$ be a vector bundle. A (global) section of $E$ is a section of the projection map $\pi$, i.e., a continuous map $ \sigma: M \rightarrow E$ such that $ \pi \circ \sigma = \operatorname{id}_M$. For every $p \in M$, we have $ \sigma(p) \in E_p$.

  A local section of $E$ over an open subset $U \subseteq M$ is a continuous map $ \sigma: U \rightarrow E$ such that $ \pi \circ \sigma = \operatorname{id}_U$.

  If $M$ is a smooth manifold, with or without boundary, and $E$ is a smooth vector bundle over $M$, then a smooth (global or local) section of $E$ is a smooth map $ \sigma$ as above. A not necessarily continuous section is called a \textbf{rough section}. The zero section is a global section $ \zeta: M \rightarrow E$ defined by $ \zeta(p) = 0\in E_p$ for all $p \in M$. The support of a section $ \sigma$ is defined as
  \begin{equation*}
    \supp(\sigma) = \overline{\{p \in M : \sigma(p) \neq 0\}}.
  \end{equation*}
\end{definition}

\begin{example}{Sections of Vector Bundles}{Sections of Vector Bundles}
  Suppose $M$ is a smooth manifold, with or without boundary.
  \begin{itemize}
    \item Sections of $TM$ are vector fields on $M$.
    \item Given an immersed submanifold $S \subseteq M$, sections of the ambient tangent bundle $TM|_S$ are called vector fields along $S$. It is different from a vector field on $S$ because the vectors lie in the tangent spaces of $M$ instead of $S$.
    \item If $E = M \times \mathbb{R}^k$ is a trivial vector bundle, then sections of $E$ are exactly (one-to-one correspondence) continuous functions from $M$ to $\mathbb{R}^k$:
      \begin{equation*}
        F:M \rightarrow \mathbb{R}^k, \Leftrightarrow \tilde{F}: M \rightarrow M \times \mathbb{R}^k, \quad \tilde{F}(p) = (p, F(p)).
      \end{equation*}
      So $C^\infty (M)$ can be identified as the space of smooth sections of the trivial line bundle $M \times \mathbb{R}$.
  \end{itemize}
\end{example}

If $E \rightarrow M$ is a smooth vector bundle, then the set of all smooth global sections of $E$ is a vector space over $\mathbb{R}$ under pointwise addition and scalar multiplication, denoted by $\Gamma(E)$.
\begin{equation}
  (c_1 \sigma_1 + c_2 \sigma_2)(p) = c_1 \sigma_1(p) + c_2 \sigma_2(p), \quad \forall p \in M.
\end{equation}
Then we have $\mathfrak{X}(M) = \Gamma(TM)$.

We can also multiply a section by a smooth function. If $f\in C^\infty(M)$ and $\sigma \in \Gamma(E)$, then we define a new section $f \sigma \in \Gamma(E)$ by
\begin{equation*}
  (f \sigma)(p) = f(p) \sigma(p), \quad \forall p \in M.
\end{equation*}

\begin{lemma}{Extension Lemma for Vector Bundles}{Extension Lemma for Vector Bundles}
  Let $ \pi: E \rightarrow M$ be a smooth vector bundle over a smooth manifold $M$, with or without boundary. If $A \subseteq M$ is closed and $\sigma: A \rightarrow E$ is a section of $E|_A$ that is smooth in a neighborhood of each point of $A$. For each open neighborhood $U$ of $A$ in $M$, there exists a smooth global section $\tilde{\sigma} \in \Gamma(E)$ such that $\tilde{\sigma}|_A = \sigma$ and $\supp(\tilde{\sigma}) \subseteq U$.
\end{lemma}

\subsection{Local and Global Frames}

We can view sections as generalizations of vector fields. Similarly, we generalize the concept of frames.

\begin{definition}{Frames of Vector Bundles}{Frames of Vector Bundles}
  Let $ \pi: E \rightarrow M$ be a vector bundle. If $U \subseteq M$ is an open subset, a $k$-tuple of local sections $(\sigma_1, \ldots, \sigma_k)$ of $E$ over $U$ is linearly independent if for each $p \in U$, the set $\{\sigma_1(p), \ldots, \sigma_k(p)\}$ is linearly independent in the fiber $E_p$. It is a local frame of $E$ over $U$ if for each $p \in U$, the set $\{\sigma_1(p), \ldots, \sigma_k(p)\}$ is a basis of the fiber $E_p$. It is called a \textbf{global frame} of $E$ if it is defined over all of $M$.
\end{definition}

\begin{proposition}{Completion of Local Frames for Vector Bundles}{Completion of Local Frames for Vector Bundles}
  Let $ \pi: E \rightarrow M$ be a smooth vector bundle of rank $k$.
  \begin{itemize}
    \item If $( \sigma_1, \ldots, \sigma_m)$ is a linearly independent $m$-tuple of smooth local sections of $E$ over an open subset $U \subseteq M$, with $m < k$, then for each $p \in U$, there exists an open neighborhood $V \subseteq U$ of $p$ and smooth local sections $ \sigma_{m+1}, \ldots, \sigma_k$ of $E$ over $V$ such that $( \sigma_1, \ldots, \sigma_k)$ is a local frame of $E$ over $V$.
    \item If $(v_1, \ldots , v_m)$ is a linearly independent $m$-tuple of vectors in the fiber $E_p$ over some point $p \in M$, with $m \leq  k$, then there exists a smooth local frame $( \sigma_1, \ldots, \sigma_k)$ of $E$ over an open neighborhood $U$ of $p$ such that $\sigma_i(p) = v_i$ for $1 \leq i \leq m$.
    \item If $A \subseteq M$ is a closed set and $( \tau_1, \ldots , \tau_k)$ is a linearly independent $k$-tuple of sections of $E|_A$ that is smooth in a neighborhood of each point of $A$, then there exists a smooth global frame $( \sigma_1, \ldots, \sigma_k)$ of $E$ over some open neighborhood $U$ of $A$ such that $\sigma_i|_A = \tau_i$ for $1 \leq i \leq k$.
  \end{itemize}
\end{proposition}

Intuitively, local frames of a vector bundle are intimately related to local trivializations. This is indeed the case.

\begin{proposition}{Local Frames and Local trivializations}{Local Frames and Local trivializations}
  Let $ \pi: E \rightarrow M$ be a smooth vector bundle. If $ \Phi: \pi^{-1}(U) \rightarrow U \times \mathbb{R}^k$ is a local trivialization of $E$ over an open subset $U \subseteq M$, then we can construct a local frame $( \sigma_1, \ldots, \sigma_k)$ of $E$ over $U$ by defining
  \begin{equation*}
    \sigma_i(p) = \Phi^{-1}(p, e_i), \quad \forall p \in U,
  \end{equation*}

  Conversely, if $( \sigma_1, \ldots, \sigma_k)$ is a smooth local frame of $E$ over an open subset $U \subseteq M$, then we can construct a smooth local trivialization $ \Phi: \pi^{-1}(U) \rightarrow U \times \mathbb{R}^k$ of $E$ over $U$ by defining
  \begin{equation*}
    \Phi(v) = (p, (v^1, \ldots, v^k)), \quad \text{where } v = v^i \sigma_i(p) \in E_p, \; p \in U.
  \end{equation*}

  Therefore, a smooth vector bundle is smoothly trivial if and only if it admits a smooth global frame.
\end{proposition}

In a local chart $(V, \varphi)$ of $M$ with coordinates $(x^i)$, and suppose there is a local frame $( \sigma_i)$ of $E$ over $V$. The define $\tilde{\varphi}: \pi^{-1}(V) \rightarrow \varphi(V) \times \mathbb{R}^k$ by
\begin{equation*}
  \tilde{\varphi}(v^i \sigma_i(p)) = (x^1(p), \ldots, x^n(p), v^1, \ldots, v^k).
\end{equation*}
Then $(\pi^{-1}(V), \tilde{\varphi})$ is a smooth chart of the total space $E$.

\begin{proposition}{Local Frame Criterion for Smoothness}{Local Frame Criterion for Smoothness}
  Let $ \pi: E \rightarrow M$ be a smooth vector bundle, and let $ \tau: M \rightarrow E$ be a rough section of $E$. If $( \sigma_i)$ is a smooth local frame of $E$ over an open subset $U \subseteq M$, then $ \tau$ is smooth on $U$ if and only if the component functions $ \tau^i: U \rightarrow \mathbb{R}$ defined by
  \begin{equation*}
    \tau(p) = \tau^i(p) \sigma_i(p), \quad \forall p \in U
  \end{equation*}
  are smooth.
\end{proposition}

\begin{proposition}{Uniqueness of Smooth Structure on $TM$}{Uniqueness of Smooth Structure on TM}
  Let $M$ be a smooth $n$-manifold, with or without boundary. The topology and smooth structure on the tangent bundle $TM$ defined before are the unique ones such that $\pi: TM \rightarrow M$ is a smooth vector bundle with the given vector space structure on each fiber by derivations, and such that every coordinate vector field on $M$ is a smooth section of $TM$.
\end{proposition}
\begin{proof}
  Suppose there is another topology and smooth structure on $TM$ satisfying the same conditions. If $(U, \varphi)$ is a smooth chart of $M$, then the corresponding frame $ \partial / \partial x^i$ of $TM$ over $U$ is a smooth local frame over $U$. So there is a smooth local trivialization $ \Phi: \pi^{-1}(U) \rightarrow U \times \mathbb{R}^n$ associated to this local frame. But this is exactly the same as the local trivialization defined before.
\end{proof}

\section{Bundle Homomorphisms}

\begin{definition}{Bundle Homomorphisms}{Bundle Homomorphisms}
  Let $ \pi: E \rightarrow M$ and $ \pi': E' \rightarrow M'$ be vector bundles. A continuous map $F: E \rightarrow E'$ is a \textbf{bundle homomorphism} from $E$ to $E'$ if there exists a map $f: M \rightarrow M'$ such that the following diagram commutes:
  \begin{equation*}
    \begin{tikzcd}
      E \arrow{r}{F} \arrow{d}{\pi} & E' \arrow{d}{\pi'} \\
      M \arrow{r}{f} & M'
    \end{tikzcd}
  \end{equation*}
  with the property that for each $p \in M$, the restriction $F|_{E_p}: E_p \rightarrow E'_{f(p)}$ is a linear map between vector spaces. We say that $F$ covers $f$.

  We can deduce that $f$ is actually continuous and if the bundles and $F$ are smooth, then $f$ is also smooth.

  A bijection bundle homomorphism whose inverse is also a bundle homomorphism is called a \textbf{bundle isomorphism}. If there exists a bundle isomorphism between two vector bundles, then they are said to be \textbf{isomorphic}.

  If $M'=M$ is the same space, we say that $F$ is a bundle homomorphism over $M$ with base map $\operatorname{id}_M$.
\end{definition}

\begin{example}{Bundle Homomorphisms}{Bundle Homomorphisms}
  \begin{itemize}
    \item If $F: M \rightarrow N$ is a smooth map between smooth manifolds, then its differential $dF: TM \rightarrow TN$ is a smooth bundle homomorphism covering $F$.
    \item If $ \pi: E \rightarrow M$ is a smooth vector bundle, and $ S \subseteq M$ is an immersed submanifold, with or without boundary, then the inclusion map $i: E|_S \rightarrow E$ is a smooth bundle homomorphism covering the inclusion map of $S$ into $M$.
  \end{itemize}
\end{example}

\begin{definition}{Linear over $C^\infty(M)$}{Linear over Cinf(M)}
  Let $ \pi: E \rightarrow M$ and $ \pi': E' \rightarrow M$ be smooth vector bundles over the same smooth manifold $M$, with or without boundary. A map $\mathcal{F}: \Gamma(E) \rightarrow \Gamma(E')$ is said to be \textbf{linear over} $C^\infty(M)$ if for all $f, g \in C^\infty(M)$ and $ \sigma, \tau \in \Gamma(E)$, we have
  \begin{equation*}
    \mathcal{F}(f \sigma + g \tau) = f \mathcal{F}(\sigma) + g \mathcal{F}(\tau).
  \end{equation*}
\end{definition}

Now, if $E \rightarrow M$ and $E' \rightarrow M$ are smooth vector bundles over a smooth manifold $M$, with or without boundary, then if $F: E \rightarrow E'$ is a smooth bundle homomorphism over $M$, then $F$ induces a map $\tilde{F}: \Gamma(E) \rightarrow \Gamma(E')$ defined by
\begin{equation}
  \tilde{F}(\sigma)(p) = F(\sigma(p)), \quad \forall p \in M.
\end{equation}
It is easily verified that $\tilde{F}$ is linear over $C^\infty(M)$.

\begin{lemma}{Bundle Homomorphism Characterization Lemma}{Bundle Homomorphism Characterization Lemma}
  Let $ \pi: E \rightarrow M$ and $ \pi': E' \rightarrow M$ be smooth vector bundles over the same smooth manifold $M$, with or without boundary. A map $\mathcal{F}: \Gamma(E) \rightarrow \Gamma(E')$ is linear over $C^\infty(M)$ if and only if there exists a unique smooth bundle homomorphism $F: E \rightarrow E'$ over $M$ such that $\mathcal{F}(\sigma) = F \circ \sigma$ for all $ \sigma \in \Gamma(E)$.
\end{lemma}
\begin{proof}
  SORRY
\end{proof}

\begin{example}{Bundle Homomorphism over Manifolds}{Bundle Homomorphism over Manifolds}
  \begin{itemize}
    \item Let $M$ be a smooth manifold, and $f\in C^\infty(M)$. Then $\mathfrak{X}(M) \rightarrow \mathfrak{X}(M)$ defined by $X \mapsto fX$ is linear over $C^\infty(M)$, and thus defines a smooth bundle homomorphism $TM \rightarrow TM$ over $M$.
    \item If $Z$ is a smooth vector field in $\mathbb{R}^3$, then the cross product with $Z$: $X \mapsto X \times Z$ is linear over $C^\infty(\mathbb{R}^3)$, and thus defines a smooth bundle homomorphism $T\mathbb{R}^3 \rightarrow T\mathbb{R}^3$ over $\mathbb{R}^3$.
    \item If $Z$ is a smooth vector field on $\mathbb{R}^n$, then the Euclidean inner product with $Z$: $X \mapsto \langle X, Z \rangle$ is linear over $C^\infty(\mathbb{R}^n)$, and thus defines a smooth bundle homomorphism $T\mathbb{R}^n \rightarrow \mathbb{R} \times \mathbb{R}^n$ over $\mathbb{R}^n$.
  \end{itemize}  
\end{example}

Note that many maps the involve differentials are NOT bundle homomorphisms because they do not satisfy the right linearity conditions. For example, the Lie bracket $[X,Y]$ of two vector fields is not linear over $C^\infty(M)$ in either argument.

\section{Subbundles}

\begin{definition}{Subbundles}{Subbundles}
  Given a vector bundle $ \pi_E: E \rightarrow M$, a \textbf{subbundle} of $E$ is a vector bundle $ \pi_D: D \rightarrow M$, where $D$ is a topological subspace of $E$, and $\pi_D$ is the restriction of $ \pi_E$ to $D$, such that for each $p \in M$, the fiber $D_p = D \cap E_p$ is a vector subspace of $E_p$.

  If $E \rightarrow M$ is a smooth vector bundle, then a \textbf{smooth subbundle} of $E$ is a subbundle $D \rightarrow M$ such that $D$ is a smooth vector bundle and an embedded submanifold of $E$, with or without boundary.
\end{definition}

The following proposition gives a useful criterion for determining when a union of subspace $D = \bigsqcup_{p \in M} D_p, D_p \subseteq E_p$ is a smooth subbundle of $E$.

\begin{lemma}{Local Frame Criterion for Subbundles}{Local Frame Criterion for Subbundles}
  Let $ \pi: E \rightarrow M$ be a smooth vector bundle, and for each $p\in M$, we are given an $m$-dimensional linear subspace $D_p \subseteq E_p$. Let $D = \bigsqcup_{p \in M} D_p \subseteq E$, then $D$ is a smooth subbundle of $E$ if and only if: 

  Each point $p \in M$ has an open neighborhood $U \subseteq M$ that there exists a smooth local sectrions $\sigma_1, \ldots, \sigma_m: U \rightarrow E$ of $E$ over $U$ such that for each $q \in U$, the set $\{\sigma_1(q), \ldots, \sigma_m(q)\}$ is a basis of the subspace $D_q \subseteq E_q$.
\end{lemma}

\begin{example}{Subbundles}{Subbundles}
  \begin{itemize}
    \item If $M$ is a smooth manifold, and $V$ is a nowhere vanishing smooth vector field on $M$, then $D \subseteq TM$ defined by $D_p = \vspan \{V(p)\}$ is a smooth subbundle of $TM$ of rank 1.
    \item Suppose $E \rightarrow M$ is any trivial bundle, and let $E_1, \ldots ,E_k$ be a smooth global frame of $E$. For any $0 \leq m \leq k$, the subspace $D_p = \vspan \{E_1(p), \ldots, E_m(p)\} \subseteq E_p$ defines a smooth subbundle $D$ of $E$ of rank $m$.
    \item Suppose $M$ is a smooth manifold, with or without boundary, and $S \subseteq M$ is an immersed $k$-dimensional submanifold, with or without boundary. Then $TS$ is a smooth subbundle of the ambient tangent bundle $TM|_S$ of rank $k$.
  \end{itemize}
\end{example}

\begin{definition}{Constant Rank Bundle Homomorphisms}{Constant Rank Bundle Homomorphisms}
  Let $ \pi: E \rightarrow M$ and $ \pi': E' \rightarrow M$ be smooth vector bundles over the same smooth manifold $M$, with or without boundary. A smooth bundle homomorphism $F: E \rightarrow E'$ over $M$ is said to have \textbf{constant rank} if for each $p \in M$, the linear map $F_p: E_p \rightarrow E'_p$ defined by the restriction of $F$ to the fiber $E_p$ has the same rank.
\end{definition}

\begin{theorem}{Smooth Subbundles from Constant Rank}{Smooth Subbundles from Constant Rank}
  Let $E,E'$ be smooth vector bundles over the same smooth manifold $M$, and let $F: E \rightarrow E'$ be a smooth bundle homomorphism over $M$. Define subsets
  \begin{equation}
    \ker F = \bigsqcup_{p \in M} \ker F_p \subseteq E, \quad \image F = \bigsqcup_{p \in M} \image F_p \subseteq E'.
  \end{equation}
  Then both $\ker F$ and $\image F$ are smooth subbundles of $E$ and $E'$ respectively if and only if $F$ has constant rank.
\end{theorem}
\begin{proof}
  Quite obvious.
\end{proof}

\begin{lemma}{Orthogonal Complement Bundles}{Orthogonal Complement Bundles}
  Let $M$ be an immersed submanifold of $\mathbb{R}^n$, with or without boundary. Let $D$ be a smooth rank-$k$ subbundle of $T \mathbb{R}^n|_M$. For each $p \in M$, define $D_p^\perp \subseteq T_p \mathbb{R}^n$ to be the orthogonal complement of $D_p$ in $T_p \mathbb{R}^n \cong \mathbb{R}^n$ with respect to the standard Euclidean inner product. Then $D^\perp = \bigsqcup_{p \in M} D_p^\perp$ is a smooth subbundle of $T \mathbb{R}^n|_M$ of rank $n - k$.

  If we take $D = TM$ to be the tangent bundle of $M$, then $D^\perp$ is called the \textbf{normal bundle} of $M$ in $\mathbb{R}^n$.
\end{lemma}

\section{Fiber Bundles}
A vector bundle is a special case of a more general concept called a fiber bundle, where the fibers are not necessarily vector spaces.

\begin{definition}{Fiber Bundles}{Fiber Bundles}
  A \textbf{fiber bundle} is a triple $(E, M, \pi)$, where $E$ and $M$ are topological spaces (or smooth manifolds, with or without boundary), and $\pi: E \rightarrow M$ is a continuous surjection (or smooth map) satisfying the following conditions:
  \begin{itemize}
    \item For each $p \in M$, the fiber $E_p = \pi^{-1}(p)$ is homeomorphic (or diffeomorphic) to a fixed topological space (or smooth manifold) $F$, called the \textbf{fiber}.
    \item For each $p \in M$, there exists an open neighborhood $U$ of $p$ in $M$ and a homeomorphism (or diffeomorphism) $\Phi: \pi^{-1}(U) \rightarrow U \times F$ called a \textbf{local trivialization} such that $\pi_U \circ \Phi = \pi$, where $\pi_U: U \times F \rightarrow U$ is the projection onto the first factor.
  \end{itemize}
\end{definition}

\begin{example}{Fiber Bundles}{Fiber Bundles}
  \begin{itemize}
    \item Every product space $M \times F$ with the projection map $\pi: M \times F \rightarrow M$ defined by $\pi(p, x) = p$ is a trivial fiber bundle over $M$ with fiber $F$.
    \item Every rank-$k$ vector bundle is a fiber bundle with fiber $\mathbb{R}^k$.
    \item The M\"obius strip is a fiber bundle over $S^1$ with fiber $[-1,1]$. It is not a trivial fiber bundle.
    \item Every covering map $\pi: E \rightarrow M$ is a fiber bundle with discrete fiber.
  \end{itemize}
\end{example}

\end{document}
