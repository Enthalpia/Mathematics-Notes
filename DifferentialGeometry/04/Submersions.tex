\documentclass[../main.tex]{subfiles}

\begin{document}
\chapter{Submersions, Immersions, and Embeddings}
We shall study the geometric properties of smooth maps by there differential.

\section{Maps of Constant Rank}
Suppose $M$ and $N$ are smooth manifolds, with or without boundary, and $F : M \to N$ is a smooth map. For each point $p \in M$, we define the \textbf{rank} of $F$ at $p$ to be the rank of the linear map $\mathrm{d} F_p : T_pM \to T_{F(p)}N$, which is just the rank of the Jacobian matrix of $F$ in any coordinate chart containing $p$ and $F(p)$, or just $\dim \range \mathrm{d} F_p$. If the rank of $F$ is the same at every point of $M$, we say that $F$ is a \textbf{map of constant rank}.

The maximum possible rank of $F$ at any point is $\min \{\dim M, \dim N\}$. If the rank of $F$ at $p$ is equal to this maximum value, we say that $F$ has \textbf{full rank} at $p$. If $F$ has full rank at every point of $M$, we say that $F$ has \textbf{constant full rank}.

\begin{definition}{smooth Submersions and Immersions}{smooth Submersions and Immersions}
  A smooth map $F : M \to N$ between smooth manifolds is a \textbf{smooth submersion} if $\mathrm{d} F_p : T_pM \to T_{F(p)}N$ is surjective for every point $p \in M$.

  A smooth map $F : M \to N$ between smooth manifolds is a \textbf{smooth immersion} if $\mathrm{d} F_p : T_pM \to T_{F(p)}N$ is injective for every point $p \in M$.
\end{definition}

From the continuity of $\mathrm{d} F$, we have the following result.
\begin{proposition}{Local Surjectivity and Local Injectivity}{Local Surjectivity of Submersions and Local Injectivity of Immersions}
  If $F : M \to N$ is a smooth map and $p \in M$. If $\mathrm{d} F_p$ is surjective, then there exists an open neighborhood $U$ of $p$ such that $F|_U : U \to N$ is a submersion. If $\mathrm{d} F_p$ is injective, then there exists an open neighborhood $U$ of $p$ such that $F|_U : U \to N$ is an immersion.
\end{proposition}
\begin{proof}
  Choose any smooth coordinate chart $(U, \varphi)$ on $M$ containing $p$ and any smooth coordinate chart $(V, \psi)$ on $N$ containing $F(p)$. Then the Jacobian matrix of $F$ has full rank at $p$. As $\mathrm{d} F$ is continuous, there exists an open neighborhood $U' \subset U$ of $p$ such that the Jacobian matrix of $F$ has full rank at every point of $U'$.
\end{proof}

\begin{example}{Submersion and Immersion}{Submersion and Immersion}
  \begin{itemize}
    \item Suppose $M_1, \ldots ,M_k$ are smooth manifolds, then the projection map
      \begin{equation*}
        \pi_i : M_1 \times \cdots \times M_k \rightarrow M_i, \quad (p_1, \ldots , p_k) \mapsto p_i
      \end{equation*}
      is a smooth submersion for each $1 \leq i \leq k$.
    \item If $ \gamma :J \to M$ is a smooth curve on a smooth manifold $M$, with or without boundary, then $\gamma$ is a smooth immersion if and only if $\gamma'(t) \neq 0$ for all $t \in J$.
    \item If $M$ is a smooth manifold then the tangent bundle projection $\pi : TM \to M$ is a smooth submersion.
  \end{itemize}
\end{example}

\begin{proposition}{Properties of Submersions and Immersions}{Properties of Submersions and Immersions}
  \begin{itemize}
    \item If $F : M \to N$ and $G : N \to P$ are smooth submersions, then $G \circ F : M \to P$ is a smooth submersion.
    \item If $F : M \to N$ and $G : N \to P$ are smooth immersions, then $G \circ F : M \to P$ is a smooth immersion.
    \item The composition of maps of constant rank need not have constant rank.
  \end{itemize}    
\end{proposition}
\begin{proof}
  For the third claim, take
  \begin{equation*}
    f: \mathbb{R}\rightarrow \mathbb{R}^2, \quad t \mapsto (t, t^2)
  \end{equation*}
  and
  \begin{equation*}
    g: \mathbb{R}^2 \rightarrow \mathbb{R}, \quad (x, y) \mapsto y.
  \end{equation*}
  Then both $f$ and $g$ have constant rank 1, but the composition
  \begin{equation*}
    g \circ f : \mathbb{R} \rightarrow \mathbb{R}, \quad t \mapsto t^2
  \end{equation*}
  does not have constant rank.
\end{proof}

\subsection{Local Diffeomorphisms}
If $M,N$ are smooth manifolds with or without boundary, a smooth map $F : M \to N$ is a \textbf{local diffeomorphism} if for each point $p \in M$, there exists an open neighborhood $U$ of $p$ such that $F(U)$ is open in $N$ and $F|_U : U \to F(U)$ is a diffeomorphism.

\begin{theorem}{The Inverse Function Theorem}{The Inverse Function Theorem}
  Suppose $M$ and $N$ are smooth manifolds, and $F : M \to N$ is a smooth map. If $p\in M$ and $\mathrm{d} F_p : T_pM \to T_{F(p)}N$ is invertible, then there exist connected open neighborhoods $U_0$ of $p$ in $M$ and $V_0$ of $F(p)$ in $N$ such that $F|_{U_0} : U_0 \to V_0$ is a diffeomorphism.
\end{theorem}
\begin{proof}
  Firstly, this implies $M,N$ have the same dimension $n$, then choose smooth charts $(U, \varphi)$ centering at $p$ and $(V, \psi)$ centering at $F(p)$ with $F(U) \subseteq V$. Then $\hat{F} = \psi \circ F \circ \varphi^{-1} : \varphi(U) \to \psi(V)$ is a smooth map between open subsets of $\mathbb{R}^n$ with invertible Jacobian matrix at $\varphi(p)$. By the Euclidean version of the Inverse Function Theorem, there exist open neighborhoods $U'$ of $\varphi(p)$ and $V'$ of $\psi(F(p))$ such that $\hat{F}|_{U'} : U' \to V'$ is a diffeomorphism. Let $U_0 = \varphi^{-1}(U')$ and $V_0 = \psi^{-1}(V')$, then $F|_{U_0} : U_0 \to V_0$ is a diffeomorphism.
\end{proof}

\begin{remark}
  NOTE that this is true only for manifolds without boundary.
\end{remark}

\begin{proposition}{Properties of Local Diffeomorphisms}{Properties of Local Diffeomorphisms}
  \begin{itemize}
    \item Compositions of local diffeomorphisms are local diffeomorphisms.
    \item Finite products of local diffeomorphisms are local diffeomorphisms.
    \item The restriction of a local diffeomorphism to an open submanifold (with or without boundary) is a local diffeomorphism.
    \item Every diffeomorphism is a local diffeomorphism. Every bijective local diffeomorphism is a diffeomorphism.
    \item A map between smooth manifolds, with or without boundary, is a local diffeomorphism if and only if it has a local diffeomorphism coordinate representation at each point.
  \end{itemize}
\end{proposition}

\begin{proposition}{Local Diffeomorphisms, Submersions, and Immersions}{Local Diffeomorphisms, Submersions, and Immersions}
  A smooth map between smooth manifolds (without boundary), is a local diffeomorphism if and only if it is both a smooth immersion and a smooth submersion.

  Moreover, if $\dim M = \dim N$, then a smooth map $F : M \to N$ is a smooth submersion if and only if it is a smooth immersion, and in either case it is a local diffeomorphism.
\end{proposition}
  
\begin{example}{Local Diffeomorphisms}{Local Diffeomorphisms}
  The map $ \mathbb{R} \rightarrow S^1$ defined by $t \mapsto (\cos t, \sin t)$ is a local diffeomorphism, but not a diffeomorphism.
\end{example}

\subsection{The Rank Theorem}

\begin{theorem}{The Rank Theorem}{The Rank Theorem}
  Suppose $M$ and $N$ are smooth manifolds of dimensions $m$ and $n$, respectively, and $F : M \to N$ is a smooth map of constant rank $k$. Then for each point $p \in M$, there exist smooth coordinate charts $(U, \varphi)$ on $M$ centered at $p$ and $(V, \psi)$ on $N$ centered at $F(p)$ such that
  \begin{equation*}
    \hat{F} = \psi \circ F \circ \varphi^{-1} : \varphi(U) \to \psi(V)
  \end{equation*}
  is given by
  \begin{equation*}
    \hat{F}(x^1, \ldots , x^m) = (x^1, \ldots , x^k, 0, \ldots , 0).
  \end{equation*}  
\end{theorem}

The linear version of the Rank Theorem is that under certain choice of basis, any linear map can be represented by a matrix of the form
\begin{equation*}
  \begin{pmatrix}
    I_k & 0 \\
    0 & 0
  \end{pmatrix},
\end{equation*}

\begin{proof}
  From locality, just replace $M,N$ by open subsets $U \subseteq \mathbb{R}^m$ and $V \subseteq \mathbb{R}^n$ containing $p$ and $F(p)$, respectively. We also assume $p = 0$ and $F(p) = 0$.
  %-----------COME BACK LATER----------------
  
  SORRY
\end{proof}

The following corollary is an immediate consequence and also can be viewed as a restatement of the Rank Theorem.
\begin{corollary}{Local Linearity}{Local Linearity}
  Let $M$ and $N$ be smooth manifolds of dimensions $m$ and $n$, respectively, and let $F : M \to N$ be a smooth map. If $M$ is connected, then $F$ has constant rank $k$ if and only if for each point $p \in M$, there exist smooth coordinate charts $(U, \varphi)$ on $M$ centered at $p$ and $(V, \psi)$ on $N$ centered at $F(p)$ such that the coordinate representation
  \begin{equation*}
    \hat{F} = \psi \circ F \circ \varphi^{-1} : \varphi(U) \to \psi(V)
  \end{equation*}
  is linear.
\end{corollary}

\begin{theorem}{Global Rank Theorem}{Global Rank Theorem}
  Let $M$ and $N$ be smooth manifolds and let $F : M \to N$ be a smooth map of constant rank $k$. Then
  \begin{itemize}
    \item If $F$ is surjective, then it is a smooth submersion.
    \item If $F$ is injective, then it is a smooth immersion.
    \item If $F$ is bijective, then it is a diffeomorphism.
  \end{itemize}
\end{theorem}
\begin{proof}
  SORRY
\end{proof}

\subsection{The Rank Theorem with Boundary}
The rank theorem does not generalize to manifolds with boundary in full generality. However, we do have the following partial result.

\begin{theorem}{The Local Immersion Theorem with Boundary}{The Local Immersion Theorem with Boundary}
  Suppose $M$ is a smooth manifold with boundary of dimension $m$, $N$ is a smooth manifold of dimension $n$, and $F : M \to N$ is a smooth immersion. Then for each point $p \in \partial M$, there exist smooth boundary charts $(U, \varphi)$ on $M$ centered at $p$ and smooth chart $(V, \psi)$ on $N$ centered at $F(p)$ such that $F(U) \subseteq V$ and the coordinate representation
  \begin{equation*}
    \hat{F}(x^1, \ldots , x^m) = (x^1, \ldots , x^m, 0, \ldots , 0).
  \end{equation*}
\end{theorem}
\begin{proof}
  SORRY
\end{proof}

\section{Embeddings}

\begin{definition}{Smooth Embeddings}{Smooth Embeddings}
  Let $M$ and $N$ be smooth manifolds, with or without boundary. A \textbf{smooth embedding} is a smooth immersion $F : M \to N$ that is also a topological embedding; that is, $F$ is a homeomorphism onto its image $F(M)$, where $F(M)$ is given the subspace topology inherited from $N$.
\end{definition}

\begin{proposition}{Compositions of Embeddings}{Compositions of Embeddings}
  If $F : M \to N$ and $G : N \to P$ are smooth embeddings, then $G \circ F : M \to P$ is a smooth embedding.
\end{proposition}

\begin{example}{Smooth Embeddings}{Smooth Embeddings}
  \begin{itemize}
    \item Let $M$ be a smooth manifold with or without boundary and $U \subseteq M$ be an open submanifold. Then the inclusion map $\iota : U \hookrightarrow M$ is a smooth embedding.
    \item If $M_1, \ldots ,M_k$ are smooth manifolds and $p_i \in M_i$ for each $1 \leq i \leq k$, then each of
      \begin{equation*}
        \iota_i : M_i \hookrightarrow M_1 \times \cdots \times M_k, \quad q \mapsto (p_1, \ldots , p_{i-1}, q, p_{i+1}, \ldots , p_k)
      \end{equation*}
      is a smooth embedding. Indeed, $\mathbb{R}^n \iota \mathbb{R}^{n+k}$ defined by $x \mapsto (x,0)$ is a smooth embedding.
  \end{itemize}
\end{example}

Here are some counterexamples that illustrate the definition.
\begin{itemize}
  \item The map $ \mathbb{R} \to \mathbb{R}^2$ defined by $t \mapsto (t^3,0)$ is a smooth map and a topological embedding, but not a smooth immersion at $t = 0$, so it is not a smooth embedding.
  \item The figure-eight curve $ \mathbb{R} \to \mathbb{R}^2$ defined by $t \mapsto (\sin t, \sin 2t)$ is a smooth immersion, but not a topological embedding because it is not a homeomorphism onto its image. (Compactness fails)
  \item Now is an interesting example that shows mere injectivity is not enough for a smooth embedding. Consider $ \gamma: \mathbb{R} \rightarrow \mathbb{T}^2$ by
    \begin{equation*}
      \gamma(t) = (e^{it}, e^{i \alpha t}),
    \end{equation*}
    where $\alpha$ is an irrational number. Then $\gamma$ is a smooth immersion, injective (because $\alpha$ is irrational), but not a smooth embedding because it is not a homeomorphism onto its image. Because $ \gamma(\mathbb{Z}) $ is dense in $\mathbb{T}^2$, but $\mathbb{Z}$ is not dense in $\mathbb{R}$.
\end{itemize}

\begin{proposition}{Criterion for Smooth Embeddings}{Criterion for Smooth Embeddings}
  Suppose $M$ and $N$ are smooth manifolds, with or without boundary, and $F : M \to N$ is an injective smooth immersion. Then $F$ is a smooth embedding if one of the following equivalent conditions holds:
  \begin{itemize}
    \item $F$ is an open or closed map.
    \item $F$ is a proper map, that is, for every compact subset $K \subseteq M$, the image $F(K)$ is compact in $N$.
    \item $M$ is compact.
    \item $M$ has empty boundary and $\dim M= \dim N$.
  \end{itemize}
\end{proposition}
\begin{proof}
  The first claim is just the definition of a topological embedding. The second claim follows from the fact that a proper map is a closed map, and the third claim follows from the fact that a compact set is closed. The fourth claim follows from the fact that if $M$ has empty boundary, then it is an open submanifold of itself, and if $\dim M = \dim N$, then $F$ is a local diffeomorphism, hence an open map.
\end{proof}

\begin{theorem}{Local Embedding Theorem}{Local Embedding Theorem}
  Suppose $M$ and $N$ are smooth manifolds, with or without boundary, and $F : M \to N$ is a smooth map. Then $F$ is a smooth immersion iff every point $p \in M$ has a neighborhood $U$ such that $F|_U : U \to N$ is a smooth embedding.
\end{theorem}
\begin{proof}
  If $F$ is a local smooth embedding on every point, then it has full rank everywhere, hence it is a smooth immersion. Conversely, if $F$ is a smooth immersion, then for each point $p \in M$:

  If $F(p)\notin \partial N$, then by the Rank Theorem, there exist coordinate charts $(U, \varphi)$ on $M$ centered at $p$ and $(V, \psi)$ on $N$ centered at $F(p)$ such that the coordinate representation
  \begin{equation*}
    \hat{F}(x^1, \ldots , x^m) = (x^1, \ldots , x^m, 0, \ldots , 0).
  \end{equation*}
  Then restricting $U$ if necessary, we have that $F|_U : U \to N$ is injective. If $F(p) \in \partial N$, then the same argument applies slight adjustment by $\mathbb{H}^n$ coordinate charts.

  Next, let $U_1$ be a precompact neighborhood of $p$ in $U$ such that $F|_{\overline{U_1}} $ is injective and has compact domain. Then $F|_{\overline{U_1}} : \overline{U_1} \to N$ is a topological embedding.
\end{proof}

\begin{remark}
  This gives a direct notion for topological immersions: a continuous map between topological spaces that is a local topological embedding at every point.
\end{remark}

\section{Submersions}

\begin{definition}{Section}{Section}
  Let $M$ and $N$ be topological spaces, and let $\pi : M \to N$ be a continuous map. A \textbf{section} of $\pi$ is a continuous map $ \sigma: N \to M$ such that $\pi \circ \sigma = \mathrm{id}_N$.

  A local section of $\pi$ over an open set $U \subseteq N$ is a continuous map $ \sigma: U \to M$ such that $\pi \circ \sigma = \mathrm{id}_U$.
\end{definition}

Really, sections are right inverses of the map $\pi$, so it is injective.

Note that global sections need not exist, for example, consider the $S^1$ projection onto $RP^1$. However, local sections always exist for submersions, as the following result shows.

\begin{theorem}{Local Section Theorem}{Local Section Theorem}
  Let $M$ and $N$ be smooth manifolds, and $\pi : M \to N$ be a smooth map. Then $\pi$ is a smooth submersion if and only if for each point $p \in M$, it is in an image of a smooth local section of $\pi$.
\end{theorem}
\begin{proof}
  Suppose $ \pi$ is a smooth submersion, and given $p \in M$. By the Rank Theorem, there exist smooth coordinate charts $(U, \varphi)$ on $M$ centered at $p$ and $(V, \psi)$ on $N$ centered at $\pi(p)$ such that the coordinate representation $\pi(x^1, \ldots , x^m) = (x^1, \ldots , x^n)$. For sufficiently small $ \epsilon$, the coordinate cube
  \begin{equation*}
    C_{ \epsilon} = \{(x^1, \ldots , x^n) \in \mathbb{R}^n : |x^i| < \epsilon, 1 \leq i \leq m\}
  \end{equation*}
  Then $ \pi(C_{ \epsilon}) $ is also a coordinate cube in $\mathbb{R}^n$. The coordinate map by
  \begin{equation*}
    \sigma(x^1, \ldots , x^n) = (x^1, \ldots , x^n, 0, \ldots , 0)
  \end{equation*}
  is a smooth local section.

  Conversely, from $ \pi \circ \sigma = \mathrm{id}_N$, we have that $\mathrm{d} \pi_{\sigma(q)} \circ \mathrm{d} \sigma_q = \mathrm{id}_{T_qN}$ for each $q \in N$. Hence, $\mathrm{d} \pi_{\sigma(q)}$ is surjective for each $q \in N$, so $\pi$ is a smooth submersion.
\end{proof}

\begin{remark}
  This theorem motivates the definition of a topological submersion: a continuous map between topological spaces such that every point in the domain is in the image of a local section.
\end{remark}

\begin{proposition}{Properties of Smooth Submersions}{Properties of Smooth Submersions}
  Let $M$ and $N$ be smooth manifolds, and let $\pi : M \to N$ be a smooth submersion. Then 
  \begin{itemize}
    \item $\pi$ is an open map.
    \item If $\pi$ is surjective, then it is a quotient map.
  \end{itemize}
\end{proposition}
\begin{proof}
  For the first claim, let $W \subseteq M$ be open and $q\in \pi(W)$, take any $p \in W$ such that $\pi(p) = q$. By the Local Section Theorem, there exists a smooth local section $ \sigma: U \to M$ of $\pi$ such that $ \sigma(q) = p$. Then $ \sigma^{-1}(W)$ is an open neighborhood of $q$ contained in $\pi(W)$, so $\pi(W)$ is open in $N$.

  The second claim follows from the first claim and the definition of a quotient map.
\end{proof}

We can see that smooth submersions plays a similar role to that of quotient maps in topology.

\begin{theorem}{Passing Smoothly to Quotient}{Passing Smoothly to Quotient}
  Suppose $M$ and $N$ are smooth manifolds, and $\pi : M \to N$ is a surjective smooth submersion. If $P$ is a smooth manifold, with or without boundary, then a map $F : N \to P$ is smooth if and only if $F \circ \pi : M \to P$ is smooth.

  Moreover, if $G : M \to P$ is a smooth map that is constant on each fiber of $\pi$, then there exists a unique smooth map $\tilde{G} : N \to P$ such that $G = \tilde{G} \circ \pi$.
\end{theorem}
\begin{proof}
  If $F : N \to P$ is smooth, then $F \circ \pi : M \to P$ is smooth by composition of smooth maps. Conversely, suppose that $F \circ \pi : M \to P$ is smooth. Given any point $q \in N$, take any $p \in M$ such that $\pi(p) = q$. By the Local Section Theorem, there exists a smooth local section $ \sigma: U \to M$ of $\pi$ such that $ \sigma(q) = p$. Then the restriction $F|_U = (F \circ \pi) \circ \sigma : U \to P$ is smooth. Since $q$ was arbitrary, $F$ is smooth.
\end{proof}

\begin{theorem}{Uniqueness of Smooth Quotients}{Uniqueness of Smooth Quotients}
  Suppose $M$ and $N_1, N_2$ are smooth manifolds, and $\pi_1 : M \to N_1$ and $\pi_2 : M \to N_2$ are surjective smooth submersions that are constant on each other's fibers. Then there exists a unique diffeomorphism $F : N_1 \to N_2$ such that $F \circ \pi_1 = \pi_2$.
\end{theorem}

\section{Smooth Covering Maps}
In general topology, a \textbf{covering map} is a continuous surjective map $ \pi: X \to Y$ such that for each point $y \in Y$, there exists an open neighborhood $U$ of $y$ such that $\pi^{-1}(U)$ is a disjoint union of open sets in $X$, each of which is homeomorphic to $U$ via $\pi$.

In the context of smooth manifolds, we have the following definition.
\begin{definition}{Smooth Covering Map}{Smooth Covering Map}
  Let $E$ and $M$ be connected smooth manifolds, with or without boundary. A \textbf{smooth covering map} is a smooth surjective map $ \pi: E \to M$ such that for each point $p \in M$, there exists an open neighborhood $U$ of $p$ such that $\pi^{-1}(U)$ is a disjoint union of open sets in $E$, each of which is diffeomorphic to $U$ via $\pi$.

  We say that $M$ is the \textbf{base space}, $E$ is the \textbf{covering space}, and if $E$ is simply connected, we say that $E$ is the \textbf{universal covering space} of $M$.
\end{definition}

\begin{proposition}{Properties of Smooth Covering Maps}{Properties of Smooth Covering Maps}
  Let $E$ and $M$ be connected smooth manifolds, with or without boundary, and let $ \pi: E \to M$ be a smooth covering map. Then 
  \begin{itemize}
    \item $ \pi$ is a local diffeomorphism, a smooth submersion, an open map and a quotient map.
    \item An injective smooth covering map is a diffeomorphism.
    \item A topological covering map between smooth manifolds is a smooth covering map if and only if it is a local diffeomorphism.
    \item Every local section of $ \pi$ is a smooth local section.
  \end{itemize}  
\end{proposition}

\begin{example}{Smooth Covering Maps}{Smooth Covering Maps}
  \begin{itemize}
    \item The map $ \mathbb{R} \to S^1$ defined by $t \mapsto (\cos t, \sin t)$ is a smooth covering map. Its universal covering space is $\mathbb{R}$ itself.
    \item For $n \geq 1$, the map $q: S^n \to \mathbb{R}P^n$ defined by $q(x) = [x]$ is a two-to-one smooth covering map.
  \end{itemize}
\end{example}

For smooth covering maps, we also have the local section theorem strengthened.
\begin{theorem}{Local Section Theorem for Smooth Covering Maps}{Local Section Theorem for Smooth Covering Maps}
  Let $E$ and $M$ be connected smooth manifolds, with or without boundary, and let $ \pi: E \to M$ be a smooth covering map. Then for each evenly covered open set $U \subseteq M$, a point $q \in U$, and $p\in \pi^{-1}(q)$, there exists a unique smooth local section $ \sigma: U \to E$ of $\pi$ such that $\sigma(q) = p$.
\end{theorem}
\begin{proof}
  This is quite obvious from the definition of smooth covering maps.
\end{proof}

\begin{proposition}{Products of Covering Maps}{Products of Covering Maps}
  Let $E_1, \ldots , E_k$ and $M_1, \ldots , M_k$ be smooth manifolds without boundary, and let $ \pi_i: E_i \to M_i$ be smooth covering maps for each $1 \leq i \leq k$. Then the product map
  \begin{equation*}
    \pi_1 \times \cdots \times \pi_k : E_1 \times \cdots \times E_k \to M_1 \times \cdots \times M_k
  \end{equation*}
  is a smooth covering map.
\end{proposition}

\begin{theorem}{Covering Space of Smooth manifolds}{Covering Space of Smooth manifolds}
  Suppose $M$ is a connected smooth $n$-manifold, and $ \pi: E \to M$ is a topological covering map, then $E$ is a topological $n$-manifold and has a unique smooth structure such that $ \pi: E \to M$ is a smooth covering map.

  Moreover, if $M$ is smooth manifold with boundary, then $E$ is a topological manifold with boundary such that $\partial E = \pi^{-1}(\partial M)$, and has a unique smooth structure with boundary such that $ \pi: E \to M$ is a smooth covering map.
\end{theorem}
\begin{proof}
  SORRY.
\end{proof}

\begin{corollary}{Existence of Universal Covering Manifold}{Existence of Universal Covering Manifold}
  If $M$ is a connected smooth manifold, there exists a simply connected smooth manifold $\tilde{M}$ called the \textbf{universal covering manifold} of $M$, and a smooth covering map $ \pi: \tilde{M} \to M$. The universal covering manifold is unique up to diffeomorphism. That is, if $\tilde{M}'$ is another simply connected smooth manifold with a smooth covering map $ \pi': \tilde{M}' \to M$, then there exists a diffeomorphism $F : \tilde{M} \to \tilde{M}'$ such that $ \pi' \circ F = \pi$.
\end{corollary}

There are not many simple criterion for a smooth map to be a smooth covering map, but we do have the following sufficient (not necessary) condition.

\begin{proposition}{Local Proper Diffeomorphism is a Smooth Covering Map}{Local Proper Diffeomorphism is a Smooth Covering Map}
  Let $E$ and $M$ be connected smooth manifolds, with or without boundary, and let $ \pi: E \to M$ be a smooth map. If $ \pi$ is a local diffeomorphism and a proper map, then it is a smooth covering map.  
\end{proposition}

\end{document}
