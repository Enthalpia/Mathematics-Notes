\documentclass[../main.tex]{subfiles}

\begin{document}
\chapter{Tangent Vectors}

The basic idea of calculus is linear approximation.

In analysis, we come across the idea of geometric tangent vectors in $\mathbb{R}^n$, which are used for ``directional derivatives'' of multivariable functions. We shall follow this path initially, and then move to a more abstract definition of tangent vectors as derivations.

\section{Tangent Vectors}
Take $S^{n-1} \subseteq \mathbb{R}^n$ for example. For a point $x \in S^{n-1}$, usually we think it as a location, expressed by coordinates $(x^1, x^2, \ldots, x^n)$. But when doing calculus, we sometimes need to think of it as a vector. Geometrically, we can think of a vector as an arrow which has arbitrary start point. We imagine tangent vectors as arrows starting from the point $x$. That is to say, they live in a copy of $\mathbb{R}^n$ that is ``attached'' to the point $x$.

\subsection{Geometric Tangent Vectors}
Given $a\in \mathbb{R}^n$, define the geometric tangent space to $\mathbb{R}^n$ at $a$ as the vector space
\begin{equation}
	\mathbb{R}^n_a = \left\{ a \right\} \times \mathbb{R}^n = \left\{ (a, v) : v \in \mathbb{R}^n \right\}, \quad (a,v) + (a,w) = (a, v+w), \quad c(a,v) = (a, cv).
\end{equation}
A geometric tangent vector in $\mathbb{R}^n$ is an element of $\mathbb{R}^n_a$ for some $a \in \mathbb{R}^n$. We shall denote $v_a = (a,v)$

From this perspective, we can think of the tangent space of $S^{n-1}$ at $a \in S^{n-1}$ as a subspace of $\mathbb{R}^n_a$: As all vectors in $\mathbb{R}^n_a$ that are perpendicular to the radius vector from the origin to $a$. To do this, we must have an inner product inherited from $\mathbb{R}^n$ via the natural isomorphism between $\mathbb{R}^n_a$ and $\mathbb{R}^n$.

This cannot be generalized to arbitrary manifolds, since there is no ambient Euclidean space to provide such a notion of perpendicularity. We shall use smooth structures to define tangent vectors in a more abstract way.

We turn to directional derivatives to motivate our definition. Every geometric tangent vector $v_a \in \mathbb{R}^n_a$ defines a map
\begin{equation}
	D_v|_a : C^\infty(\mathbb{R}^n) \to \mathbb{R}, \quad D_v|_a f = D_v f(a) = \frac{\mathrm{d} }{\mathrm{d} t} \Big|_{t=0} f(a + tv).
\end{equation}
This map is linear and satisfies the Leibniz product rule:
\begin{equation*}
	D_v|_a (fg) = f(a) D_v|_a g + g(a) D_v|_a f.
\end{equation*}
If $v_a = v^i e_{i,a}$ in the standard basis, then we have
\begin{equation*}
	D_v|_a (f) = v^i \frac{\partial f}{\partial x^i} (a).
\end{equation*}

We now reverse the process.
\begin{definition}{Derivation}{Derivation}
	A \textbf{derivation} at $a \in \mathbb{R}^n$ is a linear map
	\begin{equation*}
		w : C^\infty(\mathbb{R}^n) \to \mathbb{R}
	\end{equation*}
	that satisfies the Leibniz product rule:
	\begin{equation*}
		w(fg) = f(a) w(g) + g(a) w(f)
	\end{equation*}
	for all $f,g \in C^\infty(\mathbb{R}^n)$. The set of all derivations at $a$ is denoted by $T_a \mathbb{R}^n$. Then $T_a \mathbb{R}^n$ is a vector space under the operations
	\begin{equation*}
		(w_1 + w_2)(f) = w_1(f) + w_2(f), \quad (cw)(f) = c w(f).
	\end{equation*}
\end{definition}

It is fairly surprising that $T_a \mathbb{R}^n$ is isomorphic to the geometric tangent space $\mathbb{R}^n_a$.

\begin{lemma}{Property of Derivations}{PropertyOfDerivations}
	Suppose $a\in \mathbb{R}^n$ and $w\in T_a \mathbb{R}^n$, $f,g\in C^{\infty }(\mathbb{R}^n)$.
	\begin{itemize}
		\item If $f$ is constant, then $w(f) = 0$.
		\item If $f(a) = g(a) = 0$, then $w(fg) = 0$.
	\end{itemize}
\end{lemma}
\begin{proof}
	If $f(x)=1$, then $wf = w(ff) = f(a)wf + f(a)wf = 2wf$, so $wf=0$. If $f(x)=c$, then $wf = w(cf_1) = c w(f_1) = 0$.
\end{proof}

\begin{proposition}{The Structure of $T_a \mathbb{R}^n$}{The Structure of Ta mathbbRn}
	Let $a \in \mathbb{R}^n$. Then
	\begin{itemize}
		\item For each geometric tangent vector $v_a \in \mathbb{R}^n_a$, the map $D_v|_a$ defined above is a derivation at $a$.
		\item The map $v_a \mapsto D_v|_a$ is a vector space isomorphism from $\mathbb{R}^n_a$ to $T_a \mathbb{R}^n$.
	\end{itemize}
\end{proposition}
\begin{proof}
	To prove isomorphism:
	\begin{itemize}
		\item Linearity: we have
		      \begin{equation*}
			      D_{c_1v+c_2w}|_a f = (c_1 v + c_2 w)^i \frac{\partial f}{\partial x^i} (a) = c_1 v^i \frac{\partial f}{\partial x^i} (a) + c_2 w^i \frac{\partial f}{\partial x^i} (a) = c_1 D_v|_a f + c_2 D_w|_a f.
		      \end{equation*}
		\item Injectivity: if $D_v|_a = 0$, then for all $f \in C^\infty(\mathbb{R}^n)$, $\displaystyle D_v|_a f = v^i \frac{\partial f}{\partial x^i} (a) = 0$. Taking $f(x) = x^j$, we have $v^j = 0$ for all $j$, so $v=0$.
		\item Surjectivity: let $w \in T_a \mathbb{R}^n$. Define $v^i = w(x^i)$, and let $v_a = v^ie_i|_a$. For any $f \in C^\infty(\mathbb{R}^n)$, by Taylor's theorem, we have
		      \begin{equation*}
			      f(x) = f(a) + \sum_{i=1}^n \frac{\partial f}{\partial x^i} (a) (x^i - a^i) + R(x),
		      \end{equation*}
		      \begin{equation*}
			      R(x) = \sum_{i,j=1}^n (x^i - a^i)(x^j - a^j) \int_0^1 (1-t) \frac{\partial^2 f}{\partial x^i \partial x^j} (a + t(x - a)) \, \mathrm{d} t.
		      \end{equation*}
		      As $R(x)$ is the sum of products of functions vanishing at $a$, by the previous lemma we have $w(R) = 0$. Thus,
		      \begin{equation*}
			      w(f) = w\left( f(a) + \sum_{i=1}^n \frac{\partial f}{\partial x^i} (a) (x^i - a^i) \right) = \sum_{i=1}^n \frac{\partial f}{\partial x^i} (a) w(x^i) = \sum_{i=1}^n \frac{\partial f}{\partial x^i} (a) v^i = D_v|_a f.
		      \end{equation*}
	\end{itemize}
\end{proof}
We have thus established the equivalence. And this definition can be generalized to arbitrary smooth manifolds.

\subsection{Tangent Vectors on Manifolds}
\begin{definition}{Tangent Vectors on Manifolds}{Tangent Vectors on Manifolds}
	Let $M$ be a smooth manifold, with or without boundary, and let $p \in M$. A linear map $v: C^\infty(M) \to \mathbb{R}$ is called a \textbf{derivation at $p$} if it satisfies the Leibniz product rule:
	\begin{equation*}
		v(fg) = f(p) v(g) + g(p) v(f), \quad \forall f,g \in C^\infty(M).
	\end{equation*}
	The set of all derivations at $p$ is denoted by $T_p M$ and called the \textbf{tangent space} of $M$ at $p$. Its elements are called \textbf{tangent vectors} to $M$ at $p$.
\end{definition}

\begin{proposition}{Property of Tangent Vectors on Manifolds}{Property of Tangent Vectors on Manifolds}
	Let $M$ be a smooth manifold, with or without boundary, and let $p \in M$. If $v \in T_p M$ and $f,g \in C^\infty(M)$, then
	\begin{itemize}
		\item If $f$ is constant, then $v(f) = 0$.
		\item If $f(p) = g(p) = 0$, then $v(fg) = 0$.
	\end{itemize}
\end{proposition}

\section{The Differential of a Smooth Map}
We talk about the differential in analysis as linear approximations of functions at a given point. In the manifold case, there makes no sense to talk about linear transformations between manifolds, so we do it in terms of tangent spaces.

\begin{definition}{Differential on Manifolds}{Differential on Manifolds}
	If $M,N$ are smooth manifolds, with or without boundary, and $F: M \to N$ is a smooth map, then for each $p \in M$, we define a map
	\begin{equation}
		\mathrm{d} F_p : T_p M \to T_{F(p)} N
	\end{equation}
	to be the \textbf{differential} of $F$ at $p$, defined by
	\begin{equation}
		(\mathrm{d} F_p (v))(f) = v(f \circ F), \quad \forall f \in C^\infty(N), v \in T_p M.
	\end{equation}
\end{definition}
\begin{remark}
	This is quite natural. To give a geometric intuition, take a curve $\gamma$ in $M$ with $\gamma(0) = p$ and $\gamma'(0) = v$. Then $F \circ \gamma$ is a curve in $N$ with $(F \circ \gamma)(0) = F(p)$, and the tangent vector of $F \circ \gamma$ at $0$ is $\mathrm{d} F_p (v)$.

	Here $v$ is a directional derivative operator acting on functions on $M$. which is given: $v(g) = \frac{\mathrm{d} }{\mathrm{d} t} \big|_{t=0} g(\gamma(t))$ for $g \in C^\infty(M)$. Then $\mathrm{d} F_p (v)$ is also a directional derivative operator acting on functions on $N$: for $f \in C^\infty(N)$,
	\begin{equation*}
		(\mathrm{d} F_p (v))(f) = \frac{\mathrm{d} }{\mathrm{d} t} \Big|_{t=0} f((F \circ \gamma)(t)) = \frac{\mathrm{d} }{\mathrm{d} t} \Big|_{t=0} (f \circ F)(\gamma(t)) = v(f \circ F).
	\end{equation*}
\end{remark}
The operator $\mathrm{d} F_p$ is linear, as for $v,w \in T_p M$, $c \in \mathbb{R}$, we have
\begin{equation*}
	(\mathrm{d} F_p (c v + w))(f) = (c v + w)(f \circ F) = c v(f \circ F) + w(f \circ F) = c (\mathrm{d} F_p (v))(f) + (\mathrm{d} F_p (w))(f).
\end{equation*}
Is also follows the Leibniz product rule:
\begin{equation*}
	\begin{aligned}
		(\mathrm{d} F_p (v))(fg) & = v((fg) \circ F) = v((f \circ F)(g \circ F))                        \\
		                         & = (f \circ F)(p) v(g \circ F) + (g \circ F)(p) v(f \circ F)          \\
		                         & = f(F(p)) (\mathrm{d} F_p (v))(g) + g(F(p)) (\mathrm{d} F_p (v))(f).
	\end{aligned}
\end{equation*}

\begin{proposition}{Properties of Differential}{Properties of Differential}
	Let $M,N,P$ be smooth manifolds, with or without boundary, and let $F: M \to N$ and $G: N \to P$ be smooth maps. Then for each $p \in M$,
	\begin{itemize}
		\item $\mathrm{d} F_p : T_p M \to T_{F(p)} N$ is a linear map.
		\item (Chain Rule) $\mathrm{d} (G \circ F)_p = \mathrm{d} G_{F(p)} \circ \mathrm{d} F_p$.
		\item If $\mathrm{id}_M : M \to M$ is the identity map, then $\mathrm{d} (\mathrm{id}_M)_p$ is the identity map on $T_p M$.
		\item If $F$ is a diffeomorphism, then $\mathrm{d} F_p$ is an isomorphism, and $(\mathrm{d} F_p)^{-1} = \mathrm{d} (F^{-1})_{F(p)}$.
	\end{itemize}
\end{proposition}

Our first application of differentials is to relate tangent spaces of manifoldes to those of Euclidean spaces via charts. But first we shall prove that tangent vectors are local-behaved, for charts only give local information.

\begin{proposition}{Locality of Tangent Vectors}{Locality of Tangent Vectors}
	Let $M$ be a smooth manifold, with or without boundary, and let $p \in M$. If $v \in T_p M$ and $f,g \in C^\infty(M)$ agree on an open neighborhood of $p$, then $v(f) = v(g)$.
\end{proposition}
\begin{proof}
	Let $f,g\in C^\infty(M)$ agree on an open neighborhood $U$ of $p$. Then $h=f-g$ vanishes on $U$. Let $\psi \in C^\infty(M)$ be a smooth bump function that is $1$ on $\supp h$ and $\supp \psi \subseteq M-\left\{ p \right\}$. Then $h = h \psi$, so by the previous proposition, we have
	\begin{equation*}
		v(h) = v(h \psi) = h(p) v(\psi) + \psi(p) v(h) = 0.
	\end{equation*} Thus, $v(f) = v(g)$.
\end{proof}

\begin{proposition}{Tangent Space to Open Subsets}{Tangent Space to Open Subsets}
	Let $M$ be a smooth manifold, with or without boundary, and let $U \subseteq M$ be an open subset. Let $\iota : U \hookrightarrow M$ be the inclusion map. Then for $p\in U$, the differential
	\begin{equation*}
		\mathrm{d} \iota_p : T_p U \to T_p M
	\end{equation*}
	is an isomorphism.
\end{proposition}
\begin{proof}
	Via the extension lemma, every $f \in C^\infty(U)$ can be extended to a function $\tilde{f} \in C^\infty(M)$ such that $\tilde{f}|_U = f$. Thus, with the locality of tangent vectors, we can easily see the result.
\end{proof}
Therefore, it is safe to identify $T_p U$ with $T_p M$ via the inclusion map.

\begin{theorem}{Dimension of Tangent Space}{Dimension of Tangent Space}
	Let $M$ be a smooth manifold of dimension $n$, and let $p \in M$. Then $T_p M$ is an $n$-dimensional real vector space.
\end{theorem}
\begin{proof}
	Take a chart $(U, \varphi)$ containing $p$. Then as $\varphi$ is a diffeomorphism from $U$ to an open subset $\hat{U} \subseteq \mathbb{R}^n$, by the previous proposition, we have an isomorphism
	\begin{equation*}
		T_p M \cong T_p U \cong T_{\varphi(p)} \hat{U} \cong T_{\varphi(p)} \mathbb{R}^n.
	\end{equation*}
	Thus, by the earlier result on $\mathbb{R}^n$, we have $\dim T_p M = n$.
\end{proof}

Now we address points on the boundary of manifolds with boundary. The situation is similar. First, we shall relate the tangent spaces of $T_a \mathbb{H}^n$ to those of $\mathbb{R}^n$ when $a \in \partial \mathbb{H}^n$. As $\mathbb{H}^n$ is not an open subset of $\mathbb{R}^n$, we cannot use the previous proposition \ref{prop:Tangent Space to Open Subsets} directly. However, we have the following result.

\begin{lemma}{Inclusion of $\mathbb{H}^n$}{Inclusion of mathbbHn}
	Let $\iota : \mathbb{H}^n \hookrightarrow \mathbb{R}^n$ be the inclusion map. Then for each $a \in \partial \mathbb{H}^n$, the differential $\mathrm{d} \iota_a : T_a \mathbb{H}^n \to T_a \mathbb{R}^n$ is a linear isomorphism.
\end{lemma}
\begin{proof}
	Assume $\mathrm{d} \iota_a(v) = 0$, then for all $f \in C^\infty(\mathbb{H}^n)$, we let $\tilde{f} \in C^\infty(\mathbb{R}^n)$ be an extension of $f$. Thus, $\tilde{f} \circ \iota = f$, and we have
	\begin{equation*}
		v(f) = v(\tilde{f} \circ \iota) = (\mathrm{d} \iota_a (v))(\tilde{f}) = 0.
	\end{equation*}
	So $v=0$ and $\mathrm{d} \iota_a$ is injective.

	For surjectivity, let $w \in T_a \mathbb{R}^n$. Define $v : C^\infty(\mathbb{H}^n) \to \mathbb{R}$ by $v(f) = w(\tilde{f})$, where $\tilde{f} \in C^\infty(\mathbb{R}^n)$ is any extension of $f$. Thus
	\begin{equation*}
		v(f) = w^i \frac{\partial \tilde{f}}{\partial x^i} (a).
	\end{equation*}
	From continuity, this does not depend on the choice of extension $\tilde{f}$, as we can get the result by limiting process from points in the interior of $\mathbb{H}^n$. So we have $\mathrm{d} \iota_a (v) = w$.
\end{proof}

Therefore, it is safe to identify $T_a \mathbb{H}^n$ with $T_a \mathbb{R}^n$ via the inclusion map, even for $a \in \partial \mathbb{H}^n$.

\begin{proposition}{Dimension of Tangent Space with Boundary}{Dimension of Tangent Space with Boundary}
	Let $M$ be a smooth manifold of dimension $n$ with boundary, and let $p \in M$. Then $T_p M$ is an $n$-dimensional real vector space.
\end{proposition}

Next, as we know that for a finite-dimensional vector space, there exists a natural smooth structure on it. We shall see that the tangent space to a vector space at any point is naturally isomorphic to the vector space itself.

\begin{proposition}{Tangent Space to a Vector Space}{Tangent Space to a Vector Space}
	Let $V$ be a finite-dimensional real vector space with the standard smooth structure, and let $v \in V$. Then there is a natural isomorphism $V \cong T_v V$, defined by
	\begin{equation*}
		v \mapsto D_v|_a, \quad D_v|_a (f) = \frac{\mathrm{d} }{\mathrm{d} t} \Big|_{t=0} f(a + tv), \quad \forall f \in C^\infty(V).
	\end{equation*}
	For any liner transformation $T: V \to W$ between finite-dimensional real vector spaces, we have
	\begin{equation}
		L \cong \mathrm{d} L_a, \qquad \mathrm{d} L_a(D_v|_a) = D_{L(v)}|_{L(a)}.
	\end{equation}
\end{proposition}

Therefore, we can identify $T_v V$ with $V$ itself via the above isomorphism. For example, since $GL(n, \mathbb{R})$ is an open subset of the vector space $M_{n \times n}(\mathbb{R})$, we can identify $T_A GL(n, \mathbb{R})$ with $M_{n \times n}(\mathbb{R})$ for each $A \in GL(n, \mathbb{R})$.

For products, we have the following result.
\begin{theorem}{Tangent Space to Product Manifolds}{Tangent Space to Product Manifolds}
	Let $M_1, \ldots ,M_k$ be smooth manifolds, at most one have boundary. Let $ \pi_j : M_1 \times \cdots \times M_k \to M_j$ be the projection map onto the $j$-th factor. Then for each $p = (p_1, \ldots ,p_k) \in M_1 \times \cdots \times M_k$, the map
	\begin{equation*}
		\alpha_p : T_p (M_1 \times \cdots \times M_k) \to T_{p_1} M_1 \oplus \cdots \oplus T_{p_k} M_k, \quad \alpha_p (v) = (\mathrm{d} \pi_1)_p (v), \ldots ,(\mathrm{d} \pi_k)_p (v)
	\end{equation*}
	Is an isomorphism.
\end{theorem}
Therefore, we can identify $T_p (M_1 \times \cdots \times M_k)$ with $T_{p_1} M_1 \oplus \cdots \oplus T_{p_k} M_k$ via the above isomorphism.

\section{Computation in Coordinates}
We shall use charts to compute tangent vectors and differentials in coordinates.

Suppose $M$ is a smooth manifold of dimension $n$ (without boundary for simplicity), and $(U, \varphi)$ is a chart containing $p \in M$. Then $\varphi : U \to \hat{U} \subseteq \mathbb{R}^n$ is a diffeomorphism, thus $\mathrm{d} \varphi_p : T_p M \to T_{\varphi(p)} \mathbb{R}^n$ is an isomorphism.

In $\mathbb{R}^n$, we have the standard basis
\begin{equation*}
	\frac{\partial }{\partial x^i} \Big|_{\varphi(p)} : f \mapsto \frac{\partial f}{\partial x^i} (\varphi(p)), \quad i=1, \ldots ,n.
\end{equation*}
Therefore, the preimages of these basis vectors under $\mathrm{d} \varphi_p$ form a basis of $T_p M$, denoted by
\begin{equation}
	\frac{\partial }{\partial x^i} \Big|_p = (\mathrm{d} \varphi_p)^{-1} \left( \frac{\partial }{\partial x^i} \Big|_{\varphi(p)} \right) = \mathrm{d} (\varphi^{-1})_{\varphi(p)} \left( \frac{\partial }{\partial x^i} \Big|_{\varphi(p)} \right), \quad i=1, \ldots ,n.
\end{equation}
Acting on $f \in C^\infty(M)$, we have
\begin{equation*}
	\frac{\partial }{\partial x^i} \Big|_p (f) = \frac{\partial (f \circ \varphi^{-1})}{\partial x^i} (\varphi(p)) = \frac{\partial \hat{f}}{\partial x^i} (\hat{p}), \quad \hat{f} = f \circ \varphi^{-1}, \quad \hat{p} = \varphi(p).
\end{equation*}
which is the coordinate expression of $f$ and $p$ in $\mathbb{R}^n$. We call $\displaystyle \left\{ \frac{\partial }{\partial x^i} \Big|_p : i=1, \ldots ,n \right\}$ the \textbf{coordinate basis} of $T_p M$ induced by the chart $(U, \varphi)$.

\begin{remark}
	In $\mathbb{R}^n$, the coordinate basis vectors are just the partial derivative operators along the coordinate axes.
\end{remark}

For points on the boundary of manifolds with boundary, the situation is similar, just replacing $\mathbb{R}^n$ with $\mathbb{H}^n$, and using the inclusion isomorphism between $T_a \mathbb{H}^n$ and $T_a \mathbb{R}^n$ for $a \in \partial \mathbb{H}^n$.

\begin{theorem}{The Coordinate Basis}{The Coordinate Basis}
	Let $M$ be a smooth manifold of dimension $n$, with or without boundary, and let $p\in M$. Then take any chart $(U, \varphi)$ containing $p$. Then the coordinate vectors
	\begin{equation*}
		\frac{\partial }{\partial x^i} \Big|_p = \mathrm{d} (\varphi^{-1})_{\varphi(p)} \left( \frac{\partial }{\partial x^i} \Big|_{\varphi(p)} \right), \quad i=1, \ldots ,n
	\end{equation*}
	form a basis of $T_p M$.
\end{theorem}

This a tangent vector $v \in T_p M$ can be expressed in coordinates as
\begin{equation}
	v = v^i \frac{\partial }{\partial x^i} \Big|_p, \quad v^i = v(x^i),
\end{equation}
where $x^i = \pi_i \circ \varphi$ are the coordinate functions on $U$. The numbers $v^i$ are called the \textbf{components} of $v$ with respect to the coordinate basis induced by the chart $(U, \varphi)$.

\subsection{The Differential in Coordinates}
Now, we shall do computations of differentials of smooth maps in coordinates form. First, for simplicity consider $U \subseteq \mathbb{R}^n$ and $V \subseteq \mathbb{R}^m$ be open subsets, and let $F: U \to V$ be a smooth map. For $p \in U$, we have $\mathrm{d} F_p : T_p \mathbb{R}^n \to T_{F(p)} \mathbb{R}^m$ being a linear map. In the standard coordinate bases, we have
\begin{equation*}
	\mathrm{d} F_p \left( \frac{\partial }{\partial x^i} \Big|_p \right)f = \frac{\partial}{\partial x^i} \Big|_p (f \circ F) = \frac{\partial f}{\partial y^j} (F(p)) \frac{\partial F^j}{\partial x^i} (p) = \left( \frac{\partial F^j}{\partial x^i} (p) \frac{\partial }{\partial y^j} \Big|_{F(p)} \right) f.
\end{equation*}
Thus, we have
\begin{equation}
	\mathrm{d} F_p \left( \frac{\partial }{\partial x^i} \Big|_p \right) = \frac{\partial F^j}{\partial x^i} (p) \frac{\partial }{\partial y^j} \Big|_{F(p)}.
\end{equation}
Writing in matrix form, we have
	{
		\everymath{\displaystyle}
		\renewcommand{\arraystretch}{2.5}
		\begin{equation}
			\mathrm{d} F_p =
			\begin{pNiceMatrix}
				\frac{\partial F^1}{\partial x^1} (p) & \frac{\partial F^1}{\partial x^2} (p) & \cdots & \frac{\partial F^1}{\partial x^n} (p) \\
				\frac{\partial F^2}{\partial x^1} (p) & \frac{\partial F^2}{\partial x^2} (p) & \cdots & \frac{\partial F^2}{\partial x^n} (p) \\
				\vdots                                & \vdots                                & \ddots & \vdots                                \\
				\frac{\partial F^m}{\partial x^1} (p) & \frac{\partial F^m}{\partial x^2} (p) & \cdots & \frac{\partial F^m}{\partial x^n} (p)
			\end{pNiceMatrix}
		\end{equation}
	}
which is just the Jacobian matrix of $F$ at $p$. The same can be said if $U$ is an open subset of $\mathbb{H}^n$, so do $V$.

For a more general case, let $M,N$ be smooth manifolds of dimension $n,m$ respectively, with or without boundary, and let $F: M \to N$ be a smooth map. Take charts $(U, \varphi)$ and $(V, \psi)$ containing $p \in M$ and $F(p) \in N$ respectively. Then we have $\mathrm{d} \hat{F}_{\hat{p}} : T_{\hat{p}} \mathbb{R}^n \to T_{\hat{F}(\hat{p})} \mathbb{R}^m$ being the differential of the smooth map $\hat{F} = \psi \circ F \circ \varphi^{-1} : \hat{U} \to \hat{V}$ at $\hat{p} = \varphi(p)$. In the coordinate bases, we have
\begin{equation}
	\begin{aligned}
		\mathrm{d} F_p \left( \frac{\partial }{\partial x^i} \Big|_p \right) & = \mathrm{d} (\psi^{-1})_{\hat{F}(\hat{p})} \circ \mathrm{d} \hat{F}_{\hat{p}} \circ \mathrm{d} \varphi_p \left( \frac{\partial }{\partial x^i} \Big|_p \right)      \\
		                                                                     & = \mathrm{d} (\psi^{-1})_{\hat{F}(\hat{p})} \left( \mathrm{d} \hat{F}_{\hat{p}} \left( \frac{\partial }{\partial x^i} \Big|_{\hat{p}} \right) \right)                \\
		                                                                     & = \mathrm{d} (\psi^{-1})_{\hat{F}(\hat{p})} \left( \frac{\partial \hat{F}^j}{\partial x^i} (\hat{p}) \frac{\partial }{\partial y^j} \Big|_{\hat{F}(\hat{p})} \right) \\
		                                                                     & = \frac{\partial \hat{F}^j}{\partial x^i} (\hat{p}) \frac{\partial }{\partial y^j} \Big|_{F(p)}.
	\end{aligned}
\end{equation}
Which is just the pushforward of the Jacobian matrix of $\hat{F}$ at $\hat{p}$ via the charts.

\subsection{Change of Coordinates}
Suppose $(U, \varphi)$ and $(V, \psi)$ are two smooth charts on $M$ and $p \in U \cap V$. Denote the coordinate functions of $\varphi$ by $x^i = \pi_i \circ \varphi$ and those of $\psi$ by $\tilde{x}^i = \pi_i \circ \psi$. Therefore, any tangent vector $v \in T_p M$ can be expressed in both coordinate bases, and we want to find the relation between the components.

To do it, consider the transition map $\psi \circ \varphi^{-1} : \varphi(U \cap V) \to \psi(U \cap V)$, and we write its coordinate functions by
\begin{equation*}
	\varphi \circ \psi^{-1} (x) = ( \tilde{x}^1 (x), \ldots ,\tilde{x}^n (x) ).
\end{equation*}
We have identified $V$ with $\psi(V) \subseteq \mathbb{R}^n$ via the chart $\psi$, so we use the same notation $\tilde{x}^i$ for the coordinate functions on $V$ for simplicity. Then we have the differential
\begin{equation*}
	\mathrm{d} (\psi \circ \varphi^{-1})_{\varphi(p)} : T_{\varphi(p)} \mathbb{R}^n \to T_{\psi(p)} \mathbb{R}^n.
\end{equation*}
by the previous result, we have
\begin{equation}
	\mathrm{d} (\psi \circ \varphi^{-1})_{\varphi(p)} \left( \frac{\partial }{\partial x^i} \Big|_{\varphi(p)} \right) = \frac{\partial \tilde{x}^j}{\partial x^i} (\varphi(p)) \frac{\partial }{\partial \tilde{x}^j} \Big|_{\psi(p)}.
\end{equation}
So we have pull back to $T_p M$ via the charts:
\begin{equation}
	\begin{aligned}
		\frac{\partial }{\partial x^i} \Big|_p & = \mathrm{d} (\varphi^{-1})_{\varphi(p)} \left( \frac{\partial }{\partial x^i} \Big|_{\varphi(p)} \right)                                                   \\
		                                       & = \mathrm{d} (\psi^{-1})_{\psi(p)} \circ \mathrm{d} (\psi \circ \varphi^{-1})_{\varphi(p)} \left( \frac{\partial }{\partial x^i} \Big|_{\varphi(p)} \right) \\
		                                       & = \frac{\partial \tilde{x}^j}{\partial x^i} (\varphi(p)) \frac{\partial }{\partial \tilde{x}^j} \Big|_p.
	\end{aligned}
\end{equation}
Therefore, the components of $v$ in the two coordinate bases are related by
\begin{equation}
	\tilde{v}^j = \frac{\partial \tilde{x}^j}{\partial x^i} (\hat{p}) v^i.
\end{equation}

\section{The Tangent Bundle}
\begin{definition}{The Tangent Bundle}{The Tangent Bundle}
	Let $M$ be a smooth manifold, with or without boundary. The \textbf{tangent bundle} of $M$ is the disjoint union
	\begin{equation*}
		TM = \bigsqcup_{p \in M} T_p M = \left\{ (p,v) : p \in M, v \in T_p M \right\}.
	\end{equation*}
	The map $\pi : TM \to M$ defined by $\pi(p,v) = p$ is called the \textbf{bundle projection}.
\end{definition}

For example, the tangent bundle of $\mathbb{R}^n$ is naturally isomorphic to $\mathbb{R}^n \times \mathbb{R}^n$ via the isomorphism
\begin{equation*}
  \mathbb{R}^n \times \mathbb{R}^n \to T \mathbb{R}^n, \quad (a,v) \mapsto (a, D_v|_a).
\end{equation*}
But for general manifolds, we cannot identify $TM$ with $M \times \mathbb{R}^n$ globally because we cannot have a natural way to identify each tangent space $T_p M$ with each other.

\begin{theorem}{Structure of the Tangent Bundle}{Structure of the Tangent Bundle}
  Let $M$ be a smooth manifold of dimension $n$. Then $TM$ has a natural topology and smooth structure such that $TM$ is a smooth manifold of dimension $2n$. With this structure, the bundle projection $\pi : TM \to M$ is a smooth map.
\end{theorem}
\begin{proof}
  The ultimate intuition is to do it locally via charts. For each smooth chart $(U, \varphi)$ on $M$, note that $\pi^{-1}(U) = \bigsqcup_{p \in U} T_p M$. Define a map $\tilde{\varphi} : \pi^{-1}(U) \to \mathbb{R}^{2n}$ by
  \begin{equation}
    \tilde{\varphi} (p,v) = \tilde{\varphi} \left( v^i \frac{\partial }{\partial x^i} \Big|_p \right) = \left( x^1(p), \ldots ,x^n(p), v^1, \ldots ,v^n \right),
  \end{equation}
  So the image set is $\hat{U} \times \mathbb{R}^n$, being an open subset of $\mathbb{R}^{2n}$. It is also a bijection from $\pi^{-1}(U)$ to $\hat{U} \times \mathbb{R}^n$, because
  \begin{equation*}
    \tilde{\varphi}^{-1} (x^1, \ldots ,x^n, v^1, \ldots ,v^n) = v_i \frac{\partial }{\partial x^i} \Big|_{\varphi^{-1}(x^1, \ldots ,x^n)}.
  \end{equation*}

  Now suppose we have two smooth charts $(U, \varphi)$ and $(V, \psi)$ on $M$ and let $( \pi^{-1}(U), \tilde{\varphi} )$ and $( \pi^{-1}(V), \tilde{\psi} )$ be the corresponding charts on $TM$. Then the sets
  \begin{equation*}
    \tilde{\varphi} ( \pi^{-1}(U) \cap \pi^{-1}(V) ) = \varphi(U \cap V) \times \mathbb{R}^n, \quad \tilde{\psi} ( \pi^{-1}(U) \cap \pi^{-1}(V) ) = \psi(U \cap V) \times \mathbb{R}^n
  \end{equation*}
  are both open subsets of $\mathbb{R}^{2n}$. The transition map is given by
  \begin{equation*}
    \tilde{\psi} \circ \tilde{\varphi}^{-1} (x^1, \ldots ,x^n, v^1, \ldots ,v^n) = \left( \tilde{x}^1 (x), \ldots ,\tilde{x}^n (x), \frac{\partial \tilde{x}^j}{\partial x^1} (x) v^i, \ldots ,\frac{\partial \tilde{x}^j}{\partial x^n} (x) v^i \right),
  \end{equation*}
  which is smooth.

  Finally, choose a countable cover of charts $\left\{ (U_\alpha, \varphi_\alpha) \right\}$ of $M$, then the corresponding charts $\left\{ ( \pi^{-1}(U_\alpha), \tilde{\varphi}_\alpha ) \right\}$ form an atlas of $TM$. The conditions of \ref{lem:The Smooth Manifold Chart Lemma} are easily verified.
\end{proof}

\begin{remark}
  For smooth manifolds with boundary, the construction is similar, just replacing $\mathbb{R}^n$ with $\mathbb{H}^n$ in the above proof. We note that the only ``half-ness'' happens in the base manifold $M$, while each tangent space $T_p M$ is a full $n$-dimensional vector space, so no harm is done to the tangent bundle structure.
\end{remark}

\begin{proposition}{Single-Chart Tangent Bundle}{Single-Chart Tangent Bundle}
  If $M$ is a smooth manifold of dimension $n$ (with or without boundary) that can be covered by a single chart $(M, \varphi)$, then the tangent bundle $TM$ is diffeomorphic to $M \times \mathbb{R}^n$.
\end{proposition}
\begin{proof}
  Obvious.
\end{proof}

\begin{remark}
  NOTE that although we can locally view $TM$ as $U \times \mathbb{R}^n$ via charts, there is no natural way to identify $TM$ with $M \times \mathbb{R}^n$ globally in general. In fact, this may not be true in many cases.
\end{remark}

Putting all pointwise differentials together, we have a map
\begin{equation*}
  \mathrm{d} F : TM \to TN, \quad \mathrm{d} F (p,v) = (F(p), \mathrm{d} F_p (v)),
\end{equation*}
called the global differential of $F$.

\begin{theorem}{Global Differential is Smooth}{Global Differential is Smooth}
  Let $M,N$ be smooth manifolds, with or without boundary, and let $F: M \to N$ be a smooth map. Then the global differential
  \begin{equation*}
    \mathrm{d} F : TM \to TN
  \end{equation*}
  is a smooth map.
\end{theorem}
\begin{proof}
	From the coordinate expression, we have
	\begin{equation*}
		\mathrm{d} F (p,v) = \left( F(p), \frac{\partial F^j}{\partial x^i} (p) v^i \right),
	\end{equation*}
	which is smooth for $F$ is.
\end{proof}
\begin{proposition}{Properties of Global Differential}{Properties of Global Differential}
	Let $M,N,P$ be smooth manifolds, with or without boundary, and let $F: M \to N$ and $G: N \to P$ be smooth maps. Then for each $(p,v) \in TM$,
	\begin{itemize}
		\item (Chain Rule) $\mathrm{d} (G \circ F) = \mathrm{d} G \circ \mathrm{d} F$.
		\item If $\mathrm{id}_M : M \to M$ is the identity map, then $\mathrm{d} (\mathrm{id}_M) = \mathrm{id}_{TM}$.
		\item If $F$ is a diffeomorphism, then $\mathrm{d} F$ is a diffeomorphism, and $(\mathrm{d} F)^{-1} = \mathrm{d} (F^{-1})$.
	\end{itemize}
\end{proposition}
Just using proposition \ref{prop:Properties of Differential} would do. From now we may denote $\mathrm{d} F^{-1}$ for either $\mathrm{d} (F^{-1})$ or $(\mathrm{d} F)^{-1}$, when $F$ is a diffeomorphism.

\section{Velocity Vectors of Curves}

\begin{definition}{Curves}{Curves}
	Let $M$ be a manifold, with or without boundary. A \textbf{curve} in $M$ is a continuous map $\gamma : J \to M$, where $J \subseteq \mathbb{R}$ is an open interval. Sometimes we may want $J$ to have one or both endpoints, in which case slight modifications are needed.
\end{definition}

\begin{definition}{Velocity}{Velocity}
	Let $M$ be a smooth manifold, with or without boundary, and let $\gamma : J \to M$ be a smooth curve. The \textbf{velocity} of $\gamma$ at $t_0 \in J$ is the tangent vector
	\begin{equation}
		\gamma'(t_0) = \mathrm{d} \gamma_{t_0} \left( \frac{\mathrm{d} }{\mathrm{d} t} \Big|_{t_0} \right) \in T_{\gamma(t_0)} M.
	\end{equation}
	Other notations include
	\begin{equation*}
		\dot{\gamma}(t_0) = \frac{\mathrm{d} \gamma}{\mathrm{d} t} (t_0) = \frac{\mathrm{d} \gamma}{\mathrm{d} t} \Big|_{t = t_0}
	\end{equation*}
\end{definition}
The tangent vector $\gamma'(t_0)$ acts on functions by
\begin{equation*}
	\gamma'(t_0) (f) = \frac{\mathrm{d} }{\mathrm{d} t} \Big|_{t=t_0} f(\gamma(t)) = \frac{\mathrm{d} }{\mathrm{d} t} \Big|_{t=t_0} (f \circ \gamma)(t).
\end{equation*}
which is the rate of change of $f$ along the curve $\gamma$ at $t_0$. For a smooth chart $(U, \varphi)$ containing $\gamma(t_0)$, we can express the velocity in coordinates as
\begin{equation*}
	\gamma'(t_0) = \frac{\mathrm{d} \gamma^i}{\mathrm{d} t} (t_0) \frac{\partial }{\partial x^i} \Big|_{\gamma(t_0)} = \left( \frac{\mathrm{d} \gamma^1}{\mathrm{d} t} (t_0), \ldots ,\frac{\mathrm{d} \gamma^n}{\mathrm{d} t} (t_0) \right),
\end{equation*}
which is familiar in Euclidean space.

Next, we shall see that every tangent vector can be expressed as the velocity of some curve, which will lead us to an equivalent definition of tangent vectors.

\begin{proposition}{Tangent Vector as Velocity}{Tangent Vector as Velocity}
	Let $M$ be a smooth manifold, with or without boundary, and let $p \in M$. Then for any tangent vector $v \in T_p M$, there exists a smooth curve $\gamma : J \to M$ such that $\gamma(0) = p$ and $\gamma'(0) = v$.
\end{proposition}
\begin{proof}
	First suppose $p\in \Int M$, then let $(U, \varphi)$ be a smooth chart centering $p$. Then we write $v = v^i \partial /\partial x^i|_p$. For sufficiently small $ \epsilon$, we have a smooth curve $\gamma : (-\epsilon, \epsilon) \to U$ defined by
	\begin{equation*}
		\gamma(t) = \varphi^{-1} \left( t v^1, \ldots ,t v^n \right)
	\end{equation*}
	which is smooth because $\varphi^{-1}$ is smooth.

	Now if $p\in \partial M$, then let $(U, \varphi)$ be a smooth boundary chart centering $p$. We can similarly define a smooth curve $\gamma : [0, \epsilon) \to U$ or $(-\epsilon, 0] \to U$ by the same formula for sufficiently small $\epsilon > 0$, depending on the sign of the first component of $v$.
\end{proof}

For composition, we have the following result.
\begin{proposition}{Velocity under Composition}{Velocity under Composition}
	Let $M,N$ be smooth manifolds, with or without boundary, and let $F: M \to N$ be a smooth map. If $\gamma : J \to M$ is a smooth curve, then for each $t_0 \in J$,
	\begin{equation*}
		(F \circ \gamma)'(t_0) = \mathrm{d} F_{\gamma(t_0)} \left( \gamma'(t_0) \right).
	\end{equation*}
\end{proposition}
\begin{proof}
	Just the chain rule:
	\begin{equation*}
		(F \circ \gamma)'(t_0) (f) = \mathrm{d} (F \circ \gamma)_{t_0} \left( \frac{\mathrm{d} }{\mathrm{d} t} \Big|_{t_0} \right) (f) = \mathrm{d} F \circ \mathrm{d} \gamma_{t_0} \left( \frac{\mathrm{d} }{\mathrm{d} t} \Big|_{t_0} \right) (f) = \mathrm{d} F_{\gamma(t_0)} \left( \gamma'(t_0) \right) (f).
	\end{equation*}
\end{proof}

We can also use curve velocity to compute differentials: Suppose $M,N$ are smooth manifolds, with or without boundary, and let $F: M \to N$ be a smooth map, then to compute $\mathrm{d} F_p (v)$ for $p \in M$ and $v \in T_p M$, we can first find a smooth curve $\gamma : J \to M$ such that $\gamma(0) = p$ and $\gamma'(0) = v$, then we have
\begin{equation*}
	\mathrm{d} F_p (v) = \mathrm{d} F_p \left( \gamma'(0) \right) = (F \circ \gamma)'(0).
\end{equation*}

\section{Alternative Definition of Tangent Vectors}
\subsection{Derivations of the Space of Germs}
A smooth function element on $M$ is an ordered pair $(f,U)$, where $U \subseteq M$ is an open set and $f \in C^\infty(U)$. Two smooth function elements $(f,U)$ and $(g,V)$ are said to be equivalent at $p \in U \cap V$ if there exists an open neighborhood $W \subseteq U \cap V$ of $p$ such that $f|_W = g|_W$. The equivalence class of $(f,U)$ at $p$ is called the \textbf{germ} of $f$ at $p$, and the set if all germs of smooth functions at $p$ is denoted by $C^\infty_p(M)$.

\begin{remark}
	Intuitively, $C^\infty _p(M)$ of a smooth function at $p$ contains all distinguishable smooth functions locally around $p$.
\end{remark}

We notice that $C^\infty _p(M)$ is a real vector space and an associative algebra under operations defined by
\begin{itemize}
	\item Addition: $[(f,U)] + [(g,V)] = [(f+g, U \cap V)]$.
	\item Scalar Multiplication: $c [(f,U)] = [(cf,U)]$.
	\item Multiplication: $[(f,U)] \cdot [(g,V)] = [(fg, U \cap V)]$.
\end{itemize}
Now we denote the germ of $f$ at $p$ simply by $[f]_p$ when there is no confusion.

A derivation of $C^\infty _p(M)$ is a linear map $v : C^\infty _p(M) \to \mathbb{R}$ such that for all $[f]_p, [g]_p \in C^\infty _p(M)$,
\begin{equation}
	v( [f]_p \cdot [g]_p ) = f(p) v( [g]_p ) + g(p) v( [f]_p ).
\end{equation}
The set of all derivations of $C^\infty _p(M)$ is denoted by $\mathcal{D}_p(M)$. And it is simple to verify that $\mathcal{D}_p(M)$ is naturally isomorphic to $T_p M$.

\subsection{Equivalent Class of Curves}
This definition captures the intuitive idea of tangent vectors as ``directions'' at a point. Suppose $p$ is a point of $M$, and consider all smooth curves $\gamma : J \to M$ such that $\gamma(0) = p$. We say two such curves $\gamma_1$ and $\gamma_2$ are equivalent at $p$ if for any smooth function $f: M \to \mathbb{R}$,
\begin{equation}
	\frac{\mathrm{d} }{\mathrm{d} t} \Big|_{t=0} (f \circ \gamma_1)(t) = \frac{\mathrm{d} }{\mathrm{d} t} \Big|_{t=0} (f \circ \gamma_2)(t).
\end{equation}
The equivalence classes are denoted by $[\gamma]$, and all such equivalence classes form a set denoted by $\mathcal{V}_p(M)$, which is naturally isomorphic to $T_p M$.

\section{Categories and Functors}
A category $\mathcal{C}$ consists of
\begin{itemize}
	\item A class $\mathrm{Ob}(\mathcal{C})$, whose elements are called objects of $\mathcal{C}$.
	\item A class $\mathrm{Hom}(\mathcal{C})$, whose elements are called morphisms of $\mathcal{C}$.
	\item For each morphism $f \in \mathrm{Hom}(\mathcal{C})$, there are two objects $X,Y \in \mathrm{Ob}(\mathcal{C})$ called the source and target of $f$, denoted by $f: X \to Y$.
	\item For each triplet of objects $X,Y,Z \in \mathrm{Ob}(\mathcal{C})$, there is a mapping called composition
	      \begin{equation*}
		      \circ : \mathrm{Hom}(Y,Z) \times \mathrm{Hom}(X,Y) \to \mathrm{Hom}(X,Z), \quad (g,f) \mapsto g \circ f.
	      \end{equation*}
	      where $\mathrm{Hom}(A,B)$ is the class of all morphisms from $A$ to $B$.
\end{itemize}
The morphisms and objects must satisfy the following axioms:
\begin{itemize}
	\item (Associativity) For each $f: X \to Y$, $g: Y \to Z$ and $h: Z \to W$,
	      \begin{equation*}
		      h \circ (g \circ f) = (h \circ g) \circ f.
	      \end{equation*}
	\item (Identity) For each object $X \in \mathrm{Ob}(\mathcal{C})$, there exists an identity morphism $\mathrm{id}_X : X \to X$ such that for each $f: X \to Y$.
	      \begin{equation*}
		      \mathrm{id}_Y \circ f = f, \quad f \circ \mathrm{id}_X = f.
	      \end{equation*}
\end{itemize}

A morphism $f: X \to Y$ is called an isomorphism if there exists a morphism $g: Y \to X$ such that
\begin{equation*}
	g \circ f = \mathrm{id}_X, \quad f \circ g = \mathrm{id}_Y.
\end{equation*}



\end{document}
