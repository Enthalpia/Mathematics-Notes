\documentclass[../main.tex]{subfiles}

\begin{document}

\chapter{Topological Spaces}

The concept of topological space grew out of the study of the real line and Euclidean space and the study of continuous functions on these spaces.

\section{Topological Spaces}

\begin{definition}{Topology}{Topology}
	A \emph{topology} on a set $X$ is a collection $\mathcal{T}$ of subsets of $X$ (that is, $T \subseteq P(X)$) such that the following conditions hold:
\begin{itemize}
\item $\emptyset, X \in \mathcal{T}$.
\item The union of any collection of sets in $\mathcal{T}$ is in $\mathcal{T}$.
\item The intersection of any finite collection of sets in $\mathcal{T}$ is in $\mathcal{T}$.
\end{itemize}

A topological space is a pair $(X, \mathcal{T})$ where $X$ is a set and $\mathcal{T}$ is a topology on $X$.
\end{definition}

\begin{example}{Topologies}{Topologies}
\begin{itemize}
\item The \emph{discrete topology} on a set $X$ is the topology $\mathcal{T} = P(X)$, where $P(X)$ is the power set of $X$. In this topology, every subset of $X$ is open.
\item The \emph{indiscrete topology}, or trivial topology, on a set $X$ is the topology $\mathcal{T} = \left\{ \emptyset , X \right\}$. In this topology, only the empty set and the entire set are open.
\item Topologies on $\mathbb{R}$.
	\begin{enumerate}
		\item $\mathcal{T}_1$ consisting of $\mathbb{R}$, $\emptyset $, and all open intervals $(a,b)$.
		\item $\mathcal{T}_2$ consisting of $\mathbb{R}$, $\emptyset $, and all $\left[-n,n\right]$ for $n\in \mathbb{Z}_+$.
	\end{enumerate}
\item Topologies on $\mathbb{N}$.
	\begin{enumerate}
		\item The initial segment topology: $\mathcal{T}_1$ consisting of $\mathbb{N}$, $\emptyset $ and the set $\left\{ 1, \ldots ,n \right\}$ for $n\in \mathbb{N}$.
		\item The final segment topology: $\mathcal{T}_2$ consisting of $\mathbb{N}$, $\emptyset $ and the set $\left\{ n, n+1 \ldots  \right\}$ for $n\in \mathbb{N}$.
	\end{enumerate}
\end{itemize}
\end{example}

\begin{definition}{Finer Topologies}{Finer Topologies}
	If $\mathcal{T}_1$ and $\mathcal{T}_2$ are topologies on a set $X$ and $\mathcal{T}_1 \subseteq \mathcal{T}_2$, then $\mathcal{T}_2$ is said to be \emph{finer} than $\mathcal{T}_1$. Similarly, $\mathcal{T}_1$ is said to be \emph{coarser} than $\mathcal{T}_2$. If $\mathcal{T}_1$ is neither finer nor coarser than $\mathcal{T}_2$, then $\mathcal{T}_1$ and $\mathcal{T}_2$ are said to be \emph{incomparable}.
\end{definition}

\section{Open Sets, Closed Sets and Clopen Sets}

The axioms of a topology is the extension of the concept of open sets in Euclidean space. As we can see, every open sets in Euclidean space satisfies the axioms of a topology. Therefore, we can define open sets in a topological space as follows.

\begin{definition}{Open Sets}{Open Sets}
If $X$ is a set with a topology $\mathcal{T}$, a subset $U \subseteq X$ is \emph{open} if $U \in \mathcal{T}$.
\end{definition}

We define closed sets as the complements of open sets, which is a natural extension of the concept of closed sets in Euclidean space.
\begin{definition}{Closed Sets}{Closed Sets}
	Let $\left(X,\mathcal{T}\right)$ be a topological space. A subset $S \subseteq X$ is a \emph{closed set} if $X - S$ is open.
\end{definition}

\begin{theorem}{Properties of Closed Sets}{Properties Of Closed Sets}
	If $\left(X,\mathcal{T}\right)$ is a topological space, then
\begin{enumerate}
	\item $\emptyset$ and $X$ are closed.
	\item The intersection of any collection of closed sets is closed.
	\item The union of any finite collection of closed sets is closed.
\end{enumerate}
\end{theorem}
\begin{proof}
This is a rather straightforward consequence of the definition of closed sets. Using
\begin{equation*}
	X - \bigcap_{\alpha \in I} S_{\alpha} = \bigcup_{\alpha \in I} (X - S_{\alpha}) \quad \text{and} \quad X - \bigcup_{i=1}^n S_i = \bigcap_{i=1}^n (X - S_i),
\end{equation*}
would do.
\end{proof}
\begin{remark}
	Note that openness and closedness are not mutually exclusive. A set can be both open and closed, such as $\emptyset $ and $X$ in any topological space. Also, there are sets that are neither open nor closed, just like $(0,1]$ in the standard topology on $\mathbb{R}$.
\end{remark}

\begin{definition}{Clopen Sets}{Clopen Sets}
A subset $S$ in a topological space $(X, \mathcal{T})$ is a \emph{clopen set} if it is both open and closed.
\end{definition}
\begin{example}{Clopen Sets}{Clopen Sets}
\begin{itemize}
\item In every topological space, $\emptyset$ and $X$ are clopen.
\item In the discrete topology, every subset of $X$ is clopen.
\item In the indiscrete topology, only $\emptyset$ and $X$ are clopen.
\end{itemize}
\end{example}

\subsection{Distinct Topologies on Finite and Infinite Sets}
\begin{proposition}{Finite Set Topologies}{Finite Set Topologies}
\begin{enumerate}
	\item The number of topologies on a finite set increases as $\card{X}$ increases.
	\item if finite set $X$ has $n\in \mathbb{N}$ points, then it has at least $(n-1)!$ distinct topologies.
\end{enumerate}
\end{proposition}
\begin{proof}
Use mathematical induction. If $\card X=n$, and there are $M$ different topologies. Let  $\card Y=n+1$, the additional point is $x$. For any $\mathcal{T}$, let $\mathcal{T}' = \left\{ U\cup \left\{ x \right\}: U\in \mathcal{T} \right\}$ is a topology on $Y$ ($\mathcal{T}'$ is constructed by adding $x$ to every element of $\mathcal{T}$ ), and $\mathcal{T}'$ is a topology on $Y$. Then $\left\{ \emptyset ,Y \right\}$ is a new topology, thus $Y$ has at least $M+1$ topologies.

Furthermore, Let $X = \left\{ x_1, \ldots ,x_n \right\}$ and $Y = \left\{ x_1, \ldots ,x_{n+1} \right\}$. If $\mathcal{T}$ is any topology on $X$, fix $i\in \left\{ 1, \ldots ,n \right\}$, define $U_i$ by replacing any occurrences of $x_i$ in $Y$ with $x_{n+1}$, then $\mathcal{T}_i = \left\{ U_i : U\in \mathcal{T} \right\}$ is a topology on $Y$. Then for any topology on $X$, we can construct at least $n$ distinct topologies on $Y$. 
\end{proof}


\begin{proposition}{Infinite Set Topologies}{Infinite Set Topologies}
	If $X$ is any infinite set with cardinality $\aleph$, there are at least $2^{\aleph}$ distinct topologies on $X$.
\end{proposition}
\begin{proof}
	As $P(X)$ has cardinality $2^{\aleph}$, there are at least $2^{\aleph}$ distinct topologies on $X$: $\forall U\in P(X), \left\{ \emptyset ,U,X \right\}$ is a topology on $X$.
\end{proof}


\section{The Finite-Closed Topology}

Sometimes, describing a topology in terms of closed sets is more convenient than describing it in terms of open sets.
\begin{definition}{Finite-Closed Topology}{Finite-Closed Topology}
	Let $X$ be any nonempty set. A topology $\mathcal{T}_f$ on $X$ is called the \emph{finite-closed topology} if the closed sets in $\mathcal{T}_f$ are precisely the finite sets and $X$ itself.
\end{definition}

We shall prove that the finite-closed topology is indeed a topology.
\begin{proof}
We shall prove that the finite complement topology satisfies the axioms of a topology.
\begin{itemize}
	\item $\emptyset, X \in \mathcal{T}_f$.
	\item Let $\left\{ U_{\alpha}: \alpha\in I \right\}$ be a subset of $\mathcal{T}_f$, where $I$ is an index set. Then
	\begin{equation*}
		X - \bigcup_{\alpha \in I} U_{\alpha} = \bigcap_{\alpha \in I} (X - U_{\alpha})
	\end{equation*}
	is an intersection of finite sets, so it is finite. If all of $U_{\alpha}$ is $\emptyset $, then $X - \bigcup_{\alpha \in I} U_{\alpha} = X$. Therefore, $\bigcup_{\alpha \in I} U_{\alpha} \in \mathcal{T}_f$.
	\item Let $\left\{ U_i \right\}_{i=1}^n$ be a subset of $\mathcal{T}_f$. Then 
	\begin{equation*}
		X - \bigcup_{i=1}^n U_i = \bigcap_{i=1}^n (X - U_i)
	\end{equation*}
	is a finite intersection of finite sets, so it is finite. If there is an $i$ such that $U_i = \emptyset$, then $\bigcup_{i=1}^n U_i = \emptyset$. Therefore, $\bigcup_{i=1}^n U_i \in \mathcal{T}_f$.
\end{itemize}
\end{proof}

There are many other ways of constructing topologies.
\begin{theorem}{Topology of Preimages}{Topology Of Preimages}
Let $(Y,\mathcal{T})$ be a topological space and $X$ be a nonempty set. Let $f:X \rightarrow Y$ be a function. Then $\mathcal{T}_1 = \left\{ f^{-1}(S): S\in \mathcal{T} \right\}$ is a topology on $X$.
\end{theorem}

We shall prove a lemma first.
\begin{lemma}{Preimage of Union and Intersection}{Preimage Of Union And Intersection}
Let $f:X \rightarrow Y$. Then we have
\begin{equation}
	f^{-1}\left(\bigcup_{\alpha\in I} U_{\alpha}\right) = \bigcup_{\alpha\in I} f^{-1}(U_{\alpha})
\end{equation}
and
\begin{equation}
	f^{-1}\left(\bigcap_{\alpha\in I} U_i \right) = \bigcap_{\alpha\in I} f^{-1}(U_i) 
\end{equation}
for any $U_{\alpha} \subseteq Y$.
\end{lemma}
\begin{proof}
\begin{itemize}
	\item $\displaystyle \forall x\in f^{-1}\left(\bigcup_{\alpha\in I} U_{\alpha}\right)$,  we have $\displaystyle f(x) \in \bigcup_{\alpha\in I} U_{\alpha}$, then $f(x)\in U_{\alpha}$ for some $\alpha\in I$. Then $x\in f^{-1}(U_{\alpha})$, thus $\displaystyle x\in \bigcup_{\alpha\in I} f^{-1}(U_{\alpha})$. Other way is the same.
	\item $\displaystyle \forall x\in f^{-1}\left(\bigcap_{\alpha\in I} U_{\alpha}\right)$,  we have $\displaystyle f(x) \in \bigcap_{\alpha\in I} U_{\alpha}$, then $f(x)\in U_{\alpha}$ for all $\alpha\in I$. Then $\forall \alpha\in I, x\in f^{-1}(U_{\alpha})$, thus $\displaystyle x\in \bigcap_{\alpha\in I} f^{-1}(U_{\alpha})$. Other way is the same.
\end{itemize}
\end{proof}

\begin{remark}
Note that the image does not preserve set operation as preimages. Like for $f: \mathbb{R}\rightarrow \mathbb{R}, x \mapsto 0$, we have $f \left(\left\{ 0 \right\}\cap \left\{ 1 \right\}\right) \neq f \left(\left\{ 0 \right\}\right)\cap f \left(\left\{ 1 \right\}\right)$.
\end{remark}

We now prove theorem \ref{thm:Topology Of Preimages}

\begin{proof}
\begin{itemize}
\item First we have $f^{-1}(\emptyset) = \emptyset$ and $f^{-1}(Y) = X$, so $\emptyset, X \in \mathcal{T}_1$.
\item Let $\left\{ U_{\alpha} \right\}_{\alpha \in I}$ be a subset of $\mathcal{T}_1$. Then let $U_{\alpha} = f^{-1}\left(V_{\alpha}\right)$ where $V_{\alpha} \in \mathcal{T}$. Then
	\begin{equation*}
	\bigcup_{\alpha\in I} U_{\alpha} = f^{-1}\left(\bigcup_{\alpha\in I} V_{\alpha}\right) \in \mathcal{T}_1
	\end{equation*}
\item Similarly, we have
	\begin{equation*}
	\bigcap_{i=1}^{n} U_i = f^{-1}\left(\bigcap_{i=1}^{n} V_i\right)\in \mathcal{T}_1
	\end{equation*}
\end{itemize}
\end{proof}

\subsection{$T_1$-Spaces}
\begin{definition}{$T_1$-Spaces}{T1 Spaces}
A topological space $(X,\mathcal{T})$ is a $T_1$-space $\forall x\in X$, the set $\left\{ x \right\}$ is closed in $\mathcal{T}$.
\end{definition}

Of course, this means that any finite subset of $X$ is closed, so the finite-closed topology is indeed a $T_1$-space.

We can also rephrase the definition of $T_1$-space by separation.
\begin{theorem}{Equivalent Definition of $T_1$-space}{Equivalent Definition of T1-space}
A topological space is a $T_1$-space iff
\begin{equation}
\forall x,y\in X, x\neq y, \exists U,V\in \mathcal{T} (x\in U \land y\notin U \quad \land \quad x\notin V\land y\in V)
\end{equation}
\end{theorem}
\begin{proof}
\begin{itemize}
\item The if part: Let $U = \left\{ x \right\}^{c}$ and $V = \left\{ y \right\}^{c}$ would do.
\item The only if part: $\forall y\neq x$, let $V_y\in \mathcal{T}$ be such that $x\notin V_y \land y\in V_y$. Then $\bigcup_{y\neq x} V_y = \left\{ x \right\}^{c}$ is open so $\left\{ x \right\}$ is closed.
\end{itemize}
\end{proof}

\subsection{$T_0$-Spaces and the Sierpinski Space}
\begin{definition}{$T_0$-Space}{T0-Space}
A topological space $(X,\mathcal{T})$ is a $T_0$-space if $\forall a,b\in X, a\neq b$, there is an open set containing $a$ but not $b$, or there exists an open set containing $b$ but not $a$. That is
\begin{equation*}
\exists U\in \mathcal{T}(a\in U \land b\notin U)\quad \lor\quad \exists V\in \mathcal{T}(b\in V \land a\notin V)
\end{equation*}
\end{definition}

\begin{theorem}{}{T1 is T0}
Every $T_1$-space is a $T_0$-space.
\end{theorem}
\begin{proof}
We assume that $X$ has at least 2 points. If $(X,\mathcal{T})$ is a $T_1$-space, then $\forall a\neq b$, we have $\left\{ b \right\}$ is closed, then $X-\left\{ b \right\}\in \mathcal{T}$ contains $a$ but not $b$.
\end{proof}

The other way is not correct.
\begin{definition}{Sierpinski Space}{Sierpinski Space}
Let $X = \left\{ 0,1 \right\}$, $\mathcal{T}=\left\{ \emptyset , \left\{ 0 \right\}, X \right\}$ be a topology on $X$. This is called Sierpinski Space.
\end{definition}

It is easy to say that the Sierpinski space is a $T_0$-space but not $T_1$-space.

Similar to arbitrary topologies we have
\begin{proposition}{}{T0 Spaces on finite sets}
If $\card X = n$, then the number of $T_0$-space increases as $n$ increases.
\end{proposition}
\begin{proof}
Again by induction. If $\card X=n$, and there are $M$ different $T_0$-spaces. Let  $\card Y=n+1$, the additional point is $x$. For any $\mathcal{T}$ such that $(X, \mathcal{T})$ is a $T_0$-space. $\mathcal{T}' = \mathcal{T}\cup \left\{ U\cup \left\{ x \right\}: U\in \mathcal{T} \right\}$ is a topology on $Y$ ($\mathcal{T}'$ is constructed by $\mathcal{T}$ and adding $x$ to every element of $\mathcal{T}$ ), and $(Y,\mathcal{T}')$ is a $T_0$-space. 

We now construct a $T_0$-space that does not have this structure. Let $Y = \left\{ y_1, \ldots ,y_{n+1} \right\}$, then Let $Y_i = \left\{ y_1, \ldots ,y_i \right\}$, then let $\mathcal{T} = \left\{ Y_0, \ldots ,Y_{n+1} \right\}$, then $(Y,\mathcal{T})$ is a $T_0$-space.
\end{proof}

\begin{remark}
Our intuition of $T_0$ and $T_1$ spaces comes from ``how nicely can we distinguish points in the space''. 

In $T_0$-spaces, we can ``tell any two points apart'', that is, there does not exist two points that are either in or not in  the same open set. As the only properties of a topological space is the open sets, we can only distinguish two points if there exists an open set containing one but not the other. 

In $T_1$-spaces, we can ``tell any point apart from a closed set''. This ensures that we can tell two points from both directions. Moreover, every point is a closed set gives us convenience for the definition of limits and continuity.
\end{remark}

\subsection{Countable-Closed Topology}
\begin{definition}{Countable-Closed Topology}{Countable-Closed Topology}
Let $X$ be any infinite set. Then $\mathcal{T}_c = \left\{ U \subseteq X : X-U \text{ is countable }\lor U = \emptyset  \right\}$ is the countable-closed topology on $X$.
\end{definition}
This is very similar to the Finite-Closed Topology \ref{def:Finite-Closed Topology}.

\subsection{Intersection of Two Topologies}
\begin{proposition}{Intersection of topologies}{Intersection Of Topologies}
Let $\mathcal{T}_{\alpha}, \alpha\in I$ be topologies on $X$. Then
\begin{enumerate}
\item $\displaystyle \mathcal{T} = \bigcap_{\alpha\in I} T_{\alpha}$ is a topology on $X$.
\item If $(X,\mathcal{T}_{\alpha})$ are $T_1$-spaces, then $(X, \mathcal{T})$ is a $T_1$-space.
\end{enumerate}
\end{proposition}
\begin{proof}
\begin{enumerate}
\item First $\emptyset ,X\in \mathcal{T}$. And we have if $U_{\beta}\in \mathcal{T}$, then $\bigcup U_{\beta}\in \mathcal{T}$. Intersection the same.
\item If $\forall x\in X, \left\{ x \right\}$ is closed in $\mathcal{T}_{\alpha}$ for all $\alpha\in I$, then $X-\left\{ x \right\}\in \mathcal{T}_{\alpha}$. Then $X-\left\{ x \right\}\in \bigcap \mathcal{T}_{\alpha}$, meaning $\left\{ x \right\}$ is closed in $\mathcal{T}$.
\end{enumerate}
\end{proof}

\subsection{Door Space}
\begin{definition}{Door Space}{Door Space}
A topological space $(X,\mathcal{T})$ is a door space iff $\forall U\in \mathcal{T}$ is open or closed.
\end{definition}
\begin{example}{Door Spaces}{Door Spaces}
\begin{itemize}
\item The indiscrete space is a door space.
\end{itemize}
\end{example}

\subsection{Saturated Set}
\begin{definition}{Saturated Sets}{Saturated Sets}
If $(X,\mathcal{T})$ is a topological space and $S \subseteq X$, then $S$ is a saturated set if it is the intersection of open sets
\end{definition}
\begin{remark}
Well, it can be the intersection of infinite number of open sets, so it is not necessarily open.
\end{remark}

\begin{example}{Saturated Sets}{Saturated Sets}
\begin{itemize}
\item Every open set is a saturated set. In a finite set, every saturated set is open.
\item In a $T_1$-space every set is a saturated set.
\begin{proof}
For any set $S \subseteq X$, we have $\forall x\in X - S, \left\{ x \right\}$ is closed, so $X- \left\{ x \right\}$ is open. And
\begin{equation*}
S = \bigcap_{x\in X-S} \left(X - \left\{ x \right\}\right)
\end{equation*}
is a saturated set.
\end{proof}
\item The other way is also true: If every subset is saturated, then it is a $T_1$-space.
\begin{proof}
It is quite straightforward. $\forall A \subseteq X$, we have $X-A$ is a union of closed sets. Let $A = X - \left\{ x \right\}$ would do.
\end{proof}
\end{itemize}
\end{example}

There are obvious sets that are not saturated. Like the discrete topology of any nonempty set. 



\end{document}
