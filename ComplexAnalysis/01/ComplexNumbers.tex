\documentclass[../main.tex]{subfiles}

\begin{document}

\chapter{Complex Numbers}

\section{The Algebra of Complex Numbers}

\subsection{Arithmetic Operations}
The complex field $\mathbb{C}$ is defined to be $\mathbb{R} \times \mathbb{R}$, with $a+ib$ stands for $(a,b)$. The addition and multiplication is defined as follows.
\begin{definition}{Complex Field}{Complex Field}
The complex field $\mathbb{C}\cong \mathbb{R} \times \mathbb{R}$, with  addition and multiplication:
\begin{itemize}
\item \textbf{Addition: } $(a+ib)+(c+id) = (a+c) + (b+d)i$.
\item \textbf{Multiplication: }  $(a+ib)(c+id) = (ac-bd) + (ad+bc)i$.
\end{itemize}
\end{definition}

We can verify that $\left(\mathbb{C},+, \cdot \right)$ indeed is a field. Its additive identity being $0$ and multiplicative identity $1$.

\begin{remark}
We can also define the complex field in a more abstract way. Showing only properties of the field that has to be matched. 

Let $\mathbb{F}$ be a field. Then $\mathbb{C}$ is a subfield of $\mathbb{F}$ that matches the following properties.
\begin{itemize}
\item $\mathbb{R}$ is a subfield of $\mathbb{F}$. (isomorphic)
\item $\exists \alpha\in \mathbb{F},\alpha^2+1=0$. Let one of them be $i$.
\item $\mathbb{C}$ is the subfield generated by $\left(\mathbb{R},i\right)$.
\end{itemize}

We can prove that $\mathbb{C}$ does not depend on the choice of $\mathbb{F}$ and $i$. The existence of such a field is constructed by \ref{def:Complex Field} above.
\end{remark}

There are also isomorphic representations of the complex field. For example, 
\begin{itemize}
\item $\displaystyle \left\{ 
\begin{pmatrix}
	\alpha&\beta\\
	-\beta&\alpha
\end{pmatrix}
: \alpha,\beta\in \mathbb{R}\right\} \cong \mathbb{C}$.
\item $\mathbb{R}[x] / \left(x^2+1\right)\mathbb{R}[x] \cong \mathbb{C}$.
\end{itemize}

\subsection{Square Roots}
Now let's consider some properties that are special about the complex field. The original intuition of constructing $\mathbb{C}$ is that not all reals have square roots. We denote the square root of $-1$ to $i$ so  as to fill the blank. We now state that square root is closed under $\mathbb{C}$.
\begin{theorem}{Square Roots of Complex Numbers}{Square Roots Of Complex Numbers}
\begin{equation*}
\forall a \in \mathbb{C}, \exists b \in \mathbb{C},a=b ^2
\end{equation*}
\end{theorem}
\begin{proof}
We can explicitly find the square roots using the result that every non-negative reals have square roots.

Let $\alpha+i \beta\in \mathbb{C}$, we find 

\begin{equation*}
\left(x+iy\right)^2 = \alpha+i \beta
\end{equation*}
that is,
\begin{equation*}
\begin{aligned}
	x^2-y^2 &= \alpha\\
	2xy &= \beta
\end{aligned}
\end{equation*}
Form there we get
\begin{equation*}
\left(x^2+y^2\right)^2 = \left(x^2-y^2\right)^2 + 4x^2y^2 = \alpha^2 + \beta^2
\end{equation*}
giving
\begin{equation*}
x^2+y^2 = \sqrt{\alpha^2+\beta^2}
\end{equation*}
that is,
\begin{equation*}
\begin{cases}
x^2=\frac{1}{2}\left(\alpha + \sqrt{\alpha^2+\beta^2}\right)\\
y^2=\frac{1}{2}\left(-\alpha + \sqrt{\alpha^2+\beta^2}\right)
\end{cases}
\end{equation*}

Only two result matches, as we can see.

\end{proof} 

\subsection{Conjugation and Absolute Value}
Define $a-bi$ to be conjugation of $a+bi$. And define
\begin{equation*}
\re a = \frac{a+\overline{a}}{2}, \im a = \frac{a-\overline{a}}{2i}
\end{equation*}

\begin{theorem}{Conjugations}{Conjugations}
Let $R(a_1, \ldots ,a_n)$ be any rational operation applied to $a_1, \ldots ,a_n\in \mathbb{C}$, then
\begin{equation}
	\overline{R(a_1, \ldots ,a_n)} = R \left(\overline{a_1}, \ldots ,\overline{a_n}\right)
\end{equation}
\end{theorem}
 
Let $z = a+bi$. We denote $\left|z\right| = \sqrt{a^2+b ^2}$.

\begin{example}{Conjugates and Absolute Values}{Conjugates And Absolute Values}
\begin{itemize}
\item $\displaystyle \left|\frac{a-b}{1-\overline{a}b}\right| = 1$.
\item Lagrange's Identity:
	\begin{equation}
	\left|\sum_{i=1}^{n} a_ib_i\right|^2 = \sum_{i=1}^{n} \left|a_i\right|^2 \sum_{i=1}^{n} \left|b_i\right|^2 - \sum_{1\leq i<j\leq n} \left|a_i \overline{b_j} - a_j \overline{b_i}\right|^2 
	\end{equation}
\item $\left|a_1+\ldots +a_n\right|\leq \left|a_1\right|+\ldots +\left|a_n\right|$.
\end{itemize}
\end{example}


\section{Geometry Representation of Complex Numbers}

The most usual way to visualize complex numbers is putting it real and imaginary part into a coordinate system.

\subsection{Addition and Multiplication}
The addition of complex numbers can be visualized as vector addition.

\begin{figure}[H]
    \centering
    \incfig{complexaddition}
    \caption{Complex Addition}
    \label{fig:complexaddition}
\end{figure}

In order to derive a geometric interpretation of multiplication of complex numbers we write it into polar coordinates. If the polar coordinates of point $\left(\alpha,\beta\right)$ is $\left(r,\varphi\right)$, we know
\begin{equation*}
\begin{aligned}
\alpha = r \cos \varphi\\
\beta = r \sin \varphi
\end{aligned}
\end{equation*}

Hence we can write $z=\alpha+i \beta = r \left(\cos \varphi+\sin \varphi\right)$, which is the trigonometric form. In this form, we denote
\begin{equation*}
r = \left|z\right|, \text{ and } \varphi = \arg z
\end{equation*}

Multiplying $z_1$ and $z_2$ we get
\begin{equation*}
\begin{aligned}
	z_1z_2 &= r_1r_2 \left(\cos \varphi_1 \cos \varphi_2 - \sin \varphi_1 \sin \varphi_2 + \left(\cos \varphi_1 \sin \varphi_2 + \sin \varphi_1 \cos \varphi_2\right)i\right)\\
 &= r_1r_2 \left(\cos \left(\varphi_1 + \varphi_2\right) + i\sin \left(\varphi_1+\varphi_2\right)\right)
\end{aligned}
\end{equation*}
We see that the product has modulus $r_1r_2$ and argument $\varphi_1 + \varphi_2$. We latter result is new, for which we can express it by
\begin{equation}\label{eqt:argeq}
\arg \left(z_1z_2\right) = \arg z_1+\arg z_2
\end{equation}
By means of \ref{eqt:argeq} we can produce a geometric construction of products of complex numbers by similar triangles.
\begin{figure}[H]
    \centering
    \incfig{complex-products}
    \caption{Complex Products}
    \label{fig:complex-products}
\end{figure}
\begin{remark}
We've noticed that this definition of argument is fairly intuitive but not very stern. Angle's are not defined yet and we use trigonometric functions. However, a perfect acceptable way is to express angles by the length of an arc, which is defined by definite integral.

However, in complex analysis we do not follow that rule, because we have a more direct route: the connection between exponential functions and trigonometric functions.
\end{remark}

\subsection{Binomial Equation}

From the preceding result we derive that the powers of $a = r(\cos \varphi+i \sin \varphi)$ are given by (for $n>0$ )
\begin{equation}
a^n = r^n (\cos n \varphi + i \sin n \varphi)
\end{equation}

This equation hold trivially for $n=0$ and since 
\begin{equation*}
a^{-1} = r^{-1}(\cos \varphi-i \sin \varphi) = r^{-1}(\cos (-\varphi)+ i \sin (-\varphi))
\end{equation*}
so it holds for all $n\in \mathbb{Z}$.

To solve the $n^\text{th}$ root of a complex number $a$, we need to solve the equation
\begin{equation*}
z^n=a
\end{equation*}

Suppose $a\neq 0$ and we write $a = r(\cos \varphi + i \sin \varphi)$ and 
\begin{equation*}
z = \rho(\cos \theta + i \sin \theta)
\end{equation*}
then we have
\begin{equation*}
\rho^n(\cos n \theta + i \sin b \theta) = r(\cos \varphi+i \sin \varphi)
\end{equation*}
thus we have $\rho^n = r$ and $n \theta = \varphi + 2k \pi$.
\begin{equation*}
	z = \sqrt[n]{r} \left(\cos \left(\frac{\varphi}{n}+k \frac{2\pi}{n}\right) + i \sin \left(\frac{\varphi}{n} + k \frac{2\pi}{n}\right)\right)
\end{equation*}
where $k = 0, \ldots ,n-1$. Therefore, we get
\begin{theorem}{Root of Complex Numbers}{Root Of Complex Numbers}
There are $n$ $n^\text{th}$ root of any complex number $\neq 0$, they have the same modulus and there arguments are equally spaced
\end{theorem}

Geometrically, the $n$ $n^\text{th}$ root forms a regular $n$-polygon.
\begin{definition}{Roots of Unity}{Roots Of Unity}
The roots of $z^n=1$ are called $n^\text{th}$ root of unity. That is, if we set
\begin{equation}
\omega = \cos \frac{2\pi}{n} + i \sin \frac{2\pi}{n}
\end{equation}
then $1, \omega, \omega^2, \ldots ,\omega^{n-1}$ are the roots of unity.
\end{definition}

\subsection{Analytic Geometry}
In analytic geometry, points of a 2D plane can be represented by both $x,y$ or $z,\overline{z}$. 

For instance, the equation of a circle is $\left|z-a\right|=r$ or $(z-a)(\overline{z}-\overline{a}) = r^2$.

A straight line can be expressed as parametric equation $z = a+bt$ where $a,b\in \mathbb{C}$ and the parameter $t\in \mathbb{R}$. The angle between $z=a+bt$ and $z=a'+b't$ are just $\arg \frac{b}{b'}$.

\subsection{The Spherical Representation}
For many purposes it is useful to extent the system $\mathbb{C}$ to include the symbol $\infty $. We define $a+ \infty =\infty +a = \infty $ for all $a\in \mathbb{C}$ and $b\cdot \infty =\infty \cdot b=\infty $ for all $b\neq 0$.

In the plane there is no room for the infinity, but we can introduce an ideal point called the point at infinity. The points in the plane and the point at infinity together constructed \emph{The Extended Complex Plane}. To give a more intuitive geometric model, we consider $S^2$, which is the unit sphere in 3-dimensional space, $x_1^2+x_2^2+x_3^2=1$.

We have the 1-1 correspondence by stereographic projection.

\begin{figure}[H]
    \centering
    \incfig{stereographic-projection}
    \caption{Stereographic Projection}
    \label{fig:stereographic-projection}
\end{figure}

With every point on $S$, except $(0,0,1)$, we associate a complex number
\begin{equation*}
z = \frac{x_1+ix_2}{1-x_3}
\end{equation*}
then we have
\begin{equation*}
\left|z\right|^2 = \frac{x_1^2+x_2^2}{(1-x_3)^2} = \frac{1+x_3}{1-x_3}
\end{equation*}
hence
\begin{equation*}
x_3 = \frac{\left|z\right|^2-1}{\left|z\right|^2+1}
\end{equation*}
and further computation leads to 
\begin{equation*}
\begin{aligned}
	x_1 &= \frac{z+\overline{z}}{1+\left|z\right|^2}\\
	x_2 &= \frac{z-\overline{z}}{i(1+\left|z\right|^2)}
\end{aligned}
\end{equation*}
An we let the point $\infty $ corresponds to $(0,0,1)$.

\begin{remark}
The correspondence can be derived informally by similar triangles.
We have $x_3 = \cos 2 \theta$ and $\left|z\right| = \cot \theta$. Thus $\displaystyle  \left|z\right|^2 = \frac{\cos ^2 \theta}{\sin ^2 \theta} = \frac{1+x_3}{1-x_3}$
\begin{figure}[H]
    \centering
    \incfig{derivation-of-sphere}
    \caption{Derivation of Sphere}
    \label{fig:derivation-of-sphere}
\end{figure}
\end{remark}

It is easy to calculate the distance $d(z,z')$ between the stereographic projections of points $z$ and $z'$. Let the points are $(x_1,x_2,x_3)$ and $(x_1',x_2',x_3')$, then the distance
\begin{equation*}
	(x_1-x_1')^2+(x_2-x_2')^2+(x_3-x_3')^2 = 2-2(x_1x_1'+x_2x_2'+x_3x_3')
\end{equation*}
And we simplify by
\begin{equation*}
x_1x_1'+x_2x_2'+x_3x_3' = \frac{(1+\left|z\right|^2)(1+\left|z'\right|^2) - 2 \left|z-z'\right|^2 }{(1+\left|z\right|^2)(1+\left|z'\right|^2)}
\end{equation*}

Thus, we have
\begin{equation}
d(z,z') = \frac{2 \left|z-z'\right|}{\sqrt{(1+\left|z\right|^2)(1+\left|z'\right|^2)}}
\end{equation}

For $z' = \infty $ we have
\begin{equation}
d(z,\infty )=\frac{2}{\sqrt{1+\left|z\right|^2}}
\end{equation}


\end{document}
